\documentclass[a4paper,12pt,twoside]{memoir}

% Castellano
\usepackage[spanish,es-tabla]{babel}
\selectlanguage{spanish}
\usepackage[utf8]{inputenc}
\usepackage[T1]{fontenc}
\usepackage{lmodern} % Scalable font
\usepackage{microtype}
\usepackage{placeins}

\RequirePackage{booktabs}
\RequirePackage[table]{xcolor}
\RequirePackage{xtab}
\RequirePackage{multirow}

% Links
\PassOptionsToPackage{hyphens}{url}\usepackage[colorlinks]{hyperref}
\hypersetup{
	allcolors = {red}
}

% Ecuaciones
\usepackage{amsmath}

% Rutas de fichero / paquete
\newcommand{\ruta}[1]{{\sffamily #1}}

% Párrafos
\nonzeroparskip

% Huérfanas y viudas
\widowpenalty100000
\clubpenalty100000

% Imágenes

% Comando para insertar una imagen en un lugar concreto.
% Los parámetros son:
% 1 --> Ruta absoluta/relativa de la figura
% 2 --> Texto a pie de figura
% 3 --> Tamaño en tanto por uno relativo al ancho de página
\usepackage{graphicx}
\newcommand{\imagen}[3]{
	\begin{figure}[!h]
		\centering
		\includegraphics[width=#3\textwidth]{#1}
		\caption{#2}\label{fig:#1}
	\end{figure}
	\FloatBarrier
}

% Comando para insertar una imagen sin posición.
% Los parámetros son:
% 1 --> Ruta absoluta/relativa de la figura
% 2 --> Texto a pie de figura
% 3 --> Tamaño en tanto por uno relativo al ancho de página
\newcommand{\imagenflotante}[3]{
	\begin{figure}
		\centering
		\includegraphics[width=#3\textwidth]{#1}
		\caption{#2}\label{fig:#1}
	\end{figure}
}

% El comando \figura nos permite insertar figuras comodamente, y utilizando
% siempre el mismo formato. Los parametros son:
% 1 --> Porcentaje del ancho de página que ocupará la figura (de 0 a 1)
% 2 --> Fichero de la imagen
% 3 --> Texto a pie de imagen
% 4 --> Etiqueta (label) para referencias
% 5 --> Opciones que queramos pasarle al \includegraphics
% 6 --> Opciones de posicionamiento a pasarle a \begin{figure}
\newcommand{\figuraConPosicion}[6]{%
  \setlength{\anchoFloat}{#1\textwidth}%
  \addtolength{\anchoFloat}{-4\fboxsep}%
  \setlength{\anchoFigura}{\anchoFloat}%
  \begin{figure}[#6]
    \begin{center}%
      \Ovalbox{%
        \begin{minipage}{\anchoFloat}%
          \begin{center}%
            \includegraphics[width=\anchoFigura,#5]{#2}%
            \caption{#3}%
            \label{#4}%
          \end{center}%
        \end{minipage}
      }%
    \end{center}%
  \end{figure}%
}

%
% Comando para incluir imágenes en formato apaisado (sin marco).
\newcommand{\figuraApaisadaSinMarco}[5]{%
  \begin{figure}%
    \begin{center}%
    \includegraphics[angle=90,height=#1\textheight,#5]{#2}%
    \caption{#3}%
    \label{#4}%
    \end{center}%
  \end{figure}%
}
% Para las tablas
\newcommand{\otoprule}{\midrule [\heavyrulewidth]}
%
% Nuevo comando para tablas pequeñas (menos de una página).
\newcommand{\tablaSmall}[5]{%
 \begin{table}
  \begin{center}
   \rowcolors {2}{gray!35}{}
   \begin{tabular}{#2}
    \toprule
    #4
    \otoprule
    #5
    \bottomrule
   \end{tabular}
   \caption{#1}
   \label{tabla:#3}
  \end{center}
 \end{table}
}

%
% Nuevo comando para tablas pequeñas (menos de una página).
\newcommand{\tablaSmallSinColores}[5]{%
 \begin{table}[H]
  \begin{center}
   \begin{tabular}{#2}
    \toprule
    #4
    \otoprule
    #5
    \bottomrule
   \end{tabular}
   \caption{#1}
   \label{tabla:#3}
  \end{center}
 \end{table}
}

\newcommand{\tablaApaisadaSmall}[5]{%
\begin{landscape}
  \begin{table}
   \begin{center}
    \rowcolors {2}{gray!35}{}
    \begin{tabular}{#2}
     \toprule
     #4
     \otoprule
     #5
     \bottomrule
    \end{tabular}
    \caption{#1}
    \label{tabla:#3}
   \end{center}
  \end{table}
\end{landscape}
}

%
% Nuevo comando para tablas grandes con cabecera y filas alternas coloreadas en gris.
\newcommand{\tabla}[6]{%
  \begin{center}
    \tablefirsthead{
      \toprule
      #5
      \otoprule
    }
    \tablehead{
      \multicolumn{#3}{l}{\small\sl continúa desde la página anterior}\\
      \toprule
      #5
      \otoprule
    }
    \tabletail{
      \hline
      \multicolumn{#3}{r}{\small\sl continúa en la página siguiente}\\
    }
    \tablelasttail{
      \hline
    }
    \bottomcaption{#1}
    \rowcolors {2}{gray!35}{}
    \begin{xtabular}{#2}
      #6
      \bottomrule
    \end{xtabular}
    \label{tabla:#4}
  \end{center}
}

%
% Nuevo comando para tablas grandes con cabecera.
\newcommand{\tablaSinColores}[6]{%
  \begin{center}
    \tablefirsthead{
      \toprule
      #5
      \otoprule
    }
    \tablehead{
      \multicolumn{#3}{l}{\small\sl continúa desde la página anterior}\\
      \toprule
      #5
      \otoprule
    }
    \tabletail{
      \hline
      \multicolumn{#3}{r}{\small\sl continúa en la página siguiente}\\
    }
    \tablelasttail{
      \hline
    }
    \bottomcaption{#1}
    \begin{xtabular}{#2}
      #6
      \bottomrule
    \end{xtabular}
    \label{tabla:#4}
  \end{center}
}

%
% Nuevo comando para tablas grandes sin cabecera.
\newcommand{\tablaSinCabecera}[5]{%
  \begin{center}
    \tablefirsthead{
      \toprule
    }
    \tablehead{
      \multicolumn{#3}{l}{\small\sl continúa desde la página anterior}\\
      \hline
    }
    \tabletail{
      \hline
      \multicolumn{#3}{r}{\small\sl continúa en la página siguiente}\\
    }
    \tablelasttail{
      \hline
    }
    \bottomcaption{#1}
  \begin{xtabular}{#2}
    #5
   \bottomrule
  \end{xtabular}
  \label{tabla:#4}
  \end{center}
}



\definecolor{cgoLight}{HTML}{EEEEEE}
\definecolor{cgoExtralight}{HTML}{FFFFFF}

%
% Nuevo comando para tablas grandes sin cabecera.
\newcommand{\tablaSinCabeceraConBandas}[5]{%
  \begin{center}
    \tablefirsthead{
      \toprule
    }
    \tablehead{
      \multicolumn{#3}{l}{\small\sl continúa desde la página anterior}\\
      \hline
    }
    \tabletail{
      \hline
      \multicolumn{#3}{r}{\small\sl continúa en la página siguiente}\\
    }
    \tablelasttail{
      \hline
    }
    \bottomcaption{#1}
    \rowcolors[]{1}{cgoExtralight}{cgoLight}

  \begin{xtabular}{#2}
    #5
   \bottomrule
  \end{xtabular}
  \label{tabla:#4}
  \end{center}
}



\graphicspath{ {./img/} }

% Capítulos
\chapterstyle{bianchi}
\newcommand{\capitulo}[2]{
	\setcounter{chapter}{#1}
	\setcounter{section}{0}
	\setcounter{figure}{0}
	\setcounter{table}{0}
	\chapter*{\thechapter.\enskip #2}
	\addcontentsline{toc}{chapter}{\thechapter.\enskip #2}
	\markboth{#2}{#2}
}

% Apéndices
\renewcommand{\appendixname}{Apéndice}
\renewcommand*\cftappendixname{\appendixname}

\newcommand{\apendice}[1]{
	%\renewcommand{\thechapter}{A}
	\chapter{#1}
}

\renewcommand*\cftappendixname{\appendixname\ }

% Formato de portada
\makeatletter
\usepackage{xcolor}
\newcommand{\tutor}[1]{\def\@tutor{#1}}
\newcommand{\course}[1]{\def\@course{#1}}
\definecolor{cpardoBox}{HTML}{E6E6FF}
\def\maketitle{
  \null
  \thispagestyle{empty}
  % Cabecera ----------------
\noindent\includegraphics[width=\textwidth]{cabecera}\vspace{1cm}%
  \vfill
  % Título proyecto y escudo informática ----------------
  \colorbox{cpardoBox}{%
    \begin{minipage}{.8\textwidth}
      \vspace{.5cm}\Large
      \begin{center}
      \textbf{TFG del Grado en Ingeniería Informática}\vspace{.6cm}\\
      \textbf{\LARGE\@title{}}
      \end{center}
      \vspace{.2cm}
    \end{minipage}

  }%
  \hfill\begin{minipage}{.20\textwidth}
    \includegraphics[width=\textwidth]{escudoInfor}
  \end{minipage}
  \vfill
  % Datos de alumno, curso y tutores ------------------
  \begin{center}%
  {%
    \noindent\LARGE
    Presentado por \@author{}\\ 
    en Universidad de Burgos --- \@date{}\\
    Tutor: \@tutor{}\\
  }%
  \end{center}%
  \null
  \cleardoublepage
  }
\makeatother

\newcommand{\nombre}{Daniel Alonso Báscones} %%% cambio de comando

% Datos de portada
\title{Vulnerabilidad de redes de paquetes software II}
\author{\nombre}
\tutor{Carlos López Nozal}
\date{\today}

\begin{document}

\maketitle


\newpage\null\thispagestyle{empty}\newpage


%%%%%%%%%%%%%%%%%%%%%%%%%%%%%%%%%%%%%%%%%%%%%%%%%%%%%%%%%%%%%%%%%%%%%%%%%%%%%%%%%%%%%%%%
\thispagestyle{empty}


\noindent\includegraphics[width=\textwidth]{cabecera}\vspace{1cm}

\noindent D. Carlos López Nozal, profesor del departamento de Ingeniería Informática, área de Lenguajes y Sistemas Informáticos.

\noindent Expone:

\noindent Que el alumno D. \nombre, con DNI 71298886J, ha realizado el Trabajo final de Grado en Ingeniería Informática titulado Vulnerabilidad de redes de
paquetes software II.

\noindent Y que dicho trabajo ha sido realizado por el alumno bajo la dirección del que suscribe, en virtud de lo cual se autoriza su presentación y defensa.

\begin{center} %\large
  En Burgos, {\large \today}
\end{center}

\vfill\vfill\vfill

% Author and supervisor
% \begin{minipage}{0.45\textwidth}
% \begin{flushleft} %\large
% Vº. Bº. del Tutor:\\[2cm]
% D. Carlos López Nozal
% \end{flushleft}
% \end{minipage}
% \hfill
% \begin{minipage}{0.45\textwidth}
% \begin{flushleft} %\large
% Vº. Bº. del co-tutor:\\[2cm]
% D. José Ignacio Santos Martín
% \end{flushleft}
% \end{minipage}
% \hfill

\vfill

% para casos con sólo un tutor comentar lo anterior
% y descomentar lo siguiente
Vº. Bº. del Tutor:\\[2cm]
D. Carlos López Nozal


\newpage\null\thispagestyle{empty}\newpage




\frontmatter

% Abstract en castellano
\renewcommand*\abstractname{Resumen}
\begin{abstract}
  En la gran mayoría de los proyectos de desarrollo de software, es común utilizar bibliotecas de repositorios centralizados de paquetes. Esta práctica se lleva a cabo con el objetivo fundamental de reducir los tiempos de desarrollo y los costos asociados. Sin embargo, debido a la transitividad de las dependencias entre los paquetes, la presencia de un sólo defecto en el repositorio puede tener efectos extensos y difíciles de predecir en el conjunto de paquetes. Estos defectos pueden ocasionar errores funcionales, problemas de rendimiento o vulnerabilidades de seguridad. El riesgo asociado a estos defectos es difícil de apreciar por parte de los desarrolladores, quienes sólo importan explícitamente una pequeña fracción de las dependencias.

  En un trabajo previo, basado en la teoría de la ciencia de redes, se propuso un modelo de dependencias de paquetes de software utilizando un conjunto de datos público obtenido de \textit{Libraries.io}, generado entre los años 2020 y 2021. \textit{Libraries.io} indexa información de 7,352,165 paquetes provenientes de 32 gestores de paquetes. La versión inicial de este trabajo de grado permitió analizar las redes de paquetes \textit{PyPI}, \textit{Maven} y \textit{npm} desde la perspectiva de las vulnerabilidades y la inmunización.
  
  El objetivo de este proyecto es estudiar e implementar técnicas de \textit{webscraping} y acceso a \textit{Web API's} para generar conjuntos de datos actualizados de las redes de paquetes software de \textit{PyPI}, \textit{npm}, \textit{CRAN} y \textit{Bioconductor}. Los resultados de este proyecto incluyen la biblioteca \textit{Olivia finder} de Python, así como \textit{notebooks} de uso, conjuntos de datos actualizados y nuevos (incluyendo \textit{Bioconductor}), y el análisis de la evolución de los sistemas de paquetes desde su publicación en \textit{Libraries.io}. Además, como una aplicación práctica, se proporciona un análisis de las dependencias transitivas de un proyecto de \textit{Github}.
  
  A través del análisis de la evolución temporal de las redes de dependencias en \textit{PyPI}, se observa un crecimiento sustancial de la red. En conclusión, se recomienda que todos los trabajos que se basen en \textit{Libraries.io} repliquen sus análisis utilizando los nuevos conjuntos de datos, a fin de confirmar sus resultados.
\end{abstract}

\renewcommand*\abstractname{Descriptores}
\begin{abstract}
  Ciencia de redes,
  Vulnerabilidad de red,
  Grafo de dependencias,
  Repositorio de paquetes,
  npm,
  PyPI,
  CRAN,
  Bioconductor,
  Web Scraping.
\end{abstract}


\clearpage

% Abstract en inglés
\renewcommand*\abstractname{Abstract}
\begin{abstract}
  In the vast majority of software development projects, it is common to use libraries from centralized package repositories. This practice is carried out with the fundamental goal of reducing development time and costs. However, due to the transitivity of dependencies between packages, the presence of a single defect in the repository can have extensive and unpredictable effects on the package ecosystem. These defects can lead to \textit{functional errors}, \textit{performance issues}, or \textit{security vulnerabilities}. The risk associated with these defects is difficult for developers to appreciate, as they only explicitly import a small fraction of the dependencies.

  In a previous work, based on \textit{network science theory}, a model of software package dependencies was proposed using publicly available data obtained from \textit{Libraries.io}, generated between the years 2020 and 2021. \textit{Libraries.io} indexes information from 7,352,165 packages from 32 package managers. The initial version of this thesis work allowed for the analysis of package networks such as \textit{PyPI}, \textit{Maven}, and \textit{npm} from the perspective of \textit{vulnerability} and \textit{immunization}.
  
  The aim of this project is to study and implement techniques such as \textit{webscraping} and access to \textit{Web APIs} to generate up-to-date datasets of software package networks: \textit{PyPI}, \textit{npm}, \textit{CRAN}, and \textit{Bioconductor}. The outcomes of this project include the Python library "\textit{Olivia finder}," usage notebooks, updated and new datasets (including \textit{Bioconductor}), and the analysis of package system evolution since their publication on \textit{Libraries.io}. Additionally, as a practical application, an analysis of \textit{transitive dependencies} in a \textit{GitHub} project is provided.
  
  Through the analysis of the temporal evolution of \textit{PyPI's dependency networks}, substantial growth of the network is observed. In conclusion, it is recommended that all works relying on \textit{Libraries.io} replicate their analyses using the new datasets to confirm their results.
\end{abstract}

\renewcommand*\abstractname{Keywords}
\begin{abstract}
  Network science,
  Network vulnerability,
  Dependency graph,
  Package repository,
  npm,
  PyPI,
  CRAN,
  Bioconductor,
  Web scraping.
\end{abstract}

\clearpage

% Indices
\tableofcontents

\clearpage

\listoffigures

\clearpage

\listoftables
\clearpage

\mainmatter
\capitulo{1}{Introducción}

\section{Las redes y la teoría de grafos}

Las redes, como estructuras que representan interconexiones entre entidades, son objeto de estudio de la teoría de grafos. Estas redes se componen de nodos o vértices conectados por enlaces o aristas. La teoría de grafos, por su parte, se encarga de analizar matemáticamente estas redes y proporcionar herramientas para comprender sus propiedades y comportamientos.

Los grafos son modelos abstractos que representan las relaciones y conexiones entre entidades mediante nodos y aristas. Estos modelos tienen aplicaciones en diversos campos, como la informática, la física, la biología y las ciencias sociales. Se utilizan para comprender fenómenos complejos, analizar la propagación de información, estudiar interacciones sociales y examinar las redes de transporte, entre otros aspectos.

En el campo de la teoría de grafos, destacan importantes autores que han contribuido significativamente a su desarrollo. Leonhard Euler es considerado el fundador de la teoría de grafos, gracias a su trabajo sobre el problema de los puentes de Königsberg en el siglo XVIII. Otros destacados autores incluyen a Paul Erdős, quien realizó contribuciones fundamentales a la teoría de grafos combinatorios, y Claude Shannon, quien aplicó la teoría de grafos en la teoría de la información. Estos investigadores han sentado las bases para el estudio y aplicación de la teoría de grafos en diversas disciplinas.

\section{Los repositorios de paquetes de software}

En el ámbito del desarrollo de software, los repositorios de paquetes desempeñan un papel fundamental al ofrecer un entorno centralizado donde los desarrolladores pueden acceder, compartir y distribuir bibliotecas de código predefinidas. Estos repositorios están específicamente diseñados para diferentes plataformas y lenguajes de programación, proporcionando a los desarrolladores un acceso conveniente a una amplia gama de recursos.

A lo largo de los años, han surgido numerosos repositorios de paquetes de software para diversas plataformas y lenguajes. Por ejemplo, en el ámbito de la bioinformática, Bioconductor destaca como un repositorio importante que se centra en paquetes y herramientas para el análisis de datos genómicos. En el ecosistema de Python, PyPI (Python Package Index) es un repositorio central que alberga una gran cantidad de paquetes para una amplia variedad de aplicaciones y bibliotecas.

En el panorama actual, los repositorios de paquetes de software siguen siendo vitales para la comunidad de desarrollo. Proporcionan a los desarrolladores un acceso rápido y sencillo a una amplia gama de funcionalidades y bibliotecas de código predefinidas, lo que les permite acelerar el desarrollo de aplicaciones y proyectos. Además, estos repositorios fomentan la colaboración y el intercambio de código entre los desarrolladores, promoviendo un ecosistema de desarrollo más dinámico y eficiente.

\section{Los gestores de paquetes de software}

Los gestores de paquetes de software nos proporcionan herramientas y funcionalidades para gestionar la instalación, actualización y eliminación de bibliotecas y dependencias de un proyecto. Estos gestores se encuentran diseñados específicamente para diferentes plataformas y lenguajes de programación, brindando a los desarrolladores una forma eficiente de administrar y distribuir el código.

Entre los gestores de paquetes más populares, se destaca pip en el ecosistema de Python, que permite instalar y administrar fácilmente las bibliotecas necesarias para un proyecto Python. Por otro lado, mvn (Maven) es ampliamente utilizado en el mundo de Java para gestionar las dependencias y configuraciones de proyectos. Cada gestor de paquetes cuenta con su propia sintaxis y funcionalidades específicas, pero todos comparten el objetivo común de simplificar la gestión de bibliotecas y asegurar la resolución de dependencias.

En el panorama actual, los gestores de paquetes de software continúan desempeñando un papel crucial en el desarrollo de software. Proporcionan a los desarrolladores una forma conveniente y eficiente de administrar las bibliotecas y dependencias necesarias para sus proyectos, lo que les permite centrarse en la implementación de funcionalidades sin preocuparse por la instalación manual y la gestión de las dependencias.

Además, estos gestores mejoran la reutilización de código de la comunidad y la colaboración, ya que permiten compartir y distribuir fácilmente bibliotecas y proyectos entre desarrolladores. También facilitan la actualización y el mantenimiento de las dependencias, asegurando que los proyectos estén siempre actualizados y protegidos contra vulnerabilidades conocidas.

\section{Las redes de dependencias de paquetes de software}

Una red de dependencias de paquetes software es un conjunto de conexiones entre diferentes paquetes o bibliotecas utilizadas en el desarrollo de software. Cada paquete depende de otros para funcionar correctamente, formando una red interdependiente.

En este contexto, cada paquete tiene requisitos específicos que deben cumplirse para que funcione correctamente. Estos requisitos son satisfechos por otros paquetes, creando conexiones que permiten que el software se comporte como se espera. Si alguna de estas conexiones se rompe o no se cumple, puede producirse un fallo en el funcionamiento del software.

La red de dependencias puede ser muy compleja, ya que un solo paquete puede depender de múltiples otros paquetes, y estos, a su vez, pueden tener sus propias dependencias. Es como un tejido intrincado en el que cada hilo está conectado con otros, formando una red interconectada.

\section{Los repositorios de paquetes de software desde la perspectiva de la teoría de grafos}

Al estudiar las dependencias de paquetes como grafos, obtenemos una visión más clara de las complejas relaciones que existen en nuestros proyectos. Podemos identificar los paquetes esenciales que sostienen todo el sistema y comprender cómo un cambio en uno de ellos puede afectar a otros. Esto nos ayuda a tomar decisiones informadas y anticiparnos a posibles problemas.

Además, la teoría de grafos nos permite detectar dependencias redundantes o ciclos indeseados en nuestra red de paquetes. Podemos evaluar la estabilidad y mantenibilidad del proyecto, analizando el impacto de agregar o eliminar un paquete en toda la red. Esta comprensión más profunda nos ayuda a optimizar y fortalecer nuestro código.

Afortunadamente existen herramientas y bibliotecas que nos proporcionan esta informacion de las redes. Con ellas, y usando el enfoque de la teoria de grafos, podemos ver de una forma más clara cómo se entrelazan los paquetes entre sí y cómo se relacionan con el resto de la red.

La aplicación de la teoría de grafos a las redes de dependencias de paquetes software nos permite tomar decisiones más informadas, comprender mejor las implicaciones de las dependencias y colaborar de manera más efectiva con nuestro equipo de desarrollo. Es una forma poderosa de optimizar y mejorar nuestros proyectos y mantener un ecosistema saludable en los gestores de paquetes.

\section{OLIVIA: Open-source Library Indexes Vulnerability Identification and Analysis}

OLIVIA, desarrollada por el alumno Daniel Setó Rey como parte de su Trabajo de Fin de Grado en la Universidad de Burgos y tutorizada por los profesores Carlos López Nozal y Jose Ignacio Santos Martín en 2021, es una herramienta de código abierto que se centra en la identificación y análisis de defectos en bibliotecas de software desde la perspectiva de la teoría de grafos. Estos defectos pueden provocar errores funcionales, problemas de rendimiento e incluso problemas de seguridad. Para los desarrolladores, comprender completamente el riesgo es complicado, ya que solo importan explícitamente una pequeña parte de las dependencias utilizadas en sus proyectos.

OLIVIA utiliza un enfoque basado en la vulnerabilidad de la red de dependencias de los paquetes de software para medir la sensibilidad del repositorio a la introducción aleatoria de defectos. Su objetivo es contribuir a la comprensión de los mecanismos de propagación de defectos en el software y estudiar estrategias factibles de protección.

En la actualidad, OLIVIA está en proceso de ser publicado a nivel académico. Después de su desarrollo como proyecto de fin de carrera, se están realizando los esfuerzos necesarios para compartir sus resultados y contribuciones con la comunidad científica. Esta publicación permitirá que otros investigadores y profesionales del campo accedan a esta herramienta y se beneficien de su enfoque innovador en la identificación y análisis de vulnerabilidades en las bibliotecas de software. Además, sentará las bases para futuros avances en la comprensión de los mecanismos de propagación de defectos y la implementación de estrategias efectivas de protección en el desarrollo de software.

En este trabajo de fin de grado, no se profundiza en el modelo matemático subyacente de OLIVIA. Sin embargo, es importante tener conocimientos básicos de la teoría de grafos para comprender su funcionalidad. La teoría de grafos proporciona un marco conceptual para comprender las interconexiones y relaciones entre los paquetes de software, así como la propagación de posibles defectos en el ecosistema. Aunque no se abordan los aspectos matemáticos en detalle, tener una comprensión general de los grafos nos ayuda a apreciar y comprender mejor la utilidad y las implicaciones de OLIVIA en la identificación y análisis de vulnerabilidades en las bibliotecas de software.
\capitulo{2}{Objetivos del proyecto}

\begin{itemize}
    \item Realizar un análisis de la viabilidad de las diferentes estrategias de extracción de datos de los repositorios de paquetes, teniendo en cuenta las restricciones y particularidades de cada plataforma. Consideramos aspectos como la disponibilidad de datos, las políticas de acceso y las limitaciones técnicas para determinar la mejor manera de obtener y procesar la información requerida.
    \item Publicar los datos obtenidos con el propósito de que sirvan como referencia y recurso para futuras investigaciones en el campo, fomentando la colaboración en la comunidad científica, permitiendo a otros investigadores utilizarlos como base para nuevos análisis y descubrimientos.
    \item Realizar un análisis de las principales métricas de teoría de grafos utilizando el conjunto de datos disponible para comparar la evolución de los paquetes y evaluar el estado de los repositorios a lo largo del tiempo.
    \item Obtener las métricas propuestas por OLIVIA para el nuevo conjunto de datos y su comparación con los datos expuestos en el Trabajo de Fin de Grado anterior.
    \end{itemize}
\capitulo{3}{Conceptos teóricos}

\section{Las redes y la teoría de grafos}

Las redes, como estructuras que representan interconexiones entre entidades, son objeto de estudio de la teoría de grafos. Estas redes se componen de nodos o vértices conectados por enlaces o aristas. La teoría de grafos, por su parte, se encarga de analizar matemáticamente estas redes y proporcionar herramientas para comprender sus propiedades y comportamientos.

Los grafos son modelos abstractos que representan las relaciones y conexiones entre entidades mediante nodos y aristas. Estos modelos tienen aplicaciones en diversos campos, como la informática, la física, la biología y las ciencias sociales. Se utilizan para comprender fenómenos complejos, analizar la propagación de información, estudiar interacciones sociales y examinar las redes de transporte, entre otros aspectos.

En el campo de la teoría de grafos, destacan importantes autores que han contribuido significativamente a su desarrollo. Leonhard Euler es considerado el fundador de la teoría de grafos, gracias a su trabajo sobre el problema de los puentes de Königsberg en el siglo XVIII. Otros destacados autores incluyen a Paul Erdős, quien realizó contribuciones fundamentales a la teoría de grafos combinatorios, y Claude Shannon, quien aplicó la teoría de grafos en la teoría de la información. Estos investigadores han sentado las bases para el estudio y aplicación de la teoría de grafos en diversas disciplinas.

\section{Los repositorios de paquetes de software}

En el ámbito del desarrollo de software, los repositorios de paquetes desempeñan un papel fundamental al ofrecer un entorno centralizado donde los desarrolladores pueden acceder, compartir y distribuir bibliotecas de código predefinidas. Estos repositorios están específicamente diseñados para diferentes plataformas y lenguajes de programación, proporcionando a los desarrolladores un acceso conveniente a una amplia gama de recursos.

A lo largo de los años, han surgido numerosos repositorios de paquetes de software para diversas plataformas y lenguajes. Por ejemplo, en el ámbito de la bioinformática, Bioconductor destaca como un repositorio importante que se centra en paquetes y herramientas para el análisis de datos genómicos. En el ecosistema de Python, PyPI (Python Package Index) es un repositorio central que alberga una gran cantidad de paquetes para una amplia variedad de aplicaciones y bibliotecas.

En el panorama actual, los repositorios de paquetes de software siguen siendo vitales para la comunidad de desarrollo. Proporcionan a los desarrolladores un acceso rápido y sencillo a una amplia gama de funcionalidades y bibliotecas de código predefinidas, lo que les permite acelerar el desarrollo de aplicaciones y proyectos. Además, estos repositorios fomentan la colaboración y el intercambio de código entre los desarrolladores, promoviendo un ecosistema de desarrollo más dinámico y eficiente.

\section{Los gestores de paquetes de software}

Los gestores de paquetes de software nos proporcionan herramientas y funcionalidades para gestionar la instalación, actualización y eliminación de bibliotecas y dependencias de un proyecto. Estos gestores se encuentran diseñados específicamente para diferentes plataformas y lenguajes de programación, brindando a los desarrolladores una forma eficiente de administrar y distribuir el código.

Entre los gestores de paquetes más populares, se destaca pip en el ecosistema de Python, que permite instalar y administrar fácilmente las bibliotecas necesarias para un proyecto Python. Por otro lado, mvn (Maven) es ampliamente utilizado en el mundo de Java para gestionar las dependencias y configuraciones de proyectos. Cada gestor de paquetes cuenta con su propia sintaxis y funcionalidades específicas, pero todos comparten el objetivo común de simplificar la gestión de bibliotecas y asegurar la resolución de dependencias.

En el panorama actual, los gestores de paquetes de software continúan desempeñando un papel crucial en el desarrollo de software. Proporcionan a los desarrolladores una forma conveniente y eficiente de administrar las bibliotecas y dependencias necesarias para sus proyectos, lo que les permite centrarse en la implementación de funcionalidades sin preocuparse por la instalación manual y la gestión de las dependencias.

Además, estos gestores mejoran la reutilización de código de la comunidad y la colaboración, ya que permiten compartir y distribuir fácilmente bibliotecas y proyectos entre desarrolladores. También facilitan la actualización y el mantenimiento de las dependencias, asegurando que los proyectos estén siempre actualizados y protegidos contra vulnerabilidades conocidas.

\section{Redes de dependencias de paquetes de software}

Una red de dependencias de paquetes de software consiste en un grafo en el que se representan las relaciones de dependencias entre paquetes de software. En una red de dependencias, cada nodo representa un paquete de software, y cada enlace representa una relacion de dependencia entre dos paquetes. Existe una dependencia entre dos paquetes si un paquete requiere del otro para funcionar.

\subsection{Importancia de las redes de dependencias de paquetes de software}
En primer lugar, estas redes se pueden utilizar para identificar posibles problemas en proyectos de software, ya que cuando se actualiza un paquete, puede introducir cambios incompatibles que afecten a otros paquetes en la red.

Desde un punto de vista de control de calidad, las redes de dependencias de paquetes de software se pueden utilizar para mejorar la calidad de los proyectos de software. Por ejemplo, al analizar las dependencias entre paquetes, los desarrolladores pueden identificar áreas potenciales de mejora. Por ejemplo, pueden identificar paquetes que ya no son necesarios o que están causando problemas.

Además cabe destacar que se pueden utilizar para hacer que los proyectos de software sean más seguros, ya que, al analizar las dependencias entre paquetes, se pueden identificar posibles vulnerabilidades de seguridad o que son utilizados con frecuencia con fines maliciosos.

Por último, permiten gestionar eficientemente las actualizaciones de software. Al comprender cómo los cambios en un paquete pueden afectar a otros, los equipos de desarrollo pueden evaluar el impacto potencial de las actualizaciones y tomar decisiones informadas sobre cuándo y cómo implementarlas.

\subsection{Necesidades de las redes de dependencias de paquetes de software}

\subsubsection{Recopilacion de datos}

La recopilación de datos sobre las dependencias de los paquetes de software es un desafío complejo que puede resultar difícil de abordar. 
Uno de los principales problemas radica en la falta de una única fuente confiable de información sobre estas dependencias. En muchos casos, no existe un repositorio centralizado o una base de datos completa que contenga todos los detalles necesarios.

Debido a esta falta de una fuente única, recopilar datos sobre las dependencias de los paquetes de software a menudo requiere un esfuerzo manual y exhaustivo. Los desarrolladores y analistas deben investigar y rastrear las dependencias de cada paquete individualmente, lo que puede llevar una cantidad considerable de tiempo y recursos.
Además, la información sobre las dependencias de los paquetes de software puede dispersarse en diferentes fuentes, como documentación oficial, repositorios de código, foros de desarrolladores y otras fuentes en línea. Esta dispersión puede dificultar aún más la recopilación de datos y aumentar la posibilidad de omitir o malinterpretar información relevante.

Otro desafío asociado con la recopilación de datos es mantener la información actualizada. Las dependencias de los paquetes de software pueden cambiar con el tiempo debido a actualizaciones, nuevas versiones o cambios en los requisitos del sistema. Por lo tanto, es crucial realizar un seguimiento constante de los cambios y actualizar la información de las dependencias de manera regular para garantizar la precisión de los datos recopilados.
Para abordar estos desafíos, se han desarrollado herramientas y técnicas específicas. Algunas soluciones automatizadas, como analizadores de dependencias y herramientas de gestión de paquetes, pueden ayudar a simplificar el proceso de recopilación de datos al extraer automáticamente información sobre las dependencias de los paquetes de software. Sin embargo, incluso con estas herramientas, es posible que se requiera intervención manual para verificar y corregir posibles discrepancias o lagunas en los datos recopilados

\subsubsection{Análisis de datos}

El análisis de las redes de dependencias de paquetes de software es un proceso complejo debido a la naturaleza de estas redes, que pueden ser enormes y altamente interconectadas. Estas redes pueden contener miles o incluso millones de nodos y conexiones, lo que hace que el análisis manual sea prácticamente imposible. Por lo tanto, se requieren herramientas especializadas y experiencia en análisis de datos para abordar este desafío.

Una de las principales dificultades del análisis de datos en las redes de dependencias de paquetes de software es ser capaz de poder trabajar con la masividad de estas y poder visualizar como son. Dado que las redes pueden ser muy grandes, es esencial utilizar herramientas de visualización que permitan comprender la estructura y las interconexiones de manera más sencilla. Estas herramientas ayudan a identificar patrones, clusters y dependencias importantes dentro de la red.

Además, la complejidad de las redes de dependencias también implica la necesidad de algoritmos y técnicas de análisis avanzados. Estos algoritmos pueden abordar desafíos como la detección de comunidades de paquetes, la identificación de paquetes críticos o de alto impacto, la detección de ciclos o bucles de dependencia y la identificación de flujos de información crítica.

Si nos enfocamos en el punto de vista de calidad y robustez de la red, es necesario medir la estabilidad de las dependencias y evaluar los posibles puntos de falla o vulnerabilidades en la red. Estas métricas nos permiten identificar posibles puntos débiles en la infraestructura de software.

\subsubsection{Actualizaciones y volatilidad}

Las actualizaciones del código fuente son una parte crucial y muy frecuente, ya que proporcionan mejoras, correcciones de errores y nuevas funcionalidades. Sin embargo, estas actualizaciones también pueden tener un impacto significativo en las dependencias entre paquetes de software.

Cuando se realiza una actualización en un paquete de software, es posible que se modifiquen los requisitos o las dependencias del mismo. Por ejemplo, una actualización puede introducir una nueva versión de una biblioteca o un componente, lo que podría requerir que otros paquetes también se actualicen para mantener la compatibilidad. Esto puede afectar a las conexiones existentes en la red de dependencias de paquetes de software.

Es fundamental que las redes de dependencias se actualicen regularmente para reflejar estos cambios. Esto implica realizar un seguimiento de las actualizaciones de cada paquete y revisar que todo funcione como se espera. Sería conveniente y una buena práctica para los mantenedores de la red, que esta esté alineada con las versiones y requisitos actualizados de los paquetes individuales.

Como consecuencia de  mantener las redes actualizadas, se asegura que los desarrolladores tengan una visión más cercana a la evolución tecnológica del software lo que facilita la toma de decisiones informadas sobre futuras actualizaciones y mejoras en un determinado proyecto.

\section{Analisis de redes de dependencias de paquetes de software}

El análisis de redes es una disciplina que se enfoca en el estudio y comprensión de la estructura, interconexiones y propiedades de los sistemas complejos representados como redes. Estas redes pueden ser cualquier tipo de sistema compuesto por elementos interconectados (como redes sociales, carreteras, red eléctrica, o como en nuestro caso redes de dependencias de paquetes de software)

En el contexto de las redes de dependencias de paquetes de software, realizamos este análisis por medio de métricas. Las métricas son medidas cuantitativas que se utilizan para evaluar y describir diferentes aspectos de la red. Estas métricas proporcionan información clave sobre la importancia, la centralidad y las características de los nodos y las conexiones en la red.

Algunas de las métricas comunes en el análisis de redes de dependencias de paquetes de software incluyen el grado, que indica el número de conexiones que tiene un nodo. La centralidad de intermediación, que mide cuánto un nodo se encuentra en los caminos más cortos entre otros nodos. La centralidad de cercanía, que evalúa la distancia promedio entre un nodo y los demás nodos de la red. La centralidad de vector propio, que mide la importancia de un nodo basándose en la importancia de sus vecinos.

\subsection{El grado}

El grado de un nodo en una red es el número de aristas conectadas a ese nodo. 
En una red de dependencias de paquetes de software, el grado de un nodo representa el número de paquetes que dependen de ese paquete junto con las dependencias de este.
En redes dirigidas se hace diferencia entre grado de entrada y grado de salida.

Un alto grado en la red de dependencias de paquetes de software revela el nivel de emparejamiento que posee un determinado paquete. Esta circunstancia puede tener consecuencias tanto positivas como negativas.
Podría considerarse un indicio alentador cuando el paquete en cuestión ha sido rigurosamente probado y goza de una reputación fiable en términos de su desempeño. No obstante, también es importante reconocer que un alto grado puede encerrar ciertos riesgos, especialmente si el paquete se caracteriza por su complejidad o la presencia de errores.

La presencia de un alto grado de entrada en un paquete sugiere que una gran cantidad de otros paquetes dependen de él. Esta situación puede interpretarse como un signo de popularidad y amplio uso en el entorno del software. Cuando un paquete ha demostrado ser confiable, está bien mantenido y ha sido sometido a pruebas exhaustivas, su alto grado se convierte en una señal de que ha ganado la confianza de los desarrolladores y es considerado una opción sólida para satisfacer diversas necesidades funcionales.

Sin embargo, se debe considerar el riesgo potencial asociado a un alto grado de salida. Existe la posibilidad de que un alto grado de salida indique complejidad o presencia de errores en el paquete. Si un paquete es altamente complejo, es probable que su comprensión y mantenimiento sean una tarea dificil, lo que podría derivar en problemas de rendimiento, escalabilidad o incluso fallos en el sistema. Asimismo, si un paquete presenta errores o fallas, su alta dependencia implica que los problemas pueden propagarse rápidamente a otros paquetes, lo que afecta negativamente la estabilidad y confiabilidad del sistema en su conjunto.

Por lo tanto, resulta crucial considerar tanto el grado de un paquete como su calidad y confiabilidad. Un alto grado por sí solo no garantiza la calidad o idoneidad de un paquete, sino que debe evaluarse en conjunto con otros factores, como la presencia de test de funcionalidades, validación y pruebas, la existencia de documentación, el mantenimiento por parte de los desarrolladores y el feedback de la comunidad de desarrollo. 

\subsection{Centralidad de intermediación (Betweenness centrality)}

La centralidad de intermediación mide la importancia de un nodo en una red según la frecuencia con la que se encuentra en el camino más corto entre otros dos nodos. En una red de dependencias de paquetes de software, la centralidad de intermediación se puede utilizar para identificar paquetes que son críticos para el funcionamiento general del sistema.

Esto implica que paquetes con alta centralidad de intermediación son puntos clave intermedios entre los paquetes de ese sistema. Identificar estos paquetes críticos puede ayudar a los desarrolladores a entender dónde se encuentran potenciales problemas intermedios o cuellos de botella y dónde pueden surgir problemas si esos paquetes fallan o se ven afectados.

Es por eso que debemos ser proactivos y estar alerta cuando se trata de este paquete con alta centralidad de intermediación. Es necesario establecer medidas de seguridad sólidas para protegerlo y garantizar su integridad. Esto implica llevar a cabo una supervisión constante, identificar posibles vulnerabilidades y aplicar actualizaciones y parches de seguridad de manera oportuna si fuese necesario para garantizar la calidad de la red.

\subsection{Centralidad de cercanía (Closeness centrality)}

La centralidad de cercanía mide la distancia media entre un nodo y todos los demás nodos de la red. En nuestro caso de estudio, la centralidad de cercanía se puede utilizar para identificar paquetes que son fácilmente accesibles desde otros paquetes.

Los paquetes con alta centralidad de cercanía son aquellos que están cerca de muchos otros paquetes en términos de distancia. Esto significa que son habitualmente usados por otros paquetes y pueden tener un impacto significativo en la propagación de información o cambios a través de la red de dependencias. Identificar estos paquetes puede ayudar a los desarrolladores a comprender dónde se encuentran los puntos clave de nexo común entre otros paquetes y cómo se propagan las dependencias en el sistema.

Como consecuencia, un paquete con alta cercanía debería de estar bien testado y documentado, ya que es fácilmente posible que sea dependencia en un proyecto software.

\subsection{Centralidad de vector propio (Eigenvector centrality)}

La centralidad de vector propio mide la importancia de un nodo en una red según la importancia de sus vecinos. En una red de dependencias de paquetes de software, la centralidad de vector propio se puede utilizar para identificar paquetes influyentes en la red.

Los paquetes con alta centralidad de vector propio son aquellos que están conectados a otros paquetes importantes en la red. Esto significa que su importancia se deriva de su conexión con otros paquetes influyentes. Identificar estos paquetes puede ayudar a los desarrolladores a comprender qué paquetes tienen un impacto significativo en la estructura y el funcionamiento de la red y que los cambios que se produzcan en estos paquetes pueden afectar a otros paquetes en la red.

Existen variaciones de esta métrica, como la métrica de Katz calcula la importancia de un nodo teniendo en cuenta tanto la cantidad como la calidad de sus conexiones. Se basa en la idea de que un nodo es importante si está conectado con otros nodos importantes. Por lo tanto, asigna puntuaciones más altas a los nodos que tienen conexiones con otros nodos de alta importancia.
Por otro lado, el algoritmo de PageRank se utiliza para evaluar la relevancia de un nodo en función de la estructura de enlaces en la red. PageRank asigna puntuaciones a los nodos según la probabilidad de que un navegante aleatorio termine en ese nodo al seguir los enlaces de la red. En otras palabras, un nodo obtendrá una puntuación más alta si es enlazado por nodos importantes y relevantes en la red.

\capitulo{4}{Técnicas y herramientas}

\subsection{Entorno de desarrollo}

Se ha trabajado en un sistema operativo \textit{Ubuntu} y utilizando \textit{Visual Studio Code} 
(VSCode) como entorno de desarrollo integrado (IDE) preferido\footnote{ La elección del sistema 
operativo y el IDE depende de las preferencias personales del autor}. VSCode se destaca por 
su versatilidad y extensibilidad, lo que me permite personalizarlo y adaptarlo a mis necesidades 
específicas. Su amplia selección de extensiones me brinda herramientas adicionales y funcionalidades 
especializadas que enriquecen mi experiencia de programación.

El lenguaje de programación elegido es \textit{Python}, reconocido por su facilidad de uso y amplia 
gama de bibliotecas y frameworks disponibles\footnote{ Python se ha utilizado previamente en el 
Trabajo Final de Grado anterior}. Sin embargo, también considero esencial el uso de \textit{Bash}, 
un intérprete de comandos de Unix, para realizar diversas tareas y automatizaciones en el entorno 
de desarrollo.

Para llevar a cabo mis proyectos, cuento con un ordenador portátil de gama media equipado con un p
rocesador \textit{Intel® Core™ i5-11400H} y 16 GB de RAM. Estas especificaciones brindan un rendimiento 
adecuado para el desarrollo y la ejecución del software que acostumbro a usar\footnote{ Aunque el 
rendimiento puede variar dependiendo de los requisitos específicos del proyecto}. Sin embargo, en 
algunos casos ha habido incidentes debido al elevado consumo de memoria que requiere el procesamiento 
de la masiva cantidad de datos a la que nos hemos enfrentado.

A continuación se presenta una lista de los paquetes\footnote{Paquetes de Python} utilizados, destacando sus funcionalidades:

\begin{itemize}
  \item \textit{Pandas}: Una biblioteca de análisis de datos de alto rendimiento que proporciona 
  estructuras de datos y herramientas para manipular y analizar conjuntos de datos complejos.
  \item \textit{tqdm}: Una biblioteca que agrega una barra de progreso elegante y visual a los 
  bucles iterativos, lo que facilita el seguimiento del progreso de las operaciones en tiempo real.
  \item \textit{Requests}: Una biblioteca que simplifica el manejo de solicitudes HTTP, 
  permitiéndome realizar peticiones a servidores web y recibir respuestas de manera sencilla.
  \item \textit{BeautifulSoup4}: Una biblioteca que se utiliza para extraer información de páginas web 
  y realizar el análisis de datos web. Facilita la extracción de datos estructurados y no estructurados 
  mediante técnicas de web scraping.
  \item \textit{Selenium}: Una biblioteca que automatiza la interacción con navegadores web, lo que 
  me permite realizar pruebas de aplicaciones web o realizar acciones específicas en páginas web de 
  forma programática.
  \item \textit{Networkx}: Una biblioteca para el análisis de redes y grafos. Proporciona herramientas 
  para la creación, manipulación y estudio de estructuras de redes complejas.
  \item \textit{Matplotlib}: Una biblioteca de visualización de datos en 2D que me permite crear 
  gráficos y visualizaciones de datos de alta calidad.
  \item \textit{Pybraries}: Una biblioteca que hace de wrapper de el API de 
  \textit{libraries.io}\footnote{Libraries.io es una plataforma en línea que proporciona información 
  y datos sobre diferentes bibliotecas y paquetes de software de código abierto} para python.
  \item \textit{Typing\_extensions}: Una extensión del módulo typing de Python que proporciona 
  funcionalidades adicionales para anotaciones de tipos en tiempo de ejecución.
  \item \textit{pdoc}: Una biblioteca que me permite generar documentación automática a partir 
  de mis archivos de código fuente.
\end{itemize}

\subsection{Jupyter Notebooks y Computación en la nube}

\textit{Jupyter Notebooks} se ha convertido en una herramienta fundamental en el ámbito de la ciencia de datos y 
la programación interactiva. Estos notebooks permiten combinar código, texto explicativo y resultados 
visuales en un solo documento, lo que facilita la comunicación y colaboración en proyectos de análisis 
de datos. Los notebooks se ejecutan en un entorno interactivo, lo que permite explorar y experimentar 
con el código de manera iterativa, lo que resulta especialmente útil en tareas de análisis exploratorio 
de datos.\footnote{El análisis exploratorio de datos implica investigar y comprender los datos a través 
de la exploración visual, la estadística descriptiva y otras técnicas para obtener ideas y patrones 
clave.}

En cuanto a la computación en la nube, ha desempeñado un papel clave en el desarrollo de este TFG.
Plataformas como \textit{Kaggle} o \textit{Deepnote} proporcionan servicios de notebooks basados en la nube, lo que 
significa que los usuarios pueden acceder a un entorno de desarrollo completo sin tener que preocuparse 
por configurar y mantener su propia infraestructura. Esto es especialmente beneficioso en proyectos 
que requieren una gran cantidad de recursos computacionales, como el procesamiento de grandes volúmenes
 de datos.\footnote{El procesamiento de grandes volúmenes de datos implica trabajar con conjuntos de 
 datos masivos que pueden superar la capacidad de procesamiento de una sola máquina. La computación 
 en la nube permite distribuir y escalar el procesamiento para manejar estos volúmenes de datos.}

Además, la computación en la nube ha ayudado a reducir los costos asociados con la obtención y 
procesamiento de datos. La obtención de datos puede requerir tiempo, memoria y almacenamiento 
significativos, lo que puede ser costoso en términos de recursos locales. Al aprovechar la computación 
en la nube podemos acceder a recursos escalables y flexibles según 
sea necesario, lo que nos permite realizar análisis más eficientes y a gran escala sin incurrir en 
costos excesivos.\footnote{La escalabilidad y flexibilidad de los recursos en la computación en la 
nube se refiere a la capacidad de aumentar o disminuir la capacidad de cómputo y almacenamiento 
según las necesidades del proyecto, lo que permite un uso más eficiente de los recursos y un mejor 
control de costos.}

\subsection{Sistema de control de versiones}

\textit{GitHub} ha desempeñado un papel fundamental como plataforma de control de versiones Git en el ámbito del 
desarrollo de software colaborativo de código abierto. Como un estándar reconocido internacionalmente, 
GitHub, adquirido por \textit{Microsoft}, ha proporcionado a los desarrolladores una infraestructura sólida para
la gestión de proyectos. Además de su funcionalidad de repositorio Git público, GitHub ofrece 
herramientas integrales para el seguimiento y control de eventos relacionados con el desarrollo, 
lo que facilita la colaboración eficiente y transparente entre los miembros del equipo. 
Esta plataforma ha fomentado el desarrollo comunitario, impulsando la creación y mejora de proyectos 
de software en un entorno abierto y accesible para la comunidad global de 
desarrolladores.\footnote{El control de versiones es un sistema que registra y controla los cambios 
realizados en un proyecto a lo largo del tiempo. Permite realizar un seguimiento de las modificaciones, 
gestionar conflictos y recuperar versiones anteriores del código. Git es un sistema de control de 
versiones ampliamente utilizado en el desarrollo de software.}

GitHub, como plataforma de control de versiones basada en Git, permite a los desarrolladores 
almacenar y compartir sus repositorios de código, facilitando la colaboración y la contribución 
de múltiples personas a un proyecto. Además, ofrece herramientas como problemas, solicitudes de 
extracción y seguimiento de errores que permiten una comunicación efectiva entre los miembros del 
equipo y facilitan la gestión y resolución de problemas en el proceso de desarrollo de 
software.\footnote{Los problemas y las solicitudes de extracción son mecanismos utilizados en GitHub 
para informar y abordar problemas, sugerir cambios y revisar y fusionar contribuciones de código.}

El uso de GitHub ha fomentado el desarrollo comunitario y la creación de proyectos de software 
de calidad en un entorno colaborativo y transparente. Los desarrolladores pueden contribuir 
a proyectos existentes, realizar mejoras y correcciones de errores, y beneficiarse de la 
retroalimentación y la experiencia de otros miembros de la comunidad global de desarrolladores. 
Además, GitHub facilita la visibilidad y la accesibilidad de los proyectos, lo que permite a otros 
descubrir, aprender y utilizar el software desarrollado por la comunidad.\footnote{La comunidad 
global de desarrolladores se refiere a la amplia red de personas que colaboran, comparten conocimientos 
y contribuyen al desarrollo de software en todo el mundo.}

\subsection{Integración continua y el control de calidad}

La \textit{integración continua} y el \textit{control de calidad} desempeñan un papel crucial en el desarrollo de software. 
Para garantizar la calidad y la consistencia del proyecto, se ha utilizado \textit{SonarCloud}\footnote{SonarCloud es una herramienta de control de calidad que proporciona 
análisis estático de código para identificar problemas y mejorar la calidad del código.} como herramienta 
de control de calidad. Esta herramienta se integra con GitHub, lo que permite realizar un análisis 
automatizado de la calidad del código en cada commit. SonarCloud evalúa el código fuente en función 
de los estándares de calidad predefinidos y proporciona información detallada sobre posibles 
problemas, vulnerabilidades o malas prácticas. Esta integración continua de control de calidad 
asegura que el proyecto cumpla con los criterios de calidad deseados y permite abordar los 
problemas de manera oportuna.\footnote{El control de calidad se refiere al conjunto de procesos y técnicas utilizados para asegurar 
la calidad del software.}

Además, se ha empleado \textit{GitHub Pages} como una plataforma para alojar la documentación del código 
fuente de la biblioteca generada. GitHub Pages permite crear un sitio web estático que sirve 
como una fuente centralizada de información para los usuarios y desarrolladores del proyecto. 
Al alojar la documentación en GitHub Pages, se facilita el acceso y la navegación a través de 
la documentación, lo que mejora la usabilidad y la visibilidad del proyecto. Esta práctica de 
utilizar GitHub Pages para la documentación garantiza que la información esté siempre actualizada 
y disponible para todos los interesados en el proyecto.

\subsection{Persistencia de datos}

Se ha decidido seguir la metodología establecida en el Trabajo de Fin de Grado anterior, donde se 
emplean archivos CSV para almacenar los conjuntos de datos generados. Estos archivos CSV ofrecen una 
estructura tabular que permite representar de manera eficiente la lista de enlaces de paquetes y sus 
dependencias.\footnote{CSV (\textit{Comma-Separated Values}) es un formato de archivo que utiliza comas para 
separar los valores en una estructura tabular. Es ampliamente utilizado para el intercambio de datos 
en aplicaciones que requieren una estructura tabular sencilla.}

Además de los archivos CSV, en algunos casos se ha optado por utilizar objetos serializados para el 
almacenamiento de datos. Esta elección se basa en la facilidad que brindan los objetos serializados 
para ser guardados y cargados en los entornos de desarrollo, como los Jupyter Notebooks utilizados 
en el proyecto. Al serializar los objetos, se logra una representación compacta que puede ser 
almacenada en archivos y posteriormente restaurada sin perder la integridad de los 
datos.\footnote{La serialización es el proceso de convertir un objeto en una secuencia de 
bytes que puede ser almacenada o transmitida, y posteriormente restaurada para obtener el 
objeto original. Esto facilita la persistencia de datos complejos en entornos de programación.}

Sin embargo, la masividad de los datos ha planteado desafíos en cuanto a su almacenamiento. 
El volumen de los datos generados ha requerido el empleo de técnicas de compresión y división 
en \textit{lotes} para asegurar su conservación eficiente. Mediante la compresión\footnote{La compresión 
de datos es el proceso de reducir el tamaño de un 
archivo o conjunto de datos sin perder su contenido o información. Existen diferentes 
algoritmos de compresión que se utilizan para lograr este objetivo.}, se reduce el tamaño 
de los archivos de datos sin perder su contenido, lo que permite ahorrar espacio de 
almacenamiento. Por otro lado, la división 
en lotes consiste en dividir los datos en conjuntos más pequeños, lo cual facilita su manejo y 
procesamiento en entornos con recursos limitados.

\subsection{Gestión y organización del proyecto}

Inicialmente, se establecieron reuniones presenciales quincenales para discutir los objetivos de 
los \textit{sprints} propuestos. A medida que nos acercábamos a la etapa final del proyecto, se optó por 
realizar reuniones semanales para una mayor agilidad en la toma de decisiones. Sin embargo, debido 
a la naturaleza del proyecto, gestionar adecuadamente los sprints ha sido un desafío, ya que en 
ocasiones fue necesario replantear la forma en que estábamos abordando las tareas e incluso 
retroceder para solucionar problemas que surgieron durante el proceso.\footnote{Un sprint es un 
período de tiempo durante el cual se realiza un conjunto de tareas o actividades dentro de un 
proyecto ágil. Se utiliza comúnmente en la metodología Scrum para la gestión de proyectos.}

Se ha utilizado \textit{Microsoft Teams} como herramienta para facilitar las reuniones de forma remota, 
lo que permitió una comunicación efectiva y una colaboración fluida entre los miembros del equipo. 
Esta plataforma proporcionó un espacio para compartir documentos, discutir ideas y mantener un 
seguimiento de las tareas asignadas.\footnote{Microsoft Teams es una plataforma de colaboración 
en línea que permite la comunicación y el trabajo en equipo a través de chat, videoconferencias, 
compartición de archivos y otras funcionalidades.}

A lo largo del proyecto, se pueden distinguir varias etapas. En primer lugar, hubo una fase de 
toma de contacto, donde se adquirió un conocimiento inicial sobre los objetivos y el alcance del 
Trabajo de Fin de Grado. A continuación, se llevó a cabo una fase de investigación y aprendizaje, 
donde se profundizó en los conceptos teóricos de la ciencia de redes, aprovechando los conocimientos 
adquiridos en asignaturas como \textit{Nuevas Tecnologías}.\footnote{Asignatura del grado de Ingeniería 
Informática en la UBU que proporciona una visión general de la ciencia de redes y sus aplicaciones.}

Otra etapa clave fue el estudio de estrategias para la obtención de datos. Dado que había 
diferentes fuentes disponibles, como archivos CSV, sitios web y APIs\footnote{API (Application Programming Interface) es un 
conjunto de reglas y protocolos que permite la comunicación y la interacción entre diferentes 
software o componentes de software. Permite el intercambio de datos y la ejecución de funciones 
entre sistemas diferentes.}, se exploraron y 
seleccionaron las mejores opciones para obtener los datos necesarios. Además, se desarrolló 
una herramienta en Python que permitió la extracción de datos de estas diversas fuentes de 
manera eficiente y automatizada.

Una vez obtenidos los datos, se procedió a realizar un análisis de los mismos, aplicando técnicas 
y algoritmos propios de la ciencia de redes para extraer información relevante y obtener conclusiones 
significativas. Este análisis proporcionó una base sólida para la posterior redacción de la memoria 
del proyecto.
\capitulo{5}{Aspectos relevantes del desarrollo del proyecto}

\section {Analisis estatico de las redes: Bowtie}

El concepto de \textit{"bowtie"} hace referencia a una representación gráfica de una red
que exhibe una estructura en forma de corbata. Se trata de una visualización que segmenta
la red en diversas componentes y describe la conectividad existente entre ellas.

\begin{itemize}
    \item \textbf{Número de nodos (Nº nodes):} Indica la cantidad total de nodos o entidades individuales presentes en la red. Cada nodo representa un elemento o entidad específica dentro del contexto de la red.

    \item \textbf{Número de aristas (Nº edges):} Representa la cantidad total de aristas o conexiones existentes entre los nodos de la red. Cada arista denota una relación o interacción entre dos nodos.

    \item \textbf{Primera componente fuertemente conectada (1st SCC):} Constituye una porción de la red en la cual todos los nodos están interconectados mutuamente a través de rutas directas o indirectas. En otras palabras, se establece un camino desde cualquier nodo de esta componente hacia cualquier otro nodo presente en ella.

    \item \textbf{Segunda componente fuertemente conectada (2nd SCC):} Representa otra parte de la red donde todos los nodos se encuentran interconectados de manera mutua, sin embargo, no existen conexiones directas entre los nodos de la primera componente fuertemente conectada y los nodos de esta componente.

    \item \textbf{Componente de entrada (In component):} Corresponde a la sección de la red que abarca todos aquellos nodos desde los cuales se puede trazar al menos una ruta hacia la primera componente fuertemente conectada.

    \item \textbf{Componente de salida (Out component):} Se refiere a la porción de la red que comprende todos los nodos hacia los cuales existe al menos una ruta partiendo desde la primera componente fuertemente conectada.

    \item \textbf{Tubos (Tubes):} Son trayectorias directas que transcurren desde la componente de entrada hacia la componente de salida sin atravesar la primera componente fuertemente conectada. Estos tubos sirven como enlaces entre las componentes de entrada y salida, eludiendo la estructura de la corbata.

    \item \textbf{Tendril de entrada (In tendrils):} Representan aquellos nodos que están conectados a la componente de entrada, pero no forman parte de la primera componente fuertemente conectada ni de los tubos.

    \item \textbf{Tendril de salida (Out tendrils):} Hacen referencia a los nodos que se encuentran conectados a la componente de salida, sin embargo, no forman parte de la primera componente fuertemente conectada ni de los tubos.

    \item \textbf{Desconectados (Disconnected):} Son los nodos que no están conectados a ninguna otra componente de la corbata y no poseen conexiones entrantes o salientes con otros nodos en la red.

\end{itemize}

Estos conceptos proporcionan una descripción detallada de la estructura de la red desde la perspectiva del enfoque \textit{bowtie}, permitiendo identificar las diferentes componentes y su nivel de interconexión.

A continuacion se presentan los resultados obtenidos para cada uno de los repositorios de paquetes analizados.

\subsection{Bioconductor y CRAN}

Para el repositorio de \textit{Bioconductor}, solo tenemos un conjunto de datos obtenidos mediante \textit{Olivia Finder}.

Para el repositorio de \textit{CRAN} tenemos 4 conjuntos de datos. \ref{tab:data_bc_cran}

\begin{itemize}
    \item El primero usa los datos de \textit{Libraries.io} (\textit{Depends} e \textit{Imports}) sin discriminar versiones de paquete, es decir, para un paquete dado tenemos en cuenta todas las dependencias que ha tenido en cada una de sus versiones. Cabe destacar que este punto de vista carece de sentido, pero se ha incluido para comparar los resultados del anterior \textit{TFG} donde no se tuvo en cuenta este aspecto.

    \item El segundo conjunto de datos usa la última versión de cada paquete disponible e incluye las dependencias del tipo (\textit{Depends} e \textit{Imports}).

    \item El tercer conjunto de datos usa la última versión de cada paquete disponible e incluye las dependencias del tipo (\textit{Depends}, \textit{Imports}, \textit{Suggest} y \textit{Enhances}).

    \item El cuarto conjunto de datos es el obtenido mediante \textit{Olivia Finder} para los tipos de dependencias (\textit{Depends} e \textit{Imports}).
\end{itemize}

\begin{table}[h!]
    \centering
    \label{tab:data_bc_cran}
    \tiny
    \begin{tabular}{|l|l|l|l|l|l|l|l|l|l|}
        \hline
        \textbf{}              & \textbf{Bioconductor} & \textbf{CRAN}          & \textbf{CRAN}          & \textbf{CRAN}          & \textbf{CRAN}    \\
                               & \textbf{Scraped}      & \textbf{Librariesio 1} & \textbf{Librariesio 2} & \textbf{Librariesio 3} & \textbf{Scraped} \\
        \hline
        \textbf{Nodes (n)}     & 3509                  & 16174                  & 15647                  & 16055                  & 18671            \\
        \textbf{Edges (m)}     & 28320                 & 117724                 & 76207                  & 107370                 & 113273           \\
        \textbf{1st SCC}       & 1                     & 1405                   & 1                      & 923                    & 1                \\
        \textbf{2nd SCC}       & 1                     & 6                      & 1                      & 13                     & 1                \\
        \textbf{In Component}  & 124                   & 381                    & 79                     & 333                    & 6                \\
        \textbf{Out Component} & 0                     & 11746                  & 0                      & 11269                  & 0                \\
        \textbf{Tubes}         & 0                     & 444                    & 0                      & 666                    & 0                \\
        \textbf{In tendrils}   & 2161                  & 1680                   & 14980                  & 2373                   & 17984            \\
        \textbf{Out tendrils}  & 0                     & 481                    & 0                      & 442                    & 0                \\
        \textbf{Disconected}   & 1223                  & 37                     & 587                    & 49                     & 680              \\
        \textbf{Attack}        & 2109                  & 15123                  & 14395                  & 15056                  & 17223            \\
        \textbf{Attack/n}      & 0.6010                & 0.9350                 & 0.9200                 & 0.9378                 & 0.9224           \\
        \textbf{Failure}       & 24.8173               & 1454.5255              & 24.5910                & 957.2135               & 33.5440          \\
        \textbf{Failure/n}     & 0.0071                & 0.0899                 & 0.0016                 & 0.0596                 & 0.0018           \\
        \hline
    \end{tabular}
    \caption{Tabla de datos para Bioconductor y CRAN}
\end{table}

Se ha omitido la columna \textit{CRAN Librariesio 1} en la tabla anterior ya que no aporta información relevante.

\begin{itemize}
    \item \textit{Bioconductor} sigue siendo más pequeño en términos de \textit{nodos} (3509)
          en comparación con las fuentes de \textit{CRAN Librariesio 2}, \textit{CRAN Librariesio 3}
          y \textit{CRAN Scraped}.
    \item En cuanto al número de \textit{aristas} (\textit{edges}), \textit{CRAN Librariesio 3}
          tiene el valor más alto (107,370), seguido de \textit{CRAN Scraped} (113,273) y
          \textit{CRAN Librariesio 2} (76,207).
    \item La cantidad de \textit{nodos} en la \textit{componente fuertemente conectada}
          (\textit{1st SCC}) es similar en \textit{Bioconductor}, \textit{CRAN Librariesio 2} y
          \textit{CRAN Scraped}, mientras que \textit{CRAN Librariesio 3} tiene un número más alto
          de \textit{nodos} en esta categoría.
    \item \textit{Bioconductor} tiene un número menor de \textit{nodos} en la
          \textit{segunda componente fuertemente conectada} (\textit{2nd SCC}) en comparación con
          las fuentes de \textit{CRAN Librariesio 2}, \textit{CRAN Librariesio 3} y
          \textit{CRAN Scraped}.
    \item En términos de \textit{componentes débilmente conectadas},
          \textit{CRAN Librariesio 3} tiene el número más alto de \textit{nodos}, seguido
          de \textit{CRAN Scraped}, \textit{CRAN Librariesio 2} y \textit{Bioconductor}.
    \item El número de \textit{tubos} (\textit{tubes}) en la red es cero en todas las fuentes
          de información.
    \item \textit{Bioconductor} tiene un número mayor de \textit{nodos} en los componentes en
          forma de \textit{tendril} tanto de entrada como de salida en comparación con las fuentes
          de \textit{CRAN Librariesio 2}, \textit{CRAN Librariesio 3} y \textit{CRAN Scraped}.
    \item La categoría de \textit{nodos desconectados} muestra que \textit{CRAN Librariesio 3} tiene un número más alto de \textit{nodos desconectados}, mientras que \textit{Bioconductor} y \textit{CRAN Scraped} tienen valores más bajos.
    \item El análisis de \textit{ataque} (\textit{attack}) muestra que \textit{Bioconductor} tiene un valor más bajo en comparación con las fuentes de \textit{CRAN Librariesio 2}, \textit{CRAN Librariesio 3} y \textit{CRAN Scraped}. Esto indica una mayor resistencia a \textit{ataques} en \textit{Bioconductor}.
    \item El análisis de \textit{falla} (\textit{failure}) muestra que \textit{Bioconductor} tiene un valor más bajo en comparación con \textit{CRAN Librariesio 3} y \textit{CRAN Scraped}. Sin embargo, \textit{CRAN Librariesio 2} tiene el valor más bajo de \textit{fallas}. Esto indica una mayor \textit{robustez} en \textit{Bioconductor} y \textit{CRAN Librariesio 2} en comparación con \textit{CRAN Librariesio 3} y \textit{CRAN Scraped}.
\end{itemize}

\subsection{PyPI}

En el contexto de PyPI, se dispone de tres conjuntos de datos para el análisis: \ref{tab:data_pypi}.

El \textit{primer} conjunto de datos, al igual que en el caso anterior, utiliza los datos de \textit{libraries.io} y considera como dependencia de un paquete a todos los paquetes que hayan sido alguna vez dependencia de dicho paquete a lo largo de todo su histórico de versiones.

El \textit{segundo} conjunto de datos realiza un filtrado de los datos de \textit{libraries.io}, considerando únicamente las dependencias asociadas a la última versión de cada paquete.

El \textit{tercer} conjunto de datos abarca la red de dependencias obtenida mediante \textit{Olivia Finder}.

\begin{table}[h!]
    \centering
    \tiny
    \label{tab:data_pypi}
    \begin{tabular}{|l|l|l|l|}
        \hline
                               & \textbf{PyPI Libraries 1} & \textbf{PyPI Libraries 2} & \textbf{PyPI Scraped} \\
        \hline
        \textbf{Nodes (n)}     & 50766                     & 49306                     & 214469                \\
        \textbf{Edges (m)}     & 155369                    & 134575                    & 933955                \\
        \textbf{1st SCC}       & 7                         & 4                         & 283                   \\
        \textbf{2nd SCC}       & 4                         & 4                         & 19                    \\
        \textbf{In Component}  & 39                        & 21                        & 449                   \\
        \textbf{Out Component} & 62                        & 5                         & 138219                \\
        \textbf{Tubes}         & 13                        & 1                         & 2446                  \\
        \textbf{In tendrils}   & 23815                     & 27742                     & 30261                 \\
        \textbf{Out tendrils}  & 13                        & 11                        & 14941                 \\
        \textbf{Disconected}   & 26817                     & 21522                     & 27870                 \\
        \textbf{Attack}        & 22315                     & 19212                     & 145000                \\
        \textbf{Attack/n}      & 0.4396                    & 0.3896                    & 0.6761                \\
        \textbf{Failure}       & 15.7301                   & 8.5733                    & 489.5527              \\
        \textbf{Failure/n}     & 0.0003                    & 0.0002                    & 0.0023                \\
        \hline
    \end{tabular}
    \caption{Tabla de datos para PyPI}
\end{table}

En la sigiente comparacion omitimos el conjunto de datos \textit{PyPI Libraries 1} debido a que no es comparable con los otros dos conjuntos de datos.

\begin{itemize}
    \item El conjunto de datos \textit{PyPI Libraries 2} tiene un total de 49,306 nodos,
          mientras que el conjunto de datos \textit{PyPI Scraped} cuenta con 214,469 nodos.
          Se aprecia una diferencia significativa en términos de tamaño de red entre ambos conjuntos.
    \item En cuanto al número de aristas, \textit{PyPI Scraped} presenta un valor
          considerablemente más alto con 933,955 aristas, en comparación con las 134,575
          aristas del conjunto \textit{PyPI Libraries 2}.
    \item En relación a las componentes fuertemente conectadas, tanto \textit{PyPI Libraries 2}
          como \textit{PyPI Scraped} poseen 4 nodos en la segunda componente fuertemente
          conectada (\textit{2nd SCC}), mientras que la cantidad de nodos en la primera
          componente fuertemente conectada (\textit{1st SCC}) varía significativamente,
          siendo 4 para \textit{PyPI Libraries 2} y 283 para \textit{PyPI Scraped}.
    \item En términos de componentes débilmente conectadas, \textit{PyPI Scraped}
          presenta una mayor cantidad de nodos en la componente de entrada (\textit{In Component})
          con 449 nodos, en comparación con los 21 nodos de \textit{PyPI Libraries 2}. Por otro
          lado, \textit{PyPI Scraped} también cuenta con una mayor cantidad de nodos en la
          componente de salida (\textit{Out Component}) con 138,219 nodos, mientras que
          \textit{PyPI Libraries 2} tiene solo 5 nodos en esta categoría.
    \item El número de tubos (\textit{Tubes}) es significativamente mayor en
          \textit{PyPI Scraped} con 2,446 tubos, en comparación con el único tubo presente
          en \textit{PyPI Libraries 2}.
    \item En cuanto a los componentes en forma de tendril, \textit{PyPI Libraries 2} tiene
          27,742 nodos en los tendril de entrada (\textit{In tendrils}) y 11 nodos en los
          tendril de salida (\textit{Out tendrils}), mientras que \textit{PyPI Scraped} tiene
          30,261 nodos en los tendril de entrada y 14,941 nodos en los tendril de salida.
    \item La categoría de nodos desconectados (\textit{Disconnected}) muestra que
          \textit{PyPI Libraries 2} tiene 21,522 nodos desconectados, mientras que \textit{PyPI Scraped} tiene 27,870 nodos desconectados.
    \item En cuanto al análisis de ataque (\textit{Attack}), \textit{PyPI Libraries 2}
          tiene un valor de 19,212 y \textit{PyPI Scraped} presenta un valor más alto de
          145,000, lo que indica una mayor susceptibilidad en \textit{PyPI Scraped} en
          comparación con \textit{PyPI Libraries 2}.
    \item En el análisis de falla (\textit{Failure}), \textit{PyPI Libraries 2} tiene un
          valor de 8.5733, mientras que \textit{PyPI Scraped} muestra un valor más alto de 489.5527.
          Esto sugiere una mayor robustez en \textit{PyPI Libraries 2} en comparación con
          \textit{PyPI Scraped}.
    \item Los valores normalizados (\textit{Attack/n} y \textit{Failure/n}) indican la
          proporción de ataques y fallas en relación al número total de nodos. En este sentido,
          \textit{PyPI Scraped} presenta valores más altos en ambas métricas en comparación con
          \textit{PyPI Libraries 2}.
\end{itemize}


\subsection{NPM}

En el contexto del repositorio \textit{npm}, se disponen de cuatro conjuntos de datos
distintos, cada uno con sus propias características: \ref{tab:data_npm}

\begin{itemize}
    \item El primer conjunto de datos se basa en la información recopilada por
          \textit{libraries.io}. Este conjunto considera todas las dependencias que un paquete ha
          tenido a lo largo de su historial de versiones. Incluye tanto las dependencias de tipo
          \textit{runtime} como las de desarrollo (\textit{dev}).
    \item El segundo conjunto de datos realiza un filtrado del conjunto anterior, conservando
          únicamente las dependencias asociadas a la última versión de cada paquete. Además, se
          restringe el análisis a las dependencias de tipo \textit{runtime} y desarrollo, excluyendo
          otras dependencias no relevantes para el contexto.

    \item El tercer conjunto de datos utiliza los datos generados por \textit{Olivia Finder}
          para las dependencias de tipo \textit{runtime}. \textit{Olivia Finder} es una herramienta
          específica que permite identificar y analizar las dependencias en tiempo de ejecución de
          los paquetes en el repositorio \textit{npm}.

    \item Por último, el cuarto conjunto de datos se genera también mediante \textit{Olivia Finder},
          pero incluye tanto las dependencias de tipo \textit{runtime} como las de desarrollo
          (\textit{dev}). Esto proporciona una visión más completa de las dependencias utilizadas en
          el entorno de desarrollo y en tiempo de ejecución de los paquetes del repositorio \textit{npm}.
\end{itemize}

\begin{table}[h!]
    \centering
    \tiny
    \label{tab:data_npm}
    \begin{tabular}{|l|l|l|l|l|}
        \hline
                               & \textbf{NPM Librariesio 1} & \textbf{NPM Librariesio 2} & \textbf{NPM Scraped 1} & \textbf{NPM Scraped 2} \\
        \hline
        \textbf{Nodes (n)}     & 1074508                    & 1064531                    & 1059758                & 1832943                \\
        \textbf{Edges (m)}     & 13052831                   & 11405275                   & 4855094                & 22036615               \\
        \textbf{1st SCC}       & 26486                      & 13378                      & 26                     & 19579                  \\
        \textbf{2nd SCC}       & 175                        & 157                        & 17                     & 451                    \\
        \textbf{In Component}  & 3849                       & 1827                       & 0                      & 3718                   \\
        \textbf{Out Component} & 936295                     & 940266                     & 1                      & 1626207                \\
        \textbf{Tubes}         & 3745                       & 4260                       & 0                      & 7599                   \\
        \textbf{In tendrils}   & 17891                      & 19759                      & 0                      & 50120                  \\
        \textbf{Out tendrils}  & 69604                      & 60947                      & 0                      & 77588                  \\
        \textbf{Disconected}   & 16638                      & 24094                      & 1059731                & 48132                  \\
        \textbf{Attack}        & 975555                     & 968059                     & 258821                 & 1683928                \\
        \textbf{Attack/n}      & 0.9079                     & 0.9094                     & 0.2442                 & 0.9187                 \\
        \textbf{Failure}       & 27193.8251                 & 13633.8779                 & 62.0866                & 20934.7775             \\
        \textbf{Failure/n}     & 0.0253                     & 0.0128                     & 0.0001                 & 0.0114                 \\
        \hline
    \end{tabular}
    \caption{Tabla de datos para NPM}
\end{table}


Se han omitido los datos de \textit{NPM Librariesio 1} debido a que no se considera un conjunto
de datos relevante para el análisis.

\begin{itemize}
    \item El conjunto de datos \textit{NPM Librariesio 2} tiene un total de 1,064,531 nodos,
          mientras que el conjunto de datos \textit{NPM Scraped 1} cuenta con 1,059,758 nodos y el
          conjunto de datos \textit{NPM Scraped 2} tiene la mayor cantidad de nodos con 1,832,943.
          Se puede observar una diferencia significativa en términos de tamaño de red entre los
          conjuntos de datos.
    \item En cuanto al número de aristas, el conjunto de datos \textit{NPM Scraped 2} presenta
          un valor considerablemente más alto con 22,036,615 aristas, en comparación con las
          11,405,275 aristas del conjunto \textit{NPM Librariesio 2}.
    \item En relación a las componentes fuertemente conectadas, tanto el conjunto
          \textit{NPM Librariesio 2} como el conjunto \textit{NPM Scraped 2} poseen 157 nodos en
          la segunda componente fuertemente conectada (\textit{2nd SCC}), mientras que la cantidad
          de nodos en la primera componente fuertemente conectada (\textit{1st SCC}) varía
          significativamente, siendo 13,378 para \textit{NPM Librariesio 2} y 19,579 para
          \textit{NPM Scraped 2}.
    \item En términos de componentes débilmente conectadas, el conjunto \textit{NPM Scraped 2}
          presenta una mayor cantidad de nodos en la componente de entrada (\textit{In Component})
          con 3,718 nodos, en comparación con los 1,827 nodos del conjunto \textit{NPM Librariesio 2}.
          Por otro lado, el conjunto \textit{NPM Scraped 2} también cuenta con una mayor cantidad de
          nodos en la componente de salida (\textit{Out Component}) con 1,626,207 nodos, mientras
          que \textit{NPM Librariesio 2} tiene solo 940,266 nodos en esta categoría.
    \item El número de tubos (\textit{Tubes}) es significativamente mayor en el conjunto
          \textit{NPM Scraped 2} con 7,599 tubos, en comparación con los 4,260 tubos presentes en
          el conjunto \textit{NPM Librariesio 2}.
    \item En cuanto a los componentes en forma de tendril, \textit{NPM Librariesio 2} tiene
          19,759 nodos en los tendril de entrada (\textit{In tendrils}) y 60,947 nodos en los
          tendril de salida (\textit{Out tendrils}), mientras que \textit{NPM Scraped 2} tiene
          50,120 nodos en los tendril de entrada y 77,588 nodos en los tendril de salida.
    \item La categoría de nodos desconectados (\textit{Disconnected}) muestra que
          \textit{NPM Librariesio 2} tiene 24,094 nodos desconectados, mientras que
          \textit{NPM Scraped 1} tiene la mayor cantidad de nodos desconectados con
          1,059,731.
    \item En cuanto al análisis de ataque (\textit{Attack}), \textit{NPM Scraped 2} tiene
          un valor de 1,683,928 y \textit{NPM Librariesio 2} presenta un valor ligeramente menor
          de 968,059, lo que indica una mayor susceptibilidad a esta métrica en \textit{NPM Scraped 2}
          en comparación con \textit{NPM Librariesio 2}.
    \item En el análisis de falla (\textit{Failure}), \textit{NPM Scraped 2} muestra un valor
          más alto de 20,934.7775, mientras que \textit{NPM Librariesio 2} tiene un valor de
          13,633.8779. Esto sugiere una mayor robustez en \textit{NPM Librariesio 2} en comparación
          con \textit{NPM Scraped 2}.
    \item Los valores normalizados (\textit{Attack/n} y \textit{Failure/n}) indican la
          proporción de y fallas en relación al número total de nodos. En este sentido,
          \textit{NPM Scraped 2} presenta valores más altos en ambas métricas en comparación
          con \textit{NPM Librariesio 2}.
\end{itemize}

\section{La red de dependencias de CRAN}

CRAN (\textit{Comprehensive R Archive Network}) es un \textit{repositorio} en línea que alberga
una amplia colección de paquetes de software para el \textit{lenguaje de programación} R.
R es un \textit{entorno de programación} y un \textit{lenguaje estadístico} ampliamente
utilizado en la comunidad científica para el análisis y la visualización de datos.
El lenguaje R se destaca por su \textit{flexibilidad} y \textit{extensibilidad},
lo que permite a los investigadores y científicos implementar \textit{algoritmos estadísticos}
avanzados y realizar \textit{análisis exploratorios de datos}.
Los paquetes almacenados en CRAN ofrecen una variedad de funcionalidades especializadas,
incluyendo \textit{modelado estadístico}, \textit{gráficos}, \textit{manipulación de datos}
y \textit{visualización}, lo que permite a los usuarios ampliar las capacidades base de R.

\begin{table}[h!]
    \begin{center}
        \begin{tabular}{|l|c|}
            \hline
            \textbf{Descripción}                            & \textbf{Cantidad} \\
            \hline
            Packages in librariesio                         & 15154             \\
            Packages in scraped                             & 18195             \\
            Common packages                                 & 11589             \\
            Packages in librariesio that are not in scraped & 3565              \\
            Packages in scraped that are not in librariesio & 6606              \\
            \hline
        \end{tabular}
        \caption{Comparación de paquetes en CRAN entre los datos de libraries.io y los recolectados en este trabajo.}
        \label{tab:cran_common_packages}
    \end{center}
\end{table}

El análisis sobre los datos recopilados \ref{tab:cran_common_packages} indica que el conjunto de datos \textit{scraped}
ha experimentado un crecimiento notable en comparación con \textit{libraries.io}, tanto en
términos de la cantidad total de paquetes como en la inclusión de nuevos paquetes.
Esta ampliación evidencia una recopilación más exhaustiva y actualizada de información. \ref{fig:cran_common_packages2}
\ref{fig:cran_common_packages3}.

\begin{figure}[h!]
    \begin{center}
        \includegraphics[width=0.7\textwidth]{img/cran/circle.png}
        \caption{Comparacion de los paquetes comunes entre los dos conjuntos de datos.}
        \label{fig:cran_common_packages2}
    \end{center}
\end{figure}

\begin{figure}[h!]
    \begin{center}
        \includegraphics[width=0.6\textwidth]{img/cran/bars.png}
        \caption{Comparacion del numero de paquetes.}
        \label{fig:cran_common_packages3}
    \end{center}
\end{figure}

Por consiguiente, se concluye que el conjunto de datos \textit{scraped} proporciona una
visión más completa y actualizada de los paquetes disponibles. Al incluir tanto los paquetes
en común como los paquetes adicionales respecto a \textit{libraries.io}, \textit{scraped}
se configura como una fuente valiosa para investigaciones y análisis en el ámbito de estudio.






\subsection{El tamaño}

A continuacion realizaremos una comparacion de las medidas de tamaño entre los dos conjuntos de datos y como es
el agrupamiento de los nodos en cada uno de ellos. \ref{tab:cran_size}

\begin{table}[h!]
    \begin{center}
        \begin{tabular}{|l|c|c|c|}
            \hline
            \textbf{Medida}                & \textbf{libraries.io} & \textbf{scraped} \\
            \hline
            Number of nodes                & 15647                 & 18671            \\
            Number of edges                & 76207                 & 113273           \\
            Average degree                 & 9.740                 & 12.133           \\
            Average clustering coefficient & 0.131                 & 0.152            \\
            \hline
        \end{tabular}
        \caption{Comparación de medidas de tamaño entre los dos conjuntos de datos.}
        \label{tab:cran_size}
    \end{center}
\end{table}

La red \textit{scraped} tiene un mayor número de nodos (\textit{18671}) en comparación con la
red \textit{libraries.io} (\textit{15647}). Esto indica que \textit{scraped} contiene más elementos o entidades
interconectadas en su estructura de red.

La red \textit{scraped} también tiene un mayor número de aristas o enlaces (\textit{113273}) en
comparación con la red \textit{libraries.io} (\textit{76207}). Esto implica que \textit{scraped} tiene más
conexiones entre los nodos, lo que aumenta su grado de conectividad.

El grado promedio de los nodos en la red \textit{scraped} es más alto (\textit{12.133}) en
comparación con el de la red \textit{libraries.io} (\textit{9.740}). Esto sugiere que, en promedio, cada nodo
en \textit{scraped} tiene más conexiones con otros nodos en comparación con los nodos en \textit{libraries.io}.

El coeficiente de agrupamiento promedio en la red \textit{scraped} es
ligeramente más alto (\textit{0.152}) que en la red \textit{libraries.io} (\textit{0.131}). Esto indica que,
en promedio, los nodos en \textit{scraped} tienen una mayor tendencia a formar grupos o comunidades más densamente
interconectadas en comparación con los nodos en \textit{libraries.io}.

\subsection{El grado}

\begin{figure}[h!]
    \begin{center}
        \includegraphics[width=1\textwidth]{img/cran/distribucion_grado.png}
        \caption{Distribucion de grado \textit{libraries.io}.}
        \label{fig:cran_degree_distribution}
    \end{center}
\end{figure}

\begin{figure}[h!]
    \begin{center}
        \includegraphics[width=1\textwidth]{img/cran/distribucion_grado2.png}
        \caption{Distribucion de grado \textit{scraped}.}
        \label{fig:cran_degree_distribution_scraped}
    \end{center}
\end{figure}

Al realizar una comparativa de las distribuciones de grado entre ambos conjuntos de datos,
se evidencia un leve incremento en el grado promedio. El grado promedio del conjunto de datos
\textit{libraries.io} es de 4.87, mientras que en el conjunto \textit{scraped} alcanza un valor
de 6.06. Estos resultados indican un incremento en la conectividad y la relevancia de los
paquetes más destacados en el conjunto de datos \textit{scraped}. Tal variación en los valores
promedio refleja el aumento en la importancia y la interconexión de los paquetes dentro de este
conjunto de datos, lo cual puede estar relacionado con su mayor tamaño y actualización.
\ref{fig:cran_degree_distribution} \ref{fig:cran_degree_distribution_scraped}

Además, se observa que la tendencia a disminuir el número de dependencias es ligeramente
más pronunciada que la tendencia a aumentarlas. Esto sugiere que, en la evolución de
la red de dependencias, los paquetes tienden a reducir su dependencia directa o a
reorganizar sus conexiones con otros paquetes, lo que puede ser resultado de procesos
de \textit{refactorización}, \textit{optimización} o \textit{consolidación}.


\subsubsection{Grado de salida (\textit{out degree})}


En el análisis de la distribución del grado de salida, no se observan diferencias significativas
que puedan ser comparadas. Podemos inferir que ambas distribuciones siguen una tendencia
similar, con la única distinción de que en el conjunto de datos nuevo se encuentra un mayor
número de nodos \ref{fig:cran_out_lib} \ref{fig:cran_out_scraped}.

\begin{figure}[h!]
    \begin{center}
        \includegraphics[width=1\textwidth]{img/cran/out_deg.png}
        \caption{Distribucion de Out degree de \textit{libraries.io}.}
        \label{fig:cran_out_lib}
    \end{center}
\end{figure}

\begin{figure}[h!]
    \begin{center}
        \includegraphics[width=1\textwidth]{img/cran/out_deg2.png}
        \caption{Distribucion de Out degree de \textit{scraped}.}
        \label{fig:cran_out_scraped}
    \end{center}
\end{figure}

Al examinar el conjunto de paquetes con mayor grado de salida, se observa un aumento generalizado
en todos ellos. Estos paquetes destacados representan una centralidad de grado significativa,
lo que implica que son las dependencias más utilizadas dentro del sistema. Es comprensible que
estos paquetes, al ser ampliamente utilizados, hayan mantenido una presencia constante en la
red y hayan experimentado un incremento en el número de sus dependientes debido al surgimiento
de nuevos paquetes. \ref{fig:cran_out_libio_top}

\begin{figure}[h!]
    \begin{center}
        \includegraphics[width=1\textwidth]{img/cran/out_lib.png}
        \caption{Top paquetes con mayor grado de salida en \textit{libraries.io}.}
        \label{fig:cran_out_libio_top}
    \end{center}
\end{figure}

Esta observación refuerza la importancia y la relevancia de estos paquetes clave en el
ecosistema estudiado. Su estabilidad y el aumento en sus dependientes pueden atribuirse a
su funcionalidad y a su amplia adopción por parte de los usuarios. Además, el incremento en
la cantidad de dependientes es un indicativo del crecimiento y la evolución continua del
sistema, donde se generan nuevas relaciones de dependencia entre los paquetes existentes
y los recién agregados.

Al realizar la comparativa utilizando el nuevo conjunto de datos, se observa que los principales
representantes del ranking se mantienen presentes. Sin embargo, se han producido algunas variaciones
en el orden de algunos puestos dentro del top. Además, se ha registrado la inclusión de paquetes que
previamente se encontraban fuera del top. \ref{fig:cran_out_scraped_top}

\begin{figure}[h!]
    \begin{center}
        \includegraphics[width=1\textwidth]{img/cran/out_scr.png}
        \caption{Top paquetes con mayor grado de salida en \textit{scraped}.}
        \label{fig:cran_out_scraped_top}
    \end{center}
\end{figure}

Esta inclusión de paquetes puede atribuirse al incremento en el número de sus dependencias,
el cual ha superado a aquellos que han salido del top en este periodo de tiempo analizado.
Estos nuevos paquetes han logrado adquirir una mayor relevancia y han fortalecido su posición
dentro del conjunto de datos. Este fenómeno puede ser consecuencia de su creciente adopción y
de la expansión de su funcionalidad, lo que ha llevado a un incremento en el número de paquetes
que los utilizan como dependencias.

Este incremento lo podemos ver reprensentado en la siguiente figura \ref{fig:cran_dependents_dist}.

\begin{figure}[h!]
    \begin{center}
        \includegraphics[width=1\textwidth]{img/cran/dependents_dist.png}
        \caption{Distribucion de dependientes.}
        \label{fig:cran_dependents_dist}
    \end{center}
\end{figure}

\subsubsection{Grado de entrada (\textit{in degree})}


En el análisis de la distribución del grado de entrada, se observa un comportamiento común
en ambos conjuntos de datos. Debido a su similitud, resulta difícil extraer conclusiones
significativas. Sin embargo, se puede observar un ligero incremento en el nuevo conjunto de
datos en comparación con el conjunto de datos de libraries.io. \ref{fig:cran_in_lib} \ref{fig:cran_in_scraped}

\begin{figure}[h!]
    \begin{center}
        \includegraphics[width=0.8\textwidth]{img/cran/ind_lib.png}
        \caption{Distribucion de In degree de \textit{libraries.io}.}
        \label{fig:cran_in_lib}
    \end{center}
\end{figure}

\begin{figure}[h!]
    \begin{center}
        \includegraphics[width=0.8\textwidth]{img/cran/ind_scr.png}
        \caption{Distribucion de In degree de \textit{scraped}.}
        \label{fig:cran_in_scraped}
    \end{center}
\end{figure}

Tomando como punto de referencia la frecuencia de nodos con un grado de entrada de valor 20,
en el conjunto de datos de libraries.io se encuentran alrededor de 50 nodos, mientras que en
el nuevo conjunto de datos se registran aproximadamente 100 nodos. Estos valores pueden sugerir
un aumento en la cantidad de paquetes que reciben un número determinado de dependencias.


Si representamos el \textit{top} de \textit{in degree} de los paquetes del conjunto de datos
de \textit{libraries.io}, podemos identificar aquellos paquetes que poseen un mayor número de
dependencias. Estos paquetes podrían considerarse los más vulnerables en su primer nivel de dependencia,
sin tener en cuenta la transitividad. Al analizar los resultados, se puede concluir que un porcentaje
significativo de los paquetes destacados en este \textit{top} han mantenido su presencia en la red a
lo largo del tiempo.

Además, se observa una tendencia general a disminuir el número de dependencias para la mayoría de
los casos en el \textit{top}. Esto sugiere que, a medida que evoluciona la red de dependencias,
algunos paquetes han logrado reducir su dependencia directa o han redistribuido sus conexiones
con otros paquetes.


\begin{figure}[h!]
    \begin{center}
        \includegraphics[width=1\textwidth]{img/cran/top_ind_libio.png}
        \caption{Top paquetes con mayor grado de entrada en \textit{libraries.io}.}
        \label{fig:top_ind_libio_cran}
    \end{center}
\end{figure}


Desde la perspectiva del \textit{top} de \textit{in degree}, se observa una notable variación en
la evolución de los paquetes. Aproximadamente la mitad de los individuos que conformaban el
\textit{top} han sido reemplazados. Estos paquetes de reemplazo se caracterizan por ser nuevos
en la red, lo cual sugiere que están utilizando funcionalidades previamente desarrolladas para
generar nuevas capacidades.

Es común que aparezcan nuevos paquetes en este \textit{top}, dado el contexto de una red de
carácter científico. A medida que el software madura, es habitual que estos paquetes más
jóvenes se refactoricen con el tiempo, tendiendo a disminuir sus dependencias.

Por otro lado, existen paquetes que no solo se encontraban en el \textit{top} anterior,
sino que han incrementado sus dependencias. Este aumento puede ser indicativo de paquetes
cuya funcionalidad aún está en desarrollo y continúa evolucionando. \ref{fig:top_ind_scraped_cran}

\begin{figure}[h!]
    \begin{center}
        \includegraphics[width=1\textwidth]{img/cran/top_ind_scraped.png}
        \caption{Top paquetes con mayor grado de entrada en \textit{scraped}.}
        \label{fig:top_ind_scraped_cran}
    \end{center}
\end{figure}


Al observar la tendencia del incremento de dependencias en la red, se evidencia que, en
general, es común que el número de dependencias de los paquetes no experimente cambios
drásticos. La mayoría de los paquetes se sitúan en un rango de incremento de dependencias
de aproximadamente $\pm$10. \ref{fig:cran_dependencies_increase}


\begin{figure}[h!]
    \begin{center}
        \includegraphics[width=1\textwidth]{img/cran/dependencies_increase.png}
        \caption{Incremento de dependencias.}
        \label{fig:cran_dependencies_increase}
    \end{center}
\end{figure}

\subsection{El \textit{PageRank}}

A partir del análisis de la distribución de \textit{PageRank} en la red, se observa la presencia
de numerosos nodos con un \textit{PageRank} bajo, y a medida que se incrementa el valor del
\textit{PageRank}, la frecuencia de nodos disminuye gradualmente. Este patrón revela la
existencia de muchos nodos de baja importancia en la red, junto con un grupo reducido de
paquetes que son considerados importantes debido a sus dependencias, las cuales, a su vez,
son también dependencias relevantes en la red.

Al comparar las dos distribuciones, se aprecia una diferencia notable en la red
de \textit{libraries.io}, donde el valor máximo alcanzado por el \textit{PageRank} de
un paquete es mayor en comparación con la nueva red. Esta diferencia puede interpretarse
como un indicio de que, en la evolución de \textit{CRAN}, los paquetes más importantes
se han estabilizado, mientras que han surgido otros paquetes que están adquiriendo
relevancia. Como resultado, el \textit{PageRank} se ha distribuido de manera más equitativa
entre los paquetes de la red. \ref{fig:cran_pr_libio} \ref{fig:cran_pr_scraped}

\begin{figure}[h!]
    \begin{center}
        \includegraphics[width=1\textwidth]{img/cran/pr.png}
        \caption{Distribucion de \textit{PageRank} en \textit{libraries.io}.}
        \label{fig:cran_pr_libio}
    \end{center}
\end{figure}

\begin{figure}[h!]
    \begin{center}
        \includegraphics[width=1\textwidth]{img/cran/pr2.png}
        \caption{Distribucion de \textit{PageRank} en \textit{scraped}.}
        \label{fig:cran_pr_scraped}
    \end{center}
\end{figure}

El top de mayor \textit{PageRank} en \textit{libraries.io} nos brinda una visión de los paquetes que se
consideran como nodos centrales en la red de dependencias. Esto implica que son elementos fundamentales
para la funcionalidad y el rendimiento de otros paquetes, dado que poseen un mayor número de dependencias
directas e indirectas.

Estos paquetes tienden a ser dependientes de otros paquetes con alto \textit{PageRank}. Además, desde el
punto de vista de la vulnerabilidad, son considerados críticos debido a que suelen acumular una alta
dependencia transitiva. Esto significa que cualquier cambio o problema en estos paquetes centrales puede
tener un impacto significativo en todo el ecosistema de la red de dependencias.

Estos nodos de alto \textit{PageRank} juegan un papel crucial en el mantenimiento y la estabilidad de la
red de dependencias. Su importancia radica en su alta interconexion con otros paquetes similares. \ref{fig:cran_pr_libio_top}

\begin{figure}[h!]
    \begin{center}
        \includegraphics[width=1\textwidth]{img/cran/pr_top.png}
        \caption{Top \textit{PageRank} en \textit{libraries.io}.}
        \label{fig:cran_pr_libio_top}
    \end{center}
\end{figure}



En base a la introducción anterior, es esperable encontrar paquetes en el top de \textit{libraries.io}
que no estén presentes en el conjunto de datos de \textit{scraped}.

\begin{figure}[h!]
    \begin{center}
        \includegraphics[width=1\textwidth]{img/cran/pr_top2.png}
        \caption{Top \textit{PageRank} en \textit{scraped}.}
        \label{fig:cran_pr_scraped_top}
    \end{center}
\end{figure}

A partir del nuevo conjunto de datos \ref{fig:cran_pr_scraped_top}, se observa un descenso general en los
valores de PageRank en comparación con el conjunto de datos de \textit{libraries.io}. Además, se destaca
que un gran número de paquetes han ingresado al top de PageRank y no estaban presentes en \textit{libraries.io}.
Este hallazgo refuerza la teoría de que el PageRank, en el contexto de una red de dependencias, puede
considerarse un indicador de la vulnerabilidad de un paquete. Se puede inferir que estos paquetes recién
incorporados tienden a tener una presencia transitoria en la red y, debido a su relativa falta de madurez,
es probable que experimenten variaciones significativas a lo largo del tiempo.

Sin embargo, resulta más interesante centrar la atención en aquellos paquetes que mantienen un alto valor de
PageRank a lo largo del tiempo. Estos paquetes indican la presencia de dependencias importantes y altamente
transitivas en la red, lo que los hace potencialmente más vulnerables. Su capacidad para conservar su
importancia en el contexto de la evolución de la red sugiere que desempeñan un papel fundamental en el
funcionamiento y la estabilidad de otros paquetes. \ref{fig:Top 20 PageRank numero de dependencias transitivas en libraries.io}
\ref{fig:Top 20 PageRank numero de dependencias transitivas en scraped}

\begin{figure}[h!]
    \begin{center}
        \includegraphics[width=1\textwidth]{img/cran/pr_trans.png}
        \caption{Top 20 \textit{PageRank} numero de dependencias transitivas en libraries.io}
        \label{fig:Top 20 PageRank numero de dependencias transitivas en libraries.io}
    \end{center}
\end{figure}

\begin{figure}[h!]
    \begin{center}
        \includegraphics[width=1\textwidth]{img/cran/pr_trans2.png}dependidos
        \caption{Top 20 \textit{PageRank} numero de dependencias transitivas en scraped}
        \label{fig:Top 20 PageRank numero de dependencias transitivas en scraped}
    \end{center}
\end{figure}

Desde una perspectiva inversa, el uso del \textit{PageRank} nos permite identificar qué paquetes son dependencia
de otros paquetes importantes en la red de dependencias. Esto proporciona una visión de
la centralidad en términos de popularidad de un paquete. Al analizar el \textit{PageRank} bajo esta perspectiva, es
posible determinar qué paquetes son altamente requeridos por otros paquetes y, por lo tanto,
desempeñan un papel crucial en el funcionamiento de la red. \ref{fig:Top 20 PageRank paquetes en sraped}

\begin{figure}[h!]
    \begin{center}
        \includegraphics[width=1\textwidth]{img/cran/pr_inverted.png}
        \caption{Top 20 \textit{PageRank} paquetes en sraped}
        \label{fig:Top 20 PageRank paquetes en sraped}
    \end{center}
\end{figure}

Los paquetes mostrados en este top son los mas populares del ecosistema de \textit{R}.
Estos paquetes son los mas utilizados por otros paquetes, lo que los hace fundamentales para el
funcionamiento de la red de dependencias. No es raro que el paquete R, que implementa el \textit{core}
del lenguaje, sea el paquete mas popular.

\begin{figure}[h!]
    \begin{center}
        \includegraphics[width=1\textwidth]{img/cran/transitive_pr_scr.png}
        \caption{Top 20 \textit{PageRank} paquetes numero de dependencias transitivas en sraped}
        \label{fig:Top 20 PageRank paquetes numero de dependencias transitivas en sraped}
    \end{center}
\end{figure}

Estos paquetes a nivel de vulnerabilidad son los mas estables, ya que el numero de dependencias
transitivas que tienen es relativamente bajo. \ref{fig:Top 20 PageRank paquetes numero de dependencias transitivas en sraped}

\subsection{Impacto (\textit{Impact})}

Desde el punto de vista de esta metrica, podemos interpretar la vulnerabilidad de un paquete como el numero
de dependencias paquetes que se ven afectados por un cambio en el paquete. En este sentido, un paquete
vulnerable es aquel que tiene un alto impacto en la red de dependencias.

\begin{figure}[h!]
    \begin{center}
        \includegraphics[width=0.8\textwidth]{img/cran/impact_dist_libio.png}
        \caption{Distribución de \textit{Impact} en libraries.io}
        \label{fig:Distribución de Impact en libraries.io}
    \end{center}
\end{figure}

\begin{figure}[h!]
    \begin{center}
        \includegraphics[width=0.8\textwidth]{img/cran/impact_dist_scraped.png}
        \caption{Distribución de \textit{Impact} en scraped}
        \label{fig:Distribución de Impact en scraped}
    \end{center}
\end{figure}

A partir del análisis de la distribución del impacto, se observa una tendencia similar en ambos
conjuntos de datos. Se aprecia un ligero incremento en el impacto en el nuevo conjunto de datos,
pero en general los valores se mantienen estables. Este incremento puede atribuirse a la incorporación
de nuevos paquetes en la red, los cuales han adoptado dependencias existentes, lo que ha aumentado el
impacto de estas últimas. \ref{fig:Distribución de Impact en libraries.io} \ref{fig:Distribución de Impact en scraped}


En el top de paquetes con mayor \textit{impacto} se encuentran aquellos que son considerados los más populares
en la red, en términos de su influencia sobre otros paquetes. Es notable que los principales representantes
de este top también aparecen en el top del \textit{PageRank} a nivel de la red de paquetes. Desde el punto
de vista de la \textit{dependencia transitiva}, estos paquetes son aquellos que tienen la mayor cantidad de
dependientes en toda la red.

Al analizar el \textit{impacto} de estos paquetes, se observa un aumento en su valor, en algunos casos
significativo. Esto indica que estos paquetes han demostrado ser estables en el tiempo y en su implementación,
y que su funcionalidad es altamente útil para la red.

\begin{figure}[h!]
    \begin{center}
        \includegraphics[width=1\textwidth]{img/cran/impact_top_libio.png}
        \caption{Top paquetes con mayor \textit{Impact} en libraries.io}
        \label{fig:Top impact libraries.io}
    \end{center}
\end{figure}

\begin{figure}[h!]
    \begin{center}
        \includegraphics[width=1\textwidth]{img/cran/impact_top_scraped.png}
        \caption{Top paquetes con mayor \textit{Impact} en scraped}
        \label{fig:Top impact scraped}
    \end{center}
\end{figure}

Por otro lado, aquellos paquetes en el top que han experimentado una reducción en su \textit{impacto}
podrían indicar la aparición de otros paquetes con funcionalidades similares o mejoradas, los cuales han
sido elegidos como alternativas por los usuarios de la red. \ref{fig:Top impact libraries.io} \ref{fig:Top impact scraped}

\begin{table}[h!]
    \begin{center}
        \begin{tabular}{|l|r|r|r|}
            \hline
            \textbf{Package} & \textbf{libraries.io} & \textbf{scraped} & \textbf{increment} \\
            \hline
            utils            & 34259                 & 63116            & 28857              \\
            R                & 56281                 & 84501            & 28220              \\
            methods          & 35926                 & 61571            & 25645              \\
            lifecycle        & 1656                  & 26783            & 25127              \\
            stats            & 27322                 & 52319            & 24997              \\
            grDevices        & 19445                 & 43340            & 23895              \\
            graphics         & 19117                 & 41571            & 22454              \\
            cli              & 9144                  & 27514            & 18370              \\
            rlang            & 12362                 & 30475            & 18113              \\
            vctrs            & 9056                  & 24235            & 15179              \\
            \hline
        \end{tabular}
        \caption{Top 10 paquetes con mayor incremento en \textit{Impact}}
        \label{tab:Top 10 paquetes con mayor incremento en Impact}
    \end{center}
\end{table}

\begin{table}[h!]
    \begin{center}
        \begin{tabular}{|l|r|r|r|}
            \hline
            \textbf{Package} & \textbf{libraries.io} & \textbf{scraped} & \textbf{increment} \\
            \hline
            zeallot          & 9070                  & 67               & -9003              \\
            assertthat       & 9667                  & 704              & -8963              \\
            backports        & 9876                  & 2785             & -7091              \\
            crayon           & 10407                 & 4645             & -5762              \\
            reshape2         & 5217                  & 1232             & -3985              \\
            digest           & 11895                 & 8100             & -3795              \\
            plyr             & 6265                  & 2710             & -3555              \\
            lazyeval         & 4662                  & 1193             & -3469              \\
            formatR          & 1450                  & 90               & -1360              \\
            markdown         & 1511                  & 324              & -1187              \\
            \hline
        \end{tabular}
    \end{center}
    \caption{Top 10 paquetes con mayor decremento en \textit{Impact}}
    \label{tab:Top 10 paquetes con mayor decremento en Impact}
\end{table}


En la tablas del incremento podemos ver que los paquetes que han tenido un mayor incremento y decremento
en su impacto. \ref{tab:Top 10 paquetes con mayor incremento en Impact} \ref{tab:Top 10 paquetes con mayor decremento en Impact}



\begin{figure}[h!]
    \begin{center}
        \includegraphics[width=1\textwidth]{img/cran/impact_increase_dist.png}
        \caption{Incremento de \textit{Impact}}
        \label{fig:Incremento de Impact}
    \end{center}
\end{figure}

Si lo representamos graficamente se obtiene una tendencia similar a las distribuciones anteriores. \ref{fig:Incremento de Impact}

\subsection{Alcance (\textit{Reach})}

El alcance (\textit{Reach}) de un paquete p, equivale al número de sucesores transitivos
de p más 1 y mide el número de paquetes potencialmente afectados
por un defecto en p.

A la vista de las siguientes tablas podemos observar que  paquetes
son los que tienenen un mayor reach. \ref{tab:Top 10 paquetes con mayor Reach}

\begin{table}[h!]
    \begin{center}
        \begin{tabular}{|c|c|}
            \hline
            \textbf{libraries.io} & \textbf{scraped} \\
            \hline
            R, 14395              & R, 17223         \\
            methods, 10839        & methods, 15103   \\
            utils, 10516          & utils, 15037     \\
            stats, 9850           & stats, 14360     \\
            graphics, 7938        & graphics, 12809  \\
            grDevices, 7382       & grDevices, 12763 \\
            Rcpp, 7380            & grid, 9241       \\
            lattice, 6296         & rlang, 8993      \\
            tools, 6127           & lattice, 8945    \\
            grid, 5971            & magrittr, 8916   \\
            \hline
        \end{tabular}
        \caption{Top 10 paquetes con mayor \textit{Reach}}
        \label{tab:Top 10 paquetes con mayor Reach}
    \end{center}
\end{table}

Si comparamos los resultados para ambos conjuntos de datos nos damos cuenta de que
los representantes de este top en su gran medida coinciden, esto es explicable ya que
el \textit{Reach} de un paquete depende de los paquetes que lo usan, y estos paquetes
en R suelen ser los más populares. \ref{fig:Top reach libraries.io} \ref{fig:Top reach scraped}


\begin{figure}[h!]
    \begin{center}
        \includegraphics[width=1\textwidth]{img/cran/reach_top.png}
        \caption{Top paquetes con mayor \textit{Reach} en libraries.io}
        \label{fig:Top reach libraries.io}
    \end{center}
\end{figure}

\begin{figure}[h!]
    \begin{center}
        \includegraphics[width=1\textwidth]{img/cran/reach_top2.png}
        \caption{Incremento de \textit{Reach}}
        \label{fig:Top reach scraped}
    \end{center}
\end{figure}

Si observamos los paquetes con mayor incremento en \textit{Reach} \ref{tab:Top 10 paquetes con mayor incremento en Reach}
podemos ver que la mayoria estan presentes en el top, esta tabla nos da una idea de durante este periodo
de tiempo que paquetes han ganado mas dependientes.

\begin{table}[h!]
    \begin{center}
        \begin{tabular}{|c|c|c|c|}
            \hline
            \textbf{Paquete} & \textbf{libraries.io} & \textbf{scraped} & \textbf{increment} \\
            \hline
            lifecycle        & 1276                  & 8462             & 7186               \\
            grDevices        & 7382                  & 12763            & 5381               \\
            farver           & 48                    & 5020             & 4972               \\
            graphics         & 7938                  & 12809            & 4871               \\
            isoband          & 5                     & 4726             & 4721               \\
            utils            & 10516                 & 15037            & 4521               \\
            stats            & 9850                  & 14360            & 4510               \\
            splines          & 2016                  & 6466             & 4450               \\
            generics         & 457                   & 4895             & 4438               \\
            methods          & 10839                 & 15103            & 4264               \\
            \hline
        \end{tabular}
        \caption{Top 10 paquetes con mayor incremento en \textit{Reach}}
        \label{tab:Top 10 paquetes con mayor incremento en Reach}
    \end{center}
\end{table}

Es curioso ver que el paquete \textit{isoband} haya incrementado tanto su \textit{Reach}.
Este paquete es una implementación de la generación de bandas de confianza para curvas y mapas de contorno.
Resulta que este paquete fue protagonista de un incidente en el pasado:

El paquete \textit{isoband} estuvo en riesgo de ser archivado en \textit{CRAN}. La razón por la que este incidente causó
revuelo es que isoband es una dependencia de \textit{ggplot2} y cuando un paquete es eliminado de \textit{CRAN}, todos los
demás paquetes que dependen de él también son eliminados. Si isoband hubiera caído, \textit{ggplot2} estaría en
riesgo. Y esto habría desencadenado la eliminación de aún más paquetes. En total, la eliminación de
isoband habría llevado a la eliminación de 4747 paquetes\cite{r-bloggers_isoband_incident}.
Afortunadamente los desarrolladores de isoband pudieron solucionar el problema y el paquete no fue eliminado.
\cite{isoband_issue}.

Luego este incremento en el \textit{Reach} puede ser debido a que los desarrolladores de paquetes que dependen de isoband
se estan reenganchando al proyecto y actualizando sus paquetes para que sigan funcionando correctamente depues del incidente.



Por ultimo mostramos la distribucion de incremento de \textit{Reach} \ref{fig:Incremento de Reach}.

\begin{figure}[h!]
    \begin{center}
        \includegraphics[width=0.8\textwidth]{img/cran/reach_increment.png}
        \caption{Incremento de \textit{Reach}}
        \label{fig:Incremento de Reach}
    \end{center}
\end{figure}


\newpage


\section{La red de dependencias de Bioconductor}

El repositorio \textit{Bioconductor} para R es una plataforma científica de código abierto que ofrece
herramientas y paquetes especializados para el análisis y la interpretación de datos genómicos.
Surgió en 2001 para abordar desafíos específicos de la biología computacional y la genómica.
\textit{Bioconductor} es reconocido por su calidad, diversidad y enfoque colaborativo. Proporciona
herramientas estadísticas y bioinformáticas para el análisis de expresión génica, variantes genéticas y más.
Su importancia radica en su contribución al avance de la investigación genómica y promoción de la
reproducibilidad y transparencia científica en el campo de la genómica y la biología computacional.

El análisis de \textit{Bioconductor} es necesario como complemento al análisis de CRAN debido a que se
enfoca específicamente en el análisis de un subconjunto del ecosistema R. Mientras que CRAN es el
repositorio principal de paquetes para el lenguaje de programación R, \textit{Bioconductor} se centra
en ofrecer herramientas especializadas para el análisis de datos genómicos y biológicos.

Además, es importante destacar que, en el contexto de \textit{Bioconductor}, se ha enfrentado el
desafío de la falta de datos de referencia en \textit{libraries.io}, un repositorio de información
sobre paquetes de software. Esto ha llevado a la necesidad de abrir nuevos caminos y proporcionar
un conjunto de datos que no existía previamente. Al hacerlo, se está facilitando el análisis y el
estudio de los paquetes de \textit{Bioconductor}, y se está contribuyendo a la disponibilidad de
datos valiosos para la comunidad.

\subsection{El tamaño}

La red de \textit{Bioconductor} es una red relativamente pequeña\ref{tab:Medidas de la red de dependencias de Bioconductor}
en comparacion con los valores de las otras redes que hemos analizado.
Los valores de las métricas indican una red de dependencias de \textit{Bioconductor} de tamaño bajo,
con alta conectividad entre los nodos.

\begin{table}[h!]
    \begin{center}
        \begin{tabular}{|l|c|}
            \hline
            \textbf{Medida}                & \textbf{Valor} \\
            \hline
            Number of nodes                & 3509           \\
            Number of edges                & 28320          \\
            Average degree                 & 16.14          \\
            Average clustering coefficient & 0.077          \\
            \hline
        \end{tabular}
        \caption{Medidas de la red de dependencias de Bioconductor}
        \label{tab:Medidas de la red de dependencias de Bioconductor}
    \end{center}
\end{table}

El coeficiente de agrupamiento promedio de aproximadamente 0.0776 sugiere que la red tiene un nivel moderado de agrupamiento.
Esto implica que algunos nodos están conectados en grupos o comunidades, pero no existe una fuerte tendencia hacia la formación
de comunidades densamente interconectadas en toda la red. Esto puede indicar que existen diferentes grupos temáticos o
funcionales dentro de la red de dependencias de \textit{Bioconductor}, pero también existen conexiones entre estos grupos.
Este coeficiente de agrupamiento es mucho menor que el de \textit{CRAN}, que era de 0.15.

Respecto al grado medio podemos decir que es un poco mayor que el de \textit{CRAN},
para el que teniamos un valor de 12.13.

\subsection{El grado}

A la vista de la distribución de \emph{grado} \ref{fig:bioconductor_degree_dist}, se puede observar una distribución
de \emph{ley de potencia}, lo cual es común en las \emph{redes de dependencias}.
En la gráfica, se puede apreciar que a partir de un \emph{grado} de 50, el
número de \emph{nodos} comienza a reducirse de manera exponencial, quedando
solo unos pocos representantes. Como era de esperar, se observan \emph{hubs}
o \emph{nodos de alto grado}, con valores cercanos a 2000. Esta presencia
de \emph{hubs} indica la existencia de elementos altamente conectados que
desempeñan un papel crucial en la estructura de la red de dependencias.

\begin{figure}[h!]
    \begin{center}
        \includegraphics[width=1\textwidth]{img/bioconductor/degree_dist.png}
        \caption{Distribucion de grado de la red de dependencias de Bioconductor}
        \label{fig:bioconductor_degree_dist}
        \caption{Distribucion de grado de la red de dependencias de Bioconductor}
    \end{center}
\end{figure}

\subsubsection{El grado de salida \textit{Out degree}}

La distribución del \emph{grado de salida} en la red analizada presenta similitudes con
la distribución de la red de CRAN. En este caso, el \emph{grado de salida máximo} es
notablemente inferior debido al menor tamaño de la red, pero la tendencia general es
similar. Se puede apreciar que a partir de un \emph{grado de salida} de 10, la frecuencia
de individuos en la red se reduce significativamente. La mayoría de los nodos tienen un
\emph{grado de salida} inferior a 10, siendo estos el grupo más numeroso en la red.\ref{fig:bioconductor_out_degree_dist}

\begin{figure}[h!]
    \begin{center}
        \includegraphics[width=0.8\textwidth]{img/bioconductor/out_degree_dist.png}
        \caption{Distribucion de grado de salida de la red de dependencias de Bioconductor}
        \label{fig:bioconductor_out_degree_dist}
    \end{center}
\end{figure}

En cuanto a los paquetes que representan el \emph{mayor grado de salida}\ref{fig:bioconductor_out_degree}
en la red analizada, se observa que la mayoría de ellos también aparecen en el \emph{top} de la red de CRAN.
Este hallazgo es coherente, dado que, como se mencionó anteriormente, Bioconductor es un
subconjunto de los paquetes de R, y es esperable que estos paquetes tengan una importancia
similar en ambos repositorios.
Además, se observa que en este \emph{top} se encuentran los paquetes más importantes y exclusivos
de Bioconductor, como BiocGenerics o Biobase, entre otros. Esto nos indica que la red de Bioconductor
está especializada en el ámbito de la bioinformática, ya que estos paquetes desempeñan un papel crucial
en el análisis y procesamiento de datos biológicos. Su presencia en el \emph{top} indica la importancia
de la especialización de Bioconductor en este campo.

\begin{figure}[h!]
    \begin{center}
        \includegraphics[width=1\textwidth]{img/bioconductor/top_out_degree.png}
        \caption{Top 10 de los paquetes con mayor grado de salida en la red de dependencias de Bioconductor}
        \label{fig:bioconductor_out_degree}
    \end{center}
\end{figure}


\subsubsection{El grado de entrada \textit{In degree}}

La distribución de \emph{in degree}\ref{fig:bioconductor_in_degree_dist} en la red de Bioconductor muestra algunas diferencias con respecto
a la red de CRAN. Se observa que hay un menor número de nodos en la red de Bioconductor en comparación
con CRAN. Además, la frecuencia máxima de nodos con \emph{in degree} ya no se encuentra en los valores
1 o 2, como ocurre en CRAN. En Bioconductor, se puede apreciar que hay más nodos con \emph{in degree}
en el rango de 10 a 20 en comparación con aquellos con un valor de 1.

Otro aspecto a tener en cuenta es que el \emph{in degree} máximo en la red de Bioconductor tiene un
valor de 85, mientras que en CRAN el valor máximo es de 50. Esto indica que en Bioconductor existen
nodos con un mayor número de dependencias entrantes, lo que podría reflejar la complejidad y la
interconexión de los paquetes en Bioconductor. Estas diferencias en la distribución de \emph{in degree}
evidencian las particularidades y características propias de la red de Bioconductor en comparación con
la red de CRAN.

\begin{figure}[h!]
    \begin{center}
        \includegraphics[width=0.8\textwidth]{img/bioconductor/in_degree_dist.png}
        \caption{Distribucion de grado de entrada de la red de dependencias de Bioconductor}
        \label{fig:bioconductor_in_degree_dist}
    \end{center}
\end{figure}

Los paquetes que se encuentran en el \emph{top} de \emph{grado de entrada} \ref{fig:bioconductor_in_degree} son aquellos que presentan
un mayor número de dependencias en la red, sin tener en cuenta la transitividad de las conexiones.
Esto implica que estos paquetes son altamente dependientes de otros paquetes en la red de Bioconductor en el primer nivel de
profundidad del arbol de dependencias, lo que sugiere su importancia y relevancia en el ecosistema de Bioconductor.

\begin{figure}[h!]
    \begin{center}
        \includegraphics[width=1\textwidth]{img/bioconductor/top_in_degree.png}
        \caption{Top 10 de los paquetes con mayor grado de entrada en la red de dependencias de Bioconductor}
        \label{fig:bioconductor_in_degree}
    \end{center}
\end{figure}

\subsection{El PageRank}

La distribución de \textit{PageRank} \ref{fig:bioconductor_pagerank_dist} en la red de \textit{Bioconductor} no revela patrones claros
y sencillos de analizar. Observando el gráfico, podemos determinar que la frecuencia máxima de nodos
con el mismo valor de \textit{PageRank} se sitúa aproximadamente entre \textit{0.0002} y \textit{0.0005}.
El rango normal de valores se encuentra entre \textit{0.0002} y \textit{0.002}, aunque en algunos casos
particulares se alcanzan valores de \textit{0.003} o \textit{0.004}. Estos valores representan la importancia
relativa de los nodos en la red, teniendo en cuenta tanto las conexiones directas como las conexiones
indirectas a través de otros nodos en la red. Sin embargo, debido a la complejidad y tamaño de la red
de \textit{Bioconductor}, es necesario realizar un análisis más detallado y específico para comprender
plenamente la distribución y significado de los valores de \textit{PageRank} en esta red.

\begin{figure}[h!]
    \begin{center}
        \includegraphics[width=1\textwidth]{img/bioconductor/pagerank_dist.png}
        \caption{Distribucion de PageRank de la red de dependencias de Bioconductor}
        \label{fig:bioconductor_pagerank_dist}
    \end{center}
\end{figure}


Respecto al top de dependencias con mayor \textit{PageRank} \ref{fig:bioconductor_top_pagerank_dependencies},
estos paquetes representan una vulnerabilidad significativa debido a su alto \textit{in degree} y la interconexión entre ellos. Desde el punto de vista
del autor, sería recomendable evitar depender de alguno de estos paquetes, ya que se encuentran en un entorno
donde la probabilidad de vulnerabilidad es mayor. La dependencia de estos paquetes puede aumentar la propagación
de posibles problemas o errores a lo largo de la red de \textit{Bioconductor}. Por lo tanto, es esencial
considerar alternativas y evaluar cuidadosamente las dependencias al desarrollar o utilizar aplicaciones
basadas en esta red.

\begin{figure}[h!]
    \begin{center}
        \includegraphics[width=1\textwidth]{img/bioconductor/top_pagerrank_dependencies.png}
        \caption{Top dependencias con mayor PageRank en la red de dependencias de Bioconductor}
        \label{fig:bioconductor_top_pagerank_dependencies}
    \end{center}
\end{figure}

Realizando una inversionde la red de dependencias de Bioconductor, se obtiene la red de paquetes, desde este punto de vista
el \textit{PageRank} ahora esta ponderando la importancia de los paquetes en la red, es decir, aquellos paquetes que son dependencias
de otros paquetes importantes, son considerados como los mas importantes en la red. En la figura \ref{fig:bioconductor_top_pagerank_packages}
se muestran los paquetes con mayor \textit{PageRank} bajo este punto de vista de Bioconductor.

\begin{figure}[h!]
    \begin{center}
        \includegraphics[width=1\textwidth]{img/bioconductor/top_pagerrank_packages.png}
        \caption{Top paquetes con mayor PageRank en la red de dependencias de Bioconductor}
        \label{fig:bioconductor_top_pagerank_packages}
    \end{center}
\end{figure}








\newpage

\section{La red de dependencias de PyPI}

En el ámbito del lenguaje de programación \textit{Python}, nos enfocamos en el análisis de \textit{PyPI}
(\textit{Python Package Index}). \textit{PyPI} es un repositorio ampliamente adoptado en la comunidad de
\textit{Python} debido a su facilidad de uso, su naturaleza de código abierto y su extensa colección de
paquetes.\footnote{Es importante mencionar que existen otros repositorios interesantes como \textit{Conda}.}

La facilidad de uso de \textit{PyPI} se deriva de su diseño intuitivo y las funcionalidades que ofrece
para la gestión de paquetes en \textit{Python}. Los desarrolladores pueden acceder a \textit{PyPI} como
una fuente centralizada para descubrir, descargar e instalar una amplia variedad de paquetes y bibliotecas
desarrollados por la comunidad.

En el análisis comparativo entre los datos proporcionados por \textit{libraries.io} y los recolectados en
este trabajo, se ha revelado una sorprendente tendencia en el número de paquetes presentes en \textit{PyPI}.
Se ha observado un aumento significativo en la cantidad de paquetes disponibles en \textit{PyPI} en
comparación con datos anteriores.

Este hallazgo sugiere un crecimiento notable en el ecosistema de paquetes de \textit{Python} y evidencia
el interés y participación de la comunidad en \textit{PyPI}. Este aumento en el número de paquetes puede
ser atribuido a diversos factores, como la creciente popularidad de \textit{Python} como lenguaje de
programación, el aumento en la adopción de \textit{Python} en diferentes campos de aplicación y el
creciente número de contribuyentes que comparten sus proyectos y soluciones a través de \textit{PyPI}.


\begin{figure}[h!]
    \begin{center}
        \includegraphics[width=0.6\textwidth]{img/pypi/bar_common_packages.png}
        \caption{Comparacion de la cantidad de paquetes en PyPI entre los datos de libraries.io (2020) y los recolectados en este trabajo (2023).}
        \label{fig:pipy_common_packages_bar}
    \end{center}
\end{figure}

\begin{figure}[h!]
    \begin{center}
        \includegraphics[width=0.6\textwidth]{img/pypi/circ_common_packages.png}
        \caption{Paquetes comunes y no comunes entre los datos de libraries.io (2020) y los recolectados en este trabajo (2023).}
        \label{fig:pipy_common_packages_circle}
    \end{center}
\end{figure}

El análisis de los datos revela que aproximadamente el \textit{57 \%} de los paquetes que estaban
presentes en el conjunto de datos antiguo de PyPI se han mantenido en el nuevo conjunto de datos
\footnote{Este porcentaje se basa en datos obtenidos experimentalmente.}. Este porcentaje
relativamente elevado indica que la red de paquetes en PyPI ha logrado mantener una cantidad considerable
de paquetes de manera estable a lo largo del tiempo \ref{tab:pypi_common_packages}.

\begin{table}[h!]
    \begin{center}
        \begin{tabular}{|l|r|}
            \hline
            \textbf{Descripción}                               & \textbf{Cantidad} \\
            \hline
            Paquetes en libraries.io                           & 41,621            \\
            Paquetes en scraped                                & 197,460           \\
            Paquetes comunes                                   & 24,001            \\
            Paquetes de libraries.io no disponibles en scraped & 17,620            \\
            Paquetes en scraped que no estan en libraries.io   & 173,459           \\
            \hline
        \end{tabular}
    \end{center}
    \label{tab:pypi_common_packages}
    \caption{Comparación de paquetes en PyPI entre los datos de libraries.io (2020) y los recolectados en este trabajo (2023).}
\end{table}

Es importante destacar que este conjunto de paquetes que se ha mantenido representa aproximadamente
el \textit{12.15\%} del total de paquetes disponibles en PyPI en la actualidad. Esta proporción
da una idea de la estabilidad relativa de la red, ya que una parte significativa de los
paquetes ha logrado mantener su presencia en PyPI a pesar de los posibles cambios y
actualizaciones\footnote{Los datos se refieren al momento de la última actualización y pueden estar
    sujetos a cambios futuros.}.

\begin{figure}[h!]
    \begin{center}
        \includegraphics[width=1\textwidth]{img/pypi/pypi_popularity.png}
        \caption{Popularidad de Python a lo largo del tiempo.}
        \label{fig:pypi_popularity}
    \end{center}
\end{figure}


Esta tendencia en la popularidad de Python se ve reflejada en las estadísticas realizadas por la
empresa TIOBE \ref{fig:pypi_popularity}\footnote{\url{https://www.tiobe.com/tiobe-index/python/}}.


\subsubsection{Grado de la red}

Al examinar la distribución de grado en ambos conjuntos de datos, se observa que sigue una distribución
de ley de potencias\footnote{La distribución de ley de potencias es una característica común en las redes
    de dependencias.}.

Al analizar el gráfico, se evidencia que el grado máximo alcanzado por un paquete ha
experimentado un incremento significativo. Tomando como punto de referencia el número de nodos con grado
\textit{1000}, se puede apreciar una diferencia sustancial entre la red de \textit{libraries.io} y los
nuevos datos recolectados. Mientras que en \textit{libraries.io} se registran menos de \textit{10} paquetes
con dicho grado, en los nuevos datos se han identificado aproximadamente \textit{40} individuos
\footnote{Estos datos se basan en el análisis realizado en una fecha específica y pueden estar sujetos
    a cambios en el tiempo.}.

\begin{figure}[h!]
    \begin{center}
        \includegraphics[width=0.8\textwidth]{img/pypi/librariesio_degree_distribution.png}
        \caption{Distribucion de grado de PyPI para libraries.io.}
        \label{fig:pypi_librariesio_degree_distribution}
    \end{center}
\end{figure}

\begin{figure}[h!]
    \begin{center}
        \includegraphics[width=0.8\textwidth]{img/pypi/scraped_degree_distribution.png}
        \caption{Distribucion de grado de PyPI para los datos recolectados en este trabajo (2023).}
        \label{fig:pypi_scraped_degree_distribution}
    \end{center}
\end{figure}

Además, al calcular el grado promedio de los grafos correspondientes a \textit{libraries.io} y el nuevo
conjunto, se obtiene un valor de \textit{2.73} y \textit{4.35}, respectivamente. Estos valores indican
que, en promedio, cada paquete en la red de \textit{libraries.io} está conectado a alrededor de
\textit{2.73} otros paquetes, mientras que en los nuevos datos, cada paquete está conectado a
aproximadamente \textit{4.35} otros paquetes\footnote{Estos cálculos se realizaron utilizando una
    metodología específica y pueden variar dependiendo de la definición de conexión utilizada.}.

Estas estadísticas revelan cambios importantes en la estructura de la red de dependencias de paquetes.
El incremento en el grado máximo y el aumento en el grado promedio indican una mayor interconectividad
y complejidad en la red, lo cual puede ser atribuido al crecimiento y la evolución del ecosistema de
paquetes en Python.

Es importante destacar que la distribución de ley de potencias y la presencia de paquetes con grados
altos en la red tienen implicaciones significativas en términos de la propagación de dependencias y la
influencia de ciertos paquetes en la comunidad\footnote{Estas implicaciones pueden afectar la
    estabilidad, la modularidad y la confiabilidad del ecosistema de paquetes en Python.}.

\textbf{Grado de salida (\textit{out degree})}

El \textit{out degree} es una métrica que proporciona información
sobre el número de dependientes de un paquete dado. En el contexto de \textit{libraries.io},
analizando los datos, podemos identificar los paquetes que tienen más dependencias, es decir,
los que están en el \textit{Top} de las dependencias más utilizadas.


Los resultados obtenidos revelan una tendencia general en la cual las dependencias más populares han
experimentado un aumento en su popularidad, siendo ahora requeridas por un mayor número de paquetes
en la red de dependencias. En particular, se observa un incremento significativo en la popularidad
de las bibliotecas \textit{requests}, \textit{numpy}, \textit{pandas} y \textit{pytest} en comparación
con otros paquetes presentes en este ranking.

Estos hallazgos indican que estas bibliotecas han adquirido una mayor relevancia y utilidad en el
desarrollo de proyectos y aplicaciones en el entorno de Python.

\begin{figure}[h!]
    \begin{center}
        \includegraphics[width=1\textwidth]{img/pypi/libio_t20_outd_comparison.png}
        \caption{Top de paquetes con mayor grado de salida en PyPI para libraries.io (2020).}
        \label{fig:pypi_libio_outd_comparison}
    \end{center}
\end{figure}

\begin{figure}[h!]
    \begin{center}
        \includegraphics[width=1\textwidth]{img/pypi/libio_scraped_t20_comparation.png}
        \caption{Top de paquetes con mayor grado de salida en PyPI para scraped (2023).}
        \label{fig:pypi_scraped_outd_comparison}
    \end{center}
\end{figure}

En este análisis de ranking, se observa que ciertos paquetes se han mantenido con respecto al ranking
anterior, lo cual indica su relevancia a lo largo del tiempo. Estos
paquetes incluyen \textit{PyYAML}, \textit{click}, \textit{numpy}, \textit{pandas}, \textit{pytest},
\textit{pyyaml}, \textit{requests}, \textit{setuptools} y \textit{six}. Su presencia continua en el
ranking sugiere que son dependencias fundamentales y ampliamente utilizadas en proyectos y aplicaciones
de Python\footnote{La relevancia y utilidad de estos paquetes se basa en la percepción de la comunidad
    de desarrolladores de Python y puede variar dependiendo del contexto y los requisitos del proyecto}.

Por otro lado, se identifican paquetes que han ascendido en el ranking en comparación con la clasificación
anterior. Estos paquetes incluyen \textit{black}, \textit{coverage}, \textit{flake8}, \textit{matplotlib},
\textit{odoo}, \textit{python}, \textit{scikit}, \textit{scipy}, \textit{sphinx}, \textit{tqdm} y
\textit{typing}. Su ascenso en el ranking puede ser atribuido a su creciente popularidad y utilidad en el
desarrollo de proyectos de Python, ya que son bibliotecas ampliamente conocidas y utilizadas por la
comunidad de desarrolladores\footnote{El ascenso en el ranking puede deberse a mejoras en funcionalidad,
    adopción en proyectos populares u otros factores que influyen en su popularidad}.

Por último, se muestra la distribución de \textit{out degree} para ambos conjuntos de datos \ref{fig:pypi_libio_outd_dist} \ref{fig:pypi_scraped_outd_dist}

\begin{figure}[h!]
    \begin{center}
        \includegraphics[width=0.8\textwidth]{img/pypi/outd_libio_dist.png}
        \caption{Distribución de \textit{Out degree} para libraries.io (2020)}
        \label{fig:pypi_libio_outd_dist}
    \end{center}
\end{figure}

\begin{figure}[h!]
    \begin{center}
        \includegraphics[width=0.8\textwidth]{img/pypi/outd_scraped_dist.png}
        \caption{Distribución de \textit{Out degree} para scraped (2023)}
        \label{fig:pypi_scraped_outd_dist}
    \end{center}
\end{figure}

Al analizar los gráficos de las distribuciones de \textit{out degree}, se observa una tendencia similar
en ambos conjuntos. Se evidencia un incremento general en el número total de paquetes, lo que indica un
crecimiento continuo en el ecosistema de paquetes.

Se ha observado un aumento en el número de paquetes con un grado de salida bajo, lo que indica que estos
paquetes tienen menos dependencias externas. Este fenómeno puede deberse a la introducción de paquetes
más autónomos y autosuficientes en el ecosistema de Python, lo que reduce la necesidad de depender de
otros paquetes para su funcionamiento.

Por otro lado, se ha identificado un incremento \ref{fig:dependents_increase} en el grado de salida de los paquetes más populares.
Esto indica que estos paquetes están siendo cada vez más utilizados como dependencias por otros paquetes
en la comunidad de Python. Este aumento en el grado de salida de los paquetes populares puede ser atribuido
a su funcionalidad ampliamente reconocida y popularidad.

\begin{figure}[h!]
    \begin{center}
        \includegraphics[width=0.8\textwidth]{img/pypi/dependents_increase.png}
        \caption{Incremento del \textit{Out degree} en PyPI (2010-2023)}
        \label{fig:dependents_increase}
    \end{center}
\end{figure}


\textbf{Grado de entrada (\textit{In degree})}

Esta métrica nos da una idea del número de dependencias que tiene un paquete dado.
Analizados los datos de libraries.io y representados sobre un grafico obtenemos los siguientes resultados: \ref{fig:pypi_libio_ind_comparison}

\begin{figure}[h!]
    \begin{center}
        \includegraphics[width=1\textwidth]{img/pypi/libio_t20_ind_comparison.png}
        \caption{Top paquetes con mayor \textit{In degree} en libraries.io (2020)}
        \label{fig:pypi_libio_ind_comparison}
    \end{center}
\end{figure}

Se observa una tendencia decreciente en el número de dependencias entre los paquetes con mayor In degree
dentro del conjunto de bibliotecas de \textit{libraries.io}, lo cual sugiere una inclinación natural de los
paquetes hacia una disminución de las dependencias requeridas. Además, se aprecia que una proporción significativa
de estos paquetes, caracterizados por una elevada cantidad de dependencias, ha experimentado una desaparición,
representando aproximadamente un 25 \% de las instancias evaluadas. Por consiguiente, se puede inferir que,
en general, la presencia de un alto número de dependencias no suele correlacionarse con la estabilidad de
los paquetes en el repositorio.

\begin{figure}[h!]
    \begin{center}
        \includegraphics[width=1\textwidth]{img/pypi/scraped_t20_ind_comparison.png}
        \caption{Top paquetes con mayor \textit{In degree} en scraped (2023)}
        \label{fig:pypi_scraped_ind_comparison}
    \end{center}
\end{figure}

Si realizamos este mismo análisis con el top 20 de los paquetes con más \textit{In degree} en la actualidad \ref{fig:pypi_scraped_ind_comparison}, podemos llegar a
conclusiones similares. Se observa una tendencia decreciente en el número de dependencias de los paquetes
más influyentes, lo cual sugiere una reducción en la cantidad de paquetes que dependen directamente de ellos.
Esto puede indicar cambios en las estrategias de desarrollo, la aparición de alternativas o la evolución de la
comunidad de desarrolladores.

Estos hallazgos resaltan la importancia de considerar tanto el In degree como el grado de salida de
los paquetes al analizar la estabilidad y la evolución de la red de dependencias en el ecosistema de Python.

Como se puede apreciar, el \textit{95 \%} de los paquetes pertenecientes a este conjunto presentan una ausencia
de dependencias.
Si profundizamos en el tema, podemos apreciar que estos paquetes son de reciente aparición. La falta de
dependencias en los paquetes puede ser resultado de su diseño modular, el uso de
bibliotecas internas o la falta de necesidad de dependencias externas.

Cabe destacar un caso particular, el paquete denominado \textit{apache-airflow}
\footnote{\url{https://pypi.org/project/apache-airflow/}}, el cual ha experimentado un considerable aumento
en el número de dependencias, pasando de 41 a 185. La explicación que se atribuye a este fenómeno es la
incorporación de nuevas funcionalidades, dado que se trata de un paquete con cierta popularidad. No obstante,
desde la perspectiva del autor de este Trabajo Final de Grado, se recomienda a los desarrolladores reducir
al máximo este número de dependencias para mejorar su estabilidad\footnote{El aumento en el número de dependencias
    puede aumentar la complejidad y la posibilidad de conflictos en el entorno de desarrollo. Se sugiere evaluar
    cuidadosamente las dependencias necesarias y buscar alternativas más ligeras o mejor optimizadas si es posible}.

\begin{figure}[h!]
    \begin{center}
        \includegraphics[width=0.8\textwidth]{img/pypi/ind_libio_d.png}
        \caption{Distribucion del \textit{In degree} en libraries.io (2020)}
        \label{fig: Distribucion del In degree en libraries.io}
    \end{center}
\end{figure}

\begin{figure}[h!]
    \begin{center}
        \includegraphics[width=0.8\textwidth]{img/pypi/ind_scraped_dist.png}
        \caption{Distribucion del \textit{In degree} en scraped (2023)}
        \label{fig: Distribucion del In degree en scraped}
    \end{center}
\end{figure}

En el análisis de la distribución de In degree \ref{fig: Distribucion del In degree en libraries.io},
se ha observado una alta frecuencia de nodos con un bajo grado,
lo que indica la presencia de numerosos paquetes que no son utilizados como dependencias. Además, se ha identificado
una clara tendencia descendente en la frecuencia a medida que aumenta el \textit{In degree}.
Esto claramente representa una \textit{distribucion Power Law} \cite{enwiki:1160892030}.

La disminución en la frecuencia se vuelve significativa a medida que se alcanza un \textit{In degree} del orden
de $10^2$, donde la frecuencia se reduce a un único nodo por caso. Esto implica que existe una disminución drástica
en la cantidad de paquetes con un \textit{In degree} alto, lo cual sugiere que son menos comunes aquellos paquetes
que son ampliamente utilizados como dependencias por otros.

Si observamos la evolución \ref{fig: Distribucion del In degree en scraped}, se observa una tendencia similar en ambos casos, aunque en el estado actual se evidencia
un aumento en el número de nodos. La forma de la distribución se mantiene similar, pero se aprecia un considerable
incremento en el \textit{In degree}. Si consideramos la conclusión anteriormente obtenida, podemos constatar que
también se cumple en este caso, dado que el incremento en la frecuencia implica una disminución en el grado de
entrada.

\subsubsection{Pagerank}

La métrica de \textit{PageRank}\footnote{PageRank es un algoritmo utilizado para establecer un ranking de importancia en la web.}
Permite establecer un ranking de importancia respecto a los paquetes. A la vista de la distribución obtenida de la red
de \textit{libraries.io}, una conclusión que se puede extraer es que la mayoría de las dependencias no son consideradas
importantes, ya que pertenecen al primer grupo, con una frecuencia considerablemente alta pero una baja relevancia en
términos de \textit{PageRank}. Sin embargo, existe un grupo más reducido pero significativo que presenta una importancia
media, pero son relevantes a nivel de que tienen múltiples enlaces provenientes de diferentes paquetes. Estas dependencias
comunes desempeñan un papel crítico en la interconexión de los diferentes componentes de la red. Además, se destaca
la presencia de un conjunto selecto de dependencias con un alto \textit{PageRank}, lo que indica su gran popularidad
y relevancia en la red de paquetes. En este grupo, las dependencias son enlazadas por muchas otras dependencias,
pero no tienen dependencias propias.

En concreto, estas son las dependencias más importantes a tener en cuenta debido a que suelen ser común su
aparición \ref{fig:Top 20 pagerank en libraries.io}.

\begin{figure}[h!]
    \begin{center}
        \includegraphics[width=1\textwidth]{img/pypi/libio_t20_pr_comparison.png}
        \caption{Top 20 \textit{PageRank} en libraries.io (2020)}
        \label{fig:Top 20 pagerank en libraries.io}
    \end{center}
\end{figure}


En términos de vulnerabilidad, los paquetes más críticos son aquellos que, si experimentan problemas o inestabilidad,
pueden tener un impacto significativo en toda la red de dependencias. Una dependencia crítica con múltiples enlaces
entrantes puede generar la propagación de errores o vulnerabilidades a través de los paquetes que dependen de
ella\footnote{Una dependencia crítica es aquella que, al presentar problemas, puede afectar negativamente a
    otros paquetes que dependen de ella, causando errores o vulnerabilidades en cadena.}.

Si visualizamos el número de dependencias transitivas \ref{fig:Dependencias transitivas del top 20 pagerank en libraries.io}
de estos paquetes, podemos obtener conclusiones más precisas.
Existe un grupo de paquetes con un menor número de dependencias transitivas, lo que los hace menos vulnerables en
comparación con el resto. Además, tener un alto \textit{PageRank} en estos paquetes implica que son más confiables
en términos de dependencias\footnote{Un paquete con un alto \textit{PageRank} y un bajo número de dependencias
    transitivas es considerado confiable y menos propenso a problemas de vulnerabilidad, ya que su estructura de
    dependencias es más simple y controlada.}.

\begin{figure}[h!]
    \begin{center}
        \includegraphics[width=1\textwidth]{img/pypi/transitive libraries.png}
        \caption{Dependencias transitivas del top 20 \textit{PageRank} en libraries.io (2020)}
        \label{fig:Dependencias transitivas del top 20 pagerank en libraries.io}
    \end{center}
\end{figure}

También se observa que en este grupo de paquetes, las dependencias transitivas son en promedio
más bajas\footnote{El número de dependencias transitivas en este grupo de paquetes tiende a
    ser menor en comparación con otros grupos, lo que indica una estructura más ligera y menos
    compleja en términos de dependencias.}.

Al analizar los resultados, se puede ver que los paquetes presentes en este grupo han
experimentado una evolución considerable \ref{fig:Top PageRank en scraped}, con una reducción significativa en
su \textit{PageRank}\footnote{El \textit{PageRank} de estos paquetes ha disminuido
    en comparación con mediciones anteriores.}. Esto se explica por la desaparición de algunos
de estos paquetes, la aparición de otros nuevos que han reemplazado su importancia y
la evolución de los propios paquetes hacia una mayor estabilización, lo que ha disminuido
su vulnerabilidad.

\begin{figure}[h!]
    \begin{center}
        \includegraphics[width=1\textwidth]{img/pypi/t20_dep_pr_scraped.png}
        \caption{Top \textit{PageRank} en scraped (2023)}
        \label{fig:Top PageRank en scraped}
    \end{center}
\end{figure}

Al examinar el nuevo conjunto de datos, se puede observar una tendencia similar a la anterior.
El aumento en el número de paquetes se refleja en una alta frecuencia de paquetes con un bajo
PageRank. Además, se observa una disminución general del valor del \textit{PageRank} en la mayoría de
la red. Esta disminución del \textit{PageRank} puede interpretarse como una mejora en términos de
vulnerabilidad.

Una conclusión que se puede extraer es que el crecimiento en el número de paquetes ha llevado
a una mayor presencia de paquetes con un bajo PageRank. Esto sugiere que hay una mayor
proporción de paquetes menos importantes en la red.

En este top se pueden observar las conclusiones previamente mencionadas en relación a la red de
dependencias. La mayoría de los paquetes en este conjunto son nuevos, lo que ha llevado a una
disminución del \textit{PageRank} en comparación con el caso anterior. Sin embargo, es
interesante destacar que algunos paquetes, como \textit{c3tools} y
\textit{gftools},
se mantienen en el top, lo que sugiere que han resistido bien el paso del tiempo y podrían
considerarse estables en la red, a pesar de tener una mayor probabilidad de
vulnerabilidad.

Además, se puede observar que los tres paquetes principales en este conjunto tienen
un \textit{PageRank} considerablemente más alto que el resto. Esta diferencia en
el \textit{PageRank} podría indicar que estos paquetes son especialmente relevantes
en la red de dependencias\footnote{Los tres paquetes principales en este conjunto son
    altamente influyentes y desempeñan un papel crucial en la interconexión de otros paquetes
    en la red de dependencias.}.

\begin{figure}[h!]
    \begin{center}
        \includegraphics[width=1\textwidth]{img/pypi/t20_pkg_pr_scr.png}
        \caption{Top \textit{PageRank} paquetes en scraped (2023)}
        \label{fig:Top PageRank paquetes en scraped}
    \end{center}
\end{figure}

Si invertimos el grafo de nuestra red de dependencias \ref{fig:Top PageRank paquetes en scraped}, podemos estudiar el \textit{PageRank}
desde el punto de vista de la relevancia del paquete en la red. Un alto \textit{PageRank}
implica que el paquete tiene una cantidad significativa de enlaces entrantes desde otros paquetes
importantes. Esto sugiere que el paquete es visto como una fuente confiable de información o
recursos dentro de la red. En otras palabras, es más probable que los otros paquetes dependan
del paquete con un alto \textit{PageRank} para obtener información o llevar a cabo determinadas
tareas.

Un paquete con un alto \textit{PageRank} puede ser considerado crucial en términos de la funcionalidad
o el rendimiento de la red. Es probable que los otros paquetes dependan directa o indirectamente de
él para llevar a cabo sus propias funciones o tareas.

Como se puede ver en el top que mostramos a continuación, aparecen los paquetes más conocidos y
comúnmente usados de Python para los dos conjuntos de datos.

\subsubsection{Impacto (Impact)}

El impacto de un paquete se refiere al número de dependencias que se verían afectadas si
ocurriera un defecto en ese paquete. Esta métrica podría utilizarse para evaluar la criticidad
o importancia de un paquete en la red de dependencias y ayudar en la identificación de los
paquetes que tienen un mayor impacto en el sistema en caso de fallos.

\begin{table}[h!]
    \centering
    \caption{Comparación entre paquetes obtenidos de libraries.io (2020) y scraped (2023) para la métrica \textit{Impact}}
    \begin{tabular}{|c|c|c|}
        \hline
        \textbf{libraries.io} & \textbf{scraped}   \\
        \hline
        six, 36757            & numpy, 448177      \\
        certifi, 18739        & six, 424014        \\
        requests, 17740       & python, 422180     \\
        pyparsing, 14111      & importlib, 420861  \\
        packaging, 13433      & typing, 417287     \\
        appdirs, 12619        & colorama, 416663   \\
        setuptools, 11803     & matplotlib, 414520 \\
        python-dateutil, 9825 & chardet, 413067    \\
        numpy, 7396           & Cython, 412181     \\
        pytz, 6878            & click, 411954      \\
        \hline
    \end{tabular}
\end{table}


Resulta interesante apreciar que los paquetes del top siguen siendo prácticamente los mismos
pese al notable incremento del impacto y que además esta métrica se relaciona bastante con el
Pagerank a nivel de paquete.

\begin{figure}[h!]
    \begin{center}
        \includegraphics[width=1\textwidth]{img/pypi/librariesio_impact_distribution.png}
        \caption{Distribución del impacto en la red de libraries.io (2020)}
        \label{fig:Distribución del impacto en la red de libraries.io}
    \end{center}
\end{figure}

A la vista de la distribución del impacto en la red de libraries.io \ref{fig:Distribución del impacto en la red de libraries.io}, se observa un patrón que se
asemeja al comportamiento de la distribución del grado de salida (\textit{out degree}). Se puede notar
que existen numerosos paquetes con un impacto bajo, y la única explicación plausible es que estos paquetes
no son ampliamente utilizados como dependencias en otros paquetes.

Es común que la mayoría de los paquetes tengan un impacto relativamente bajo, en el rango de alrededor
de 10 paquetes. A medida que aumentamos el valor del impacto, la frecuencia de paquetes con un impacto
alto disminuye significativamente.

Este patrón sugiere que la mayoría de los paquetes en la red de libraries.io no tienen una influencia
crítica en las dependencias y, por lo tanto, su fallo o defecto tendría un impacto limitado en el
sistema en general. Sin embargo, se
identifican ciertos paquetes cuyo impacto es notablemente mayor, lo cual indica que son cruciales
y tienen una influencia significativa en las dependencias.

\begin{figure}[h!]
    \begin{center}
        \includegraphics[width=1\textwidth]{img/pypi/scraped_impact_distribution.png}
        \caption{Distribución del impacto en la red scraped (2023)}
        \label{fig:Distribución del impacto en la red scraped}
    \end{center}
\end{figure}

Al evaluar el estado actual de la red, se observan cambios significativos en la distribución del
impacto, que muestra similitudes con la distribución del grado de salida (\textit{out degree}),
aunque también presenta variaciones y la formación de grupos con tendencias similares.

En particular, se ha observado un considerable aumento en el impacto general de los paquetes en la
red. Ahora se identifica la presencia de un número considerable de paquetes con un impacto del orden
de 10², lo cual indica que su influencia en las dependencias ha aumentado
significativamente.

Además, se distingue un segundo grupo más reducido de paquetes con un impacto alto, en el orden de 10000.
Este grupo de paquetes merece especial atención debido a su impacto significativo
en la red de dependencias.

Asimismo, se ha identificado otro grupo no existente anteriormente que resulta notable debido a
su impacto elevado.


Al analizar el incremento del impacto en la red de dependencias, se observa una distinción entre
dos casos. Por un lado, hay paquetes que han experimentado una disminución en su nivel de impacto
o han mantenido un grado de \emph{vulnerabilidad} estable a lo largo del tiempo. Por otro lado,
existen paquetes que han experimentado un aumento significativo en su impacto.

Es notable que estos paquetes que han experimentado un aumento considerable en su impacto
están interrelacionados y forman parte de componentes altamente conectados dentro de la red.
Esta \emph{interconexión} entre los paquetes permite un aumento grupal del impacto, amplificando
así la magnitud de las consecuencias en caso de fallos o defectos.

\begin{table}[h!]
    \centering
    \caption{Comparación del incremento del impacto para de libraries.io (2020) y scraped (2023)}
    \begin{tabular}{|c|c|c|c|}
        \hline
        \textbf{Paquete} & \textbf{libraries.io} & \textbf{scraped} & \textbf{incremento} \\
        \hline
        numpy            & 7396                  & 448177           & 440781              \\
        importlib        & 24                    & 420861           & 420837              \\
        colorama         & 1795                  & 416663           & 414868              \\
        typing           & 2434                  & 417287           & 414853              \\
        matplotlib       & 748                   & 414520           & 413772              \\
        Cython           & 75                    & 412181           & 412106              \\
        chardet          & 1707                  & 413067           & 411360              \\
        BeautifulSoup4   & 75                    & 411391           & 411316              \\
        genshi           & 5                     & 411290           & 411285              \\
        cssselect        & 260                   & 411490           & 411230              \\
        \hline
    \end{tabular}
\end{table}


A la vista del top de incremento del impacto se aprecia similitud entre los paquetes seleccionados,
los cuales resultan muy familiares para los desarrolladores que usamos el lenguaje Python ya que
son paquetes muy usados en casi todo tipo de software.

Si analizamos el decremento se puede ver que no es tan acentuado como el incremento, no
podemos sacar muchas conclusiones de ello más que estos paquetes han disminuido el número de
dependencias transitivas que poseían, simplemente quedémonos con observar la tendencia y ver qué
paquetes han sido los más afectados. \ref{tab:Disminución del impacto en libraries.io y scraped}

\begin{table}[h!]
    \centering
    \label{tab:Disminución del impacto en libraries.io y scraped}
    \begin{tabular}{|c|c|c|c|}
        \hline
        \textbf{Paquete}              & \textbf{librariesio} & \textbf{scraped} & \textbf{incremento} \\
        \hline
        python-dateutil               & 9825                 & 0                & -9825               \\
        importlib-metadata            & 4677                 & 0                & -4677               \\
        backports.functools-lru-cache & 1944                 & 0                & -1944               \\
        async-timeout                 & 1828                 & 0                & -1828               \\
        asn1crypto                    & 2503                 & 817              & -1686               \\
        oslo.i18n                     & 1503                 & 0                & -1503               \\
        futures                       & 2277                 & 816              & -1461               \\
        oslo.utils                    & 1242                 & 0                & -1242               \\
        singledispatch                & 1213                 & 87               & -1126               \\
        pyasn1-modules                & 1101                 & 0                & -1101               \\
        \hline
    \end{tabular}
    \caption{Disminución del impacto en libraries.io (2020) y scraped (2023)}
\end{table}




\subsubsection{Reach}

La métrica llamada \emph{Reach} , que se refiere a la vulnerabilidad
frente a fallos en una red de paquetes, se utiliza para medir el alcance de los paquetes afectados
por un fallo aleatorio en la red. Se define como la media aritmética del alcance de los nodos
en la red.

La vulnerabilidad de la red se cuantifica al calcular el número esperado de paquetes comprometidos
por un fallo aleatorio, asumiendo que las probabilidades de fallo son independientes y siguen
una distribución uniforme.

A nivel de paquete se refiere al número de paquetes que se verían afectados por un fallo en un
paquete o alguna de sus dependencias transitivas, es decir el número de sucesores transitivos
de un paquete más 1.\footnote{El alcance a nivel de paquete se
    define como el número de paquetes que se verían afectados por un fallo en un paquete o alguna
    de sus dependencias transitivas.}.


\begin{figure}[h!]
    \begin{center}
        \includegraphics[width=1\textwidth]{img/pypi/top_librariesio_reach_evolution.png}
        \caption{Top Reach en libraries.io (2020)}
    \end{center}
\end{figure}


Bajo esta definición, al analizar el top 20 de paquetes con el mayor \textit{Reach} en una red compuesta por aproximadamente 40000 nodos, resulta llamativo observar que algunos paquetes
presentan un \textit{Reach} tan elevado. Además, es importante destacar que estos paquetes se
encuentran entre los más populares y utilizados en Python, lo cual justifica el valor alcanzado.
Sin embargo, desde el punto de vista de la vulnerabilidad de la red, resulta preocupante, ya que
un fallo en alguno de estos paquetes representaría un peligro significativo.

En relación a estos paquetes, en el estado actual de la red, resulta sorprendente el incremento
que han experimentado en su alcance. Este incremento puede ser explicado por el crecimiento en el
número de nodos de la red. Estos valores destacados para los paquetes en cuestión nos proporcionan
una idea clara de la vulnerabilidad que introduce su presencia en la red. Si consideramos que
\textit{PyPI} actualmente cuenta con aproximadamente \textit{200000} paquetes, un fallo en alguno de estos paquetes
que se encuentran en el \textit{top} del ranking tendría un impacto \textit{considerable} en la integridad de la red.


\begin{figure}[h!]
    \begin{center}
        \includegraphics[width=1\textwidth]{img/pypi/top_scraped_reach_evolution.png}
        \caption{Top Reach en scraped (2023)}
    \end{center}
\end{figure}


Si comparamos las distribuciones de \textit{reach} para los dos conjuntos de datos, podemos ver que tienen una
tendencia similar a la distribución de \textit{grado de salida}.

\begin{figure}[h!]
    \begin{center}
        \includegraphics[width=0.8\textwidth]{img/pypi/librariesio_reach_distribution.png}
        \caption{Distribución del Reach en libraries.io (2020)}
    \end{center}
\end{figure}

\begin{figure}[h!]
    \begin{center}
        \includegraphics[width=0.8\textwidth]{img/pypi/scraped_reach_distribution.png}
        \caption{Distribución del Reach en scraped (2023)}
    \end{center}
\end{figure}

El nuevo conjunto de datos muestra la existencia de tres grupos distintos. El primer grupo presenta
un valor de \textit{Reach} bajo, lo que indica que no hay un nivel significativo de vulnerabilidad. En el
segundo grupo, el valor de \textit{Reach} se sitúa en el orden de \textit{10000}, lo cual representa un riesgo mayor.
Aunque el número de paquetes pertenecientes a este grupo no es excesivamente alto, es importante
tenerlo en cuenta debido a su nivel de vulnerabilidad. Por último, el tercer grupo se caracteriza
por tener un valor de \textit{Reach} muy alto.

Según mi interpretación, estos dos grupos de alto Reach representan nodos pertenecientes a componentes
fuertemente conexos y son en los que habría que poner el foco para proteger la estabilidad de la red.

\begin{figure}[h!]
    \begin{center}
        \includegraphics[width=0.8\textwidth]{img/pypi/reach_increment.png}
        \caption{Incremento del Reach (2020-2023)}
    \end{center}
\end{figure}

En relación al incremento del Reach, se pueden identificar dos tendencias distintas. En la primera
tendencia, se observa que el Reach se mantiene relativamente estable, con fluctuaciones dentro de un
rango aproximado de \textit{±15,000}. Por otro lado, el segundo grupo exhibe un notable aumento en el valor
del Reach. Un fallo en un paquete perteneciente a este grupo podría generar graves problemas en la red.
Este incremento puede ser atribuido al crecimiento en el número de nodos de la red. Como resultado de
este crecimiento, han surgido dependencias transitivas que han contribuido significativamente a este
aumento considerable en el Reach.

\subsubsection{Componente fuertemente conexo}

En un \textit{componente fuertemente conexo}, todos los nodos están directa o indirectamente conectados
entre sí. No importa si los caminos son directos o implican múltiples pasos a través de otros nodos,
lo fundamental es que existe una ruta dirigida desde cualquier nodo al resto de los nodos del componente.

Bajo la red de libraries.io no se identifican componentes fuertemente conexos. Esto se debe principalmente
al tamaño de la red, que es relativamente pequeño. Sin embargo, en el nuevo conjunto de datos, se ve claramente
la existencia de componentes fuertemente conexos. \ref{table:scc}

\begin{table}
    \begin{tabular}{|c|c|c|c|c|c|c|c|}
        \hline
        \textbf{Size} & \textbf{Avg}    & \textbf{Density} & \textbf{Diameter} & \textbf{Clustering}  & \textbf{Transitive}   \\
                      & \textbf{degree} &                  &                   & \textbf{coefficient} & \textbf{dependencies} \\
        \hline
        283           & 8.890           & 0.015            & 14                & 0.196                & 206873                \\
        9             & 4.947           & 0.137            & 8                 & 0.358                & 23807                 \\
        8             & 3.000           & 0.214            & 6                 & 0.222                & 6288                  \\
        8             & 5.250           & 0.375            & 3                 & 0.383                & 6784                  \\
        8             & 4.750           & 0.339            & 5                 & 0.498                & 6752                  \\
        6             & 5.000           & 0.500            & 2                 & 0.776                & 4536                  \\
        6             & 4.666           & 0.466            & 3                 & 0.470                & 4452                  \\
        6             & 3.666           & 0.366            & 3                 & 0.448                & 4440                  \\
        5             & 2.800           & 0.350            & 4                 & 0.000                & 3770                  \\
        5             & 6.000           & 0.750            & 2                 & 0.777                & 3855                  \\
        5             & 3.200           & 0.400            & 4                 & 0.386                & 3765                  \\
        5             & 3.600           & 0.450            & 3                 & 0.421                & 3755                  \\
        4             & 3.000           & 0.500            & 3                 & 0.406                & 5276                  \\
        4             & 2.500           & 0.416            & 3                 & 0.000                & 3016                  \\
        4             & 3.000           & 0.500            & 3                 & 0.406                & 3220                  \\
        4             & 3.000           & 0.500            & 3                 & 0.700                & 2988                  \\
        4             & 3.000           & 0.500            & 3                 & 0.406                & 3048                  \\
        4             & 3.000           & 0.500            & 3                 & 0.406                & 3064                  \\
        4             & 4.500           & 0.750            & 2                 & 0.800                & 2948                  \\
        3             & 2.666           & 0.666            & 2                 & 0.000                & 3780                  \\
        \hline
    \end{tabular}
    \caption{Componentes fuertemente conexos más grandes en scraped (2023)}
    \label{table:scc}
\end{table}

A partir de estos datos, se pueden extraer varias conclusiones. Existen componentes fuertemente conexos de diversos
tamaños, desde pequeños hasta muy grandes. Algunos componentes tienen una alta importancia medida por el pagerank
del nodo principal. La densidad y el coeficiente de agrupamiento varían entre los componentes, lo que sugiere diferentes
patrones de conexiones. Además, el número de dependencias transitivas varía ampliamente en funcion del tamaño del componente.

\begin{figure}[h!]
    \begin{center}
        \includegraphics[width=1\textwidth]{img/pypi/scc1_dist.png}
        \caption{Distribucion de grado del mayor componente fuertemente conexo (283 nodos) en scraped (2023)}
    \end{center}
\end{figure}

\begin{figure}[h!]
    \begin{center}
        \includegraphics[width=1.2\textwidth]{img/pypi/scc1.png}
        \caption{Mayor componente fuertemente conexo (283 nodos) en scraped (2023)}
    \end{center}
\end{figure}


\capitulo{6}{Trabajos relacionados}

Los gestores de paquetes de software son herramientas esenciales que facilitan la reutilización y
la construcción eficiente de sistemas de software. Estos gestores permiten a los desarrolladores y
profesionales de la informática acceder a bibliotecas de código previamente desarrolladas, lo que
agiliza el proceso de desarrollo al evitar tener que crear funcionalidades desde cero.

Dado que el software se distribuye a través de estos gestores de paquetes, es crucial para los
investigadores y profesionales tener acceso a datos explícitos sobre las redes de dependencia
de software. Estas \textit{redes de dependencia} están compuestas por las relaciones entre los
diferentes \textit{paquetes de software} y pueden resultar opacas si no se cuenta con información precisa.

Para poder analizar y razonar sobre los ecosistemas y productos de software cada vez más
complejos, los investigadores y profesionales dependen de conjuntos de datos públicos disponibles.
Un ejemplo de ello es el conjunto de datos publicado en \textit{libraries.io}, sin embargo,
lamentablemente ha quedado desatendido y no se ha actualizado con regularidad.

Aquí es donde entra en juego el valor de este trabajo en particular, ya que proporciona un nuevo
conjunto de datos que se centra en los gestores de paquetes \textit{CRAN}, \textit{Bioconductor},
\textit{PyPI} y \textit{npm}. Estos conjuntos de datos actualizados ofrecen una perspectiva más
reciente y completa sobre la información disponible hasta la fecha. Esto es especialmente relevante
para los autores de trabajos relacionados que han utilizado el conjunto de datos de \textit{libraries.io}
en sus investigaciones, ya que ahora tienen la oportunidad de actualizar sus estudios utilizando
esta nueva información más actualizada. Con esto, se puede mejorar la calidad de la investigación
y garantizar que los resultados y conclusiones reflejen con precisión el estado actual de los
ecosistemas de software.

A continuación se presentan algunos de los trabajos relacionados que han utilizado el conjunto de
datos de \textit{libraries.io} en sus investigaciones.
\begin{itemize}
    \item \textbf{On the impact of security vulnerabilities in the npm and rubygems dependency networks}\cite{zerouali2022impact}:
          Este artículo estudia empíricamente las vulnerabilidades de seguridad que afectan a los paquetes npm y RubyGems.
          Analiza cómo y cuándo se descubren y solucionan estas vulnerabilidades y cómo cambia su prevalencia con el tiempo.
          También analiza cómo los paquetes vulnerables exponen a sus dependientes directos e indirectos a vulnerabilidades.
    \item \textbf{Dependency solving is still hard, but we are getting better at it}\cite{abate2020dependency}:
          Este artículo trata sobre la resolución de dependencias, que es un problema difícil (NP-completo) en todos los modelos
          de componentes no triviales debido a versiones mutuamente incompatibles de los mismos paquetes o conflictos de
          paquetes declarados explícitamente.
    \item \textbf{On the usage of JavaScript, Python and Ruby packages in Docker Hub images}\cite{zerouali2021usage}:
          Este artículo analiza empíricamente el uso de paquetes de terceros de JavaScript, Python y Ruby en imágenes de Docker Hub.
          Estudia cuán prevalentes, desactualizados y vulnerables son estos paquetes en imágenes de la comunidad que se basan en
          imágenes base de node, Python y Ruby.
    \item \textbf{Mining single statement bugs at massive scale}\cite{richter2022tssb}:
          Este artículo presenta un enfoque para extraer y analizar bugs de una sola declaración de código fuente de proyectos
          de software.
    \item \textbf{Technical lag of dependencies in major package managers}\cite{stringer2020technical}:
          Este artículo trata sobre el retraso técnico de las dependencias en los principales gestores de paquetes.
          Las bibliotecas de terceros utilizadas por un proyecto (dependencias) pueden quedar fácilmente desactualizadas con el
          tiempo, un fenómeno llamado retraso técnico.
    \item \textbf{Identifying critical projects via pagerank and truck factor}\cite{pfeiffer2021identifying}:
          Este artículo trata sobre la identificación de proyectos críticos a través de PageRank y el factor de eje (Truck Factor).
          Recientemente, el equipo de código abierto de Google presentó el puntaje de criticidad, una métrica para evaluar la
          “influencia e importancia” de un proyecto en un ecosistema a partir de señales específicas del proyecto, como el número
          de dependientes, la frecuencia de confirmación, etc.
    \item \textbf{Identifying versions of libraries used in stack overflow code snippets}\cite{zerouali2021identifying}:
          Este artículo trata sobre la identificación de versiones de bibliotecas utilizadas en fragmentos de código de Stack Overflow.
    \item \textbf{Intertwining Communities: Exploring Libraries that Cross Software Ecosystems}\cite{kannee2023intertwining}:
          Este artículo trata sobre la exploración de bibliotecas que cruzan ecosistemas de software.
\end{itemize}
\capitulo{7}{Conclusiones y Líneas de trabajo futuras}


\section{Conclusiones}

En esta conclusion se pretende dar una introducción al escaneo y análisis de repositorios software, destacando la dificultad asociada debido
a la \emph{elevada cantidad de datos}\footnote{La cantidad de datos almacenados en los repositorios software puede ser enormemente extensa,
    lo que implica desafíos en términos de procesamiento y almacenamiento.} y la necesidad de comprender la temática de los mismos. Se ha de tener en cuenta
también el \emph{tiempo de ejecución}\footnote{El tiempo requerido para realizar un escaneo y análisis exhaustivo de un repositorio puede
    ser considerable debido a la cantidad de operaciones y tareas involucradas.} y el \emph{gasto de recursos}\footnote{El análisis de grandes
    volúmenes de datos puede requerir altos consumos de memoria RAM y espacio en disco, lo que puede afectar el rendimiento general del sistema.}
asociados a esta tarea. La computacion en la nube es una solución a estos problemas, ya que permite el procesamiento de grandes volúmenes de
datos de forma distribuida, escalable y eficiente.

Respecto a la persistencia de los datos, la elección entre una base de datos y archivos CSV para almacenar los datos de las redes de dependencias 
de paquetes depende de varios factores. Si se requiere una estructura de datos más compleja, consultas sofisticadas y escalabilidad a largo plazo, 
una base de datos es la opción más adecuada. Sin embargo, si se trabaja con conjuntos de datos más pequeños, se valora la simplicidad y la portabilidad, 
y no se necesitan capacidades avanzadas de gestión de datos, los archivos CSV pueden ser una solución práctica y eficiente.
En nuestro caso, se ha optado por el uso de archivos CSV debido a la simplicidad de la estructura de datos y ser la estructura usada por OLIVIA para
almacenar los datos de las redes de dependencias de paquetes. No obstante, se ha de tener en cuenta que el uso de archivos CSV puede implicar
problemas, por lo que se recomienda el uso de una base de datos para el almacenamiento de los datos por las ventajas que ofrecen en términos de 
rendimiento y escalabilidad.
Sobre el almacenamiento en GitHub, se ve necesario \emph{particionar los datos} ya que existe una limitación en el tamaño máximo de archivo de 100 MB.

En cuanto a la extracción de datos, se ha de tener en cuenta que la extracción de datos de los repositorios software es una tarea compleja debido
a la gran cantidad de información que se puede extraer de los mismos. Por ello, se ha de tener en cuenta que la extracción de datos debe ser
\emph{flexible y extensible}, ya que se pretende que sea capaz de adaptarse a diferentes repositorios software y a diferentes tipos de análisis.

En cuanto al análisis de los datos, el análisis de los datos es una tarea compleja debido a la naturaleza amplia y compleja de la ciencia de redes. 
Requiere el uso de técnicas y herramientas especializadas, así como un profundo conocimiento y comprensión de los principios de la ciencia de redes. 
Sin embargo, este análisis puede brindar información valiosa sobre las interdependencias y la estructura de los paquetes mas importantes en los repositorios software,
proponemos una vision general de los paquetes mas importantes respecto a distinta metricas de centralidad y aportamos una vision de evolucion de los paquetes
al comparar los resultados de libraries.io con los obtenidos por en este trabajo.

Por último, se propone el diseño de una herramienta capaz de llevar a cabo la recoleccion de datos, gestionando
los posibles inconvenientes que puedan surgir durante el proceso. Se ha de tener en cuenta que el diseño de la herramienta debe ser extensible, 
ya que se pretende que sea capaz de adaptarse a diferentes repositorios software y a diferentes tipos de análisis y debido a su
naturaleza opensource y experimental, se ha realizado un esfuerzo adicional en la documentación del código y en el control de calidad del mismo.


\section{Líneas de trabajo futuras}

\subsection{Mejoras en la herramienta}

Con el objetivo de mejorar la herramienta desarrollada, se identifican diversas áreas de enfoque. 
En primer lugar, se propone realizar mejoras en el diseño y pulir detalles técnicos para optimizar su rendimiento y eficiencia. 
Esto implica revisar y refinar la arquitectura de la herramienta, identificar posibles cuellos de botella y aplicar técnicas de 
optimización que permitan un procesamiento más rápido y una mayor escalabilidad.

Adicionalmente, se propone añadir soporte para nuevos repositorios, con el fin de ampliar la compatibilidad de la herramienta y 
permitir su aplicación en una variedad de entornos y plataformas. Esto implica adaptar la herramienta para interactuar con 
diferentes sistemas de control de versiones y repositorios, asegurando la interoperabilidad y facilitando su adopción por 
parte de un mayor número de usuarios.

Por último, es necesario replantearse la persistencia de datos. Esto implica evaluar la forma en que se almacenan y gestionan 
los datos recolectados durante el análisis de los repositorios software. Se pueden considerar opciones como el uso de bases de 
datos para un almacenamiento más eficiente y la implementación de mecanismos de respaldo y recuperación de datos para garantizar
 su integridad y disponibilidad a largo plazo.

\subsection{Análisis de los datos}

Se propone realizar un análisis más exhaustivo a partir de los datos recolectados. Esto implica la aplicación de técnicas de análisis
de redes complejas para identificar patrones y tendencias en las redes de dependencias de paquetes. Ahí entra la comunidad científica, que puede
aportar una visión más amplia y profunda de los datos que aportamos en este trabajo, así como una mayor comprensión de los mismos. Esto permitirá 
obtener información con la que actualizar sus investigaciones continuar con el estudio de las redes de dependencias de paquetes.

\subsection{herramienta de visualización}

Seria un punto interesante el desarrollo de una herramienta de visualización que permita representar gráficamente las redes de dependencias
de paquetes y el calculo de metricas asociadas. Esto implica el diseño e implementación de una interfaz gráfica de usuario que permita la 
interacción con la herramienta y la visualización de los resultados del análisis de los repositorios software. Esto permitirá a los usuarios
explorar y comprender mejor las redes de dependencias de paquetes, así como identificar patrones y tendencias en los datos.



\bibliographystyle{plain}
\bibliography{bibliografia}

\end{document}

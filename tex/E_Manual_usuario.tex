\apendice{Documentación de usuario}

\section{Introducción}

Se detallan a continuación las referencias necesarias para la instalación y utilización de la biblioteca.

\section{Requisitos de usuarios}

Para el correcto funcionamiento del sistema, se establecen una serie de requisitos indispensables.
En primer lugar, se requiere disponer de una \textit{conexión a internet} estable, lo cual permitirá el
acceso y uso de las funcionalidades ofrecidas.

Se necesita instalar las \textit{dependencias} necesarias de la biblioteca a partir del archivo \textit{requirements.txt},
mediante la utilización de un gestor de paquetes como \textit{pip}.

Adicionalmente, para garantizar el funcionamiento, se requiere disponer de \textit{Python} en su versión
3.8 o superior, lo cual asegurará una compatibilidad adecuada con el sistema y sus funcionalidades.

Cabe destacar que en caso de desechar la ejecución de los \textit{notebooks}, es posible que se deban
satisfacer ciertas dependencias adicionales a la biblioteca principal. Estas dependencias, aunque opcionales,
deberán ser satisfechas individualmente por cada usuario, según sus necesidades y preferencias específicas.

\section{Instalación}

El repositorio presenta una estructura bien definida que establece las ubicaciones de los recursos utilizados por los cuadernos,
scripts y otros elementos. Se recomienda encarecidamente respetar esta estructura durante el uso del repositorio, ya que esto
asegurará un funcionamiento óptimo y coherente.

Sin embargo, es importante destacar que existe la posibilidad de personalizar la ubicación de la carpeta de código de la
biblioteca (\textit{olivia\_finder/olivia\_finder}) si se configura correctamente la variable de entorno \texttt{PYTHONPATH}
para que apunte a la biblioteca. Esto permite una mayor flexibilidad en el desarrollo y la organización del proyecto.

A continuación se muestra una propuesta de estructura de directorios que podría ser utilizada en un entorno de desarrollo:

\begin{verbatim}
    .
    |-- olivia_finder               # Biblioteca
    |   |-- config.ini
    |   |-- data_source
    |   |-- __init__.py
    |   |-- myrequests
    |   |-- package_manager.py
    |   |-- package.py
    |   `-- utilities
    |-- requirements.txt            # Dependencias
    `-- test_package_manager.py     # Script de demostración
    
\end{verbatim}

En esta estructura, el directorio principal del proyecto contiene la carpeta \textit{olivia\_finder}, que incluye
los archivos y módulos relevantes para la biblioteca. Además, se encuentra el archivo \textit{config.ini} para la configuración.

Además, se incluyen los archivos \textit{requirements.txt} y \textit{test\_package\_manager.py}, que son de utilidad
para gestionar las dependencias y realizar una demostración básica de funcionamiento de la biblioteca, respectivamente.

\section{Manual del usuario}

Con el fin de adquirir una comprensión más profunda sobre las capacidades y aplicaciones de la biblioteca,
se sugiere acceder a los notebooks de demostración de funcionalidad que han sido preparados:


\begin{itemize}
    \item \textbf{Olivia Finder - Data manipulation.ipynb}\footnote{\url{https://github.com/dab0012/olivia-finder/blob/master/notebooks/olivia_finder/Olivia\%20Finder\%20-\%20Data\%20manipulation.ipynb}}:
          Este cuaderno de demostración aborda la manipulación de datos y presenta diversas funcionalidades ofrecidas por la biblioteca.
    \item \textbf{Olivia Finder - Github repository example.ipynb}\footnote{\url{https://github.com/dab0012/olivia-finder/blob/master/notebooks/olivia_finder/Olivia\%20Finder\%20-\%20Github\%20repository\%20example.ipynb}}:
          En este cuaderno, se proporciona un ejemplo práctico de cómo utilizar la biblioteca Olivia Finder para interactuar con repositorios de GitHub y extraer información relevante.
    \item \textbf{Olivia Finder - Usage.ipynb}\footnote{\url{https://github.com/dab0012/olivia-finder/blob/master/notebooks/olivia_finder/Olivia\%20Finder\%20-\%20Usage.ipynb}}:
          Este cuaderno de uso brinda una visión general de las funcionalidades de la biblioteca.
\end{itemize}

\newpage

A continuación, se expone un fragmento de código en lenguaje Python que ejemplifica un procedimiento
para la adquisición de datos provenientes del repositorio de paquetes \textit{CRAN}.

\begin{small}
    \begin{verbatim}
    # Add the environment variable OLIVIA_FINDER_CONFIG_FILE_PATH
    import os
    os.environ['OLIVIA_FINDER_CONFIG_FILE_PATH'] = "olivia_finder/config.ini"
    
    import tqdm
    from olivia_finder.package_manager import PackageManager
    from olivia_finder.data_source.repository_scrapers.cran import CranScraper
    
    # Test cran package manager
    cran_pm_scraper = PackageManager(data_sources=[CranScraper()])
    
    # Test fetch package names
    test_packages = cran_pm_scraper.fetch_package_names()[300:350]
    
    # Test fetch packages
    progress_bar = tqdm.tqdm(total=len(test_packages))
    cran_pm_scraper.fetch_packages(
        test_packages, progress_bar=progress_bar, extend=True
    )
    
    # Export to csv
    df = cran_pm_scraper.export_dataframe(full_data=True)
    df.to_csv("test.csv", index=False)
    
    print(df.head(50))
    \end{verbatim}
\end{small}




\capitulo{1}{Introducción}


\section{Las redes de dependencias de paquetes de software}

Una red de dependencias de paquetes software es un conjunto de conexiones entre diferentes paquetes o bibliotecas utilizadas en el desarrollo de software. Cada paquete depende de otros para funcionar correctamente, formando una red interdependiente.

En este contexto, cada paquete tiene requisitos específicos que deben cumplirse para que funcione correctamente. Estos requisitos son satisfechos por otros paquetes, creando conexiones que permiten que el software se comporte como se espera. Si alguna de estas conexiones se rompe o no se cumple, puede producirse un fallo en el funcionamiento del software.

La red de dependencias puede ser muy compleja, ya que un solo paquete puede depender de múltiples otros paquetes, y estos, a su vez, pueden tener sus propias dependencias. Es como un tejido intrincado en el que cada hilo está conectado con otros, formando una red interconectada.

\section{Los repositorios de paquetes de software desde la perspectiva de la teoría de grafos}

Al estudiar las dependencias de paquetes como grafos, obtenemos una visión más clara de las complejas relaciones que existen en nuestros proyectos. Podemos identificar los paquetes esenciales que sostienen todo el sistema y comprender cómo un cambio en uno de ellos puede afectar a otros. Esto nos ayuda a tomar decisiones informadas y anticiparnos a posibles problemas.

Además, la teoría de grafos nos permite detectar dependencias redundantes o ciclos indeseados en nuestra red de paquetes. Podemos evaluar la estabilidad y mantenibilidad del proyecto, analizando el impacto de agregar o eliminar un paquete en toda la red. Esta comprensión más profunda nos ayuda a optimizar y fortalecer nuestro código.

Afortunadamente existen herramientas y bibliotecas que nos proporcionan esta informacion de las redes. Con ellas, y usando el enfoque de la teoria de grafos, podemos ver de una forma más clara cómo se entrelazan los paquetes entre sí y cómo se relacionan con el resto de la red.

La aplicación de la teoría de grafos a las redes de dependencias de paquetes software nos permite tomar decisiones más informadas, comprender mejor las implicaciones de las dependencias y colaborar de manera más efectiva con nuestro equipo de desarrollo. Es una forma poderosa de optimizar y mejorar nuestros proyectos y mantener un ecosistema saludable en los gestores de paquetes.

\section{OLIVIA: Open-source Library Indexes Vulnerability Identification and Analysis}

OLIVIA, desarrollada por el alumno Daniel Setó Rey como parte de su Trabajo de Fin de Grado en la Universidad de Burgos y tutorizada por los profesores Carlos López Nozal y Jose Ignacio Santos Martín en 2021, es una herramienta de código abierto que se centra en la identificación y análisis de defectos en bibliotecas de software desde la perspectiva de la teoría de grafos. Estos defectos pueden provocar errores funcionales, problemas de rendimiento e incluso problemas de seguridad. Para los desarrolladores, comprender completamente el riesgo es complicado, ya que solo importan explícitamente una pequeña parte de las dependencias utilizadas en sus proyectos.

OLIVIA utiliza un enfoque basado en la vulnerabilidad de la red de dependencias de los paquetes de software para medir la sensibilidad del repositorio a la introducción aleatoria de defectos. Su objetivo es contribuir a la comprensión de los mecanismos de propagación de defectos en el software y estudiar estrategias factibles de protección.

En la actualidad, OLIVIA está en proceso de ser publicado a nivel académico. Después de su desarrollo como proyecto de fin de carrera, se están realizando los esfuerzos necesarios para compartir sus resultados y contribuciones con la comunidad científica. Esta publicación permitirá que otros investigadores y profesionales del campo accedan a esta herramienta y se beneficien de su enfoque innovador en la identificación y análisis de vulnerabilidades en las bibliotecas de software. Además, sentará las bases para futuros avances en la comprensión de los mecanismos de propagación de defectos y la implementación de estrategias efectivas de protección en el desarrollo de software.

En este trabajo de fin de grado, no se profundiza en el modelo matemático subyacente de OLIVIA. Sin embargo, es importante tener conocimientos básicos de la teoría de grafos para comprender su funcionalidad. La teoría de grafos proporciona un marco conceptual para comprender las interconexiones y relaciones entre los paquetes de software, así como la propagación de posibles defectos en el ecosistema. Aunque no se abordan los aspectos matemáticos en detalle, tener una comprensión general de los grafos nos ayuda a apreciar y comprender mejor la utilidad y las implicaciones de OLIVIA en la identificación y análisis de vulnerabilidades en las bibliotecas de software.



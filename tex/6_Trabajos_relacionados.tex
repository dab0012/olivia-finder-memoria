\capitulo{6}{Trabajos relacionados}

Los gestores de paquetes de software son herramientas esenciales que facilitan la reutilización y
la construcción eficiente de sistemas de software. Estos gestores permiten a los desarrolladores y
profesionales de la informática acceder a bibliotecas de código previamente desarrolladas, lo que
agiliza el proceso de desarrollo al evitar tener que crear funcionalidades desde cero.

Dado que el software se distribuye a través de estos gestores de paquetes, es crucial para los
investigadores y profesionales tener acceso a datos explícitos sobre las redes de dependencia
de software. Estas \textit{redes de dependencia} están compuestas por las relaciones entre los
diferentes \textit{paquetes de software} y pueden resultar opacas si no se cuenta con información precisa.

Para poder analizar y razonar sobre los ecosistemas y productos de software cada vez más
complejos, los investigadores y profesionales dependen de conjuntos de datos públicos disponibles.
Un ejemplo de ello es el conjunto de datos publicado en \textit{libraries.io}, sin embargo,
lamentablemente ha quedado desatendido y no se ha actualizado con regularidad.

Aquí es donde entra en juego el valor de este trabajo en particular, ya que proporciona un nuevo
conjunto de datos que se centra en los gestores de paquetes \textit{CRAN}, \textit{Bioconductor},
\textit{PyPI} y \textit{npm}. Estos conjuntos de datos actualizados ofrecen una perspectiva más
reciente y completa sobre la información disponible hasta la fecha. Esto es especialmente relevante
para los autores de trabajos relacionados que han utilizado el conjunto de datos de \textit{libraries.io}
en sus investigaciones, ya que ahora tienen la oportunidad de actualizar sus estudios utilizando
esta nueva información más actualizada. Con esto, se puede mejorar la calidad de la investigación
y garantizar que los resultados y conclusiones reflejen con precisión el estado actual de los
ecosistemas de software.

A continuación se presentan algunos de los trabajos relacionados que han utilizado el conjunto de
datos de \textit{libraries.io} en sus investigaciones.
\begin{itemize}
    \item \textbf{On the impact of security vulnerabilities in the npm and rubygems dependency networks}\cite{zerouali2022impact}:
          Este artículo estudia empíricamente las vulnerabilidades de seguridad que afectan a los paquetes npm y RubyGems.
          Analiza cómo y cuándo se descubren y solucionan estas vulnerabilidades y cómo cambia su prevalencia con el tiempo.
          También analiza cómo los paquetes vulnerables exponen a sus dependientes directos e indirectos a vulnerabilidades.
    \item \textbf{Dependency solving is still hard, but we are getting better at it}\cite{abate2020dependency}:
          Este artículo trata sobre la resolución de dependencias, que es un problema difícil (NP-completo) en todos los modelos
          de componentes no triviales debido a versiones mutuamente incompatibles de los mismos paquetes o conflictos de
          paquetes declarados explícitamente.
    \item \textbf{On the usage of JavaScript, Python and Ruby packages in Docker Hub images}\cite{zerouali2021usage}:
          Este artículo analiza empíricamente el uso de paquetes de terceros de JavaScript, Python y Ruby en imágenes de Docker Hub.
          Estudia cuán prevalentes, desactualizados y vulnerables son estos paquetes en imágenes de la comunidad que se basan en
          imágenes base de node, Python y Ruby.
    \item \textbf{Mining single statement bugs at massive scale}\cite{richter2022tssb}:
          Este artículo presenta un enfoque para extraer y analizar bugs de una sola declaración de código fuente de proyectos
          de software.
    \item \textbf{Technical lag of dependencies in major package managers}\cite{stringer2020technical}:
          Este artículo trata sobre el retraso técnico de las dependencias en los principales gestores de paquetes.
          Las bibliotecas de terceros utilizadas por un proyecto (dependencias) pueden quedar fácilmente desactualizadas con el
          tiempo, un fenómeno llamado retraso técnico.
    \item \textbf{Identifying critical projects via pagerank and truck factor}\cite{pfeiffer2021identifying}:
          Este artículo trata sobre la identificación de proyectos críticos a través de PageRank y el factor de eje (Truck Factor).
          Recientemente, el equipo de código abierto de Google presentó el puntaje de criticidad, una métrica para evaluar la
          “influencia e importancia” de un proyecto en un ecosistema a partir de señales específicas del proyecto, como el número
          de dependientes, la frecuencia de confirmación, etc.
    \item \textbf{Identifying versions of libraries used in stack overflow code snippets}\cite{zerouali2021identifying}:
          Este artículo trata sobre la identificación de versiones de bibliotecas utilizadas en fragmentos de código de Stack Overflow.
    \item \textbf{Intertwining Communities: Exploring Libraries that Cross Software Ecosystems}\cite{kannee2023intertwining}:
          Este artículo trata sobre la exploración de bibliotecas que cruzan ecosistemas de software.
\end{itemize}
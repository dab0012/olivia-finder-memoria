\apendice{Documentación técnica de programación}

\section{Introducción}
Se recoge en este apartado la información más relevante para la extensión
o adaptación del código de la biblioteca.

\section{Estructura de directorios}
El proyecto de software está disponible en el repositorio público \url{https://github.com/dab0012/olivia-finder}.
Este repositorio incluye los objetos de persistencia, los datos extraídos en los análisis y todo el
conjunto de scripts y demás material utilizado. Sin embargo, si lo que le interesa son los datos generados de las redes de dependencias,
es recomendable acceder a ellos desde los siguientes mirrors en Zenodo: \texttt{xxxxxxxxxxx}.

\subsection{Estructura de directorios del repositorio}

El repositorio alberga una amplia gama de contenido, que incluye tanto la biblioteca Olivia-Finder que hemos
desarrollado como la biblioteca OLIVIA realizada en el anterior Trabajo de Fin de Grado. Además, se encuentra
disponible documentación detallada sobre ambas bibliotecas, que proporciona información exhaustiva sobre su
funcionamiento y características.

Dentro del repositorio, también se pueden encontrar una serie de notebooks que ofrecen demostraciones interactivas
de la funcionalidad de las bibliotecas. Estos notebooks permiten a los usuarios explorar y experimentar con los
datos generados por las redes que hemos creado, lo que facilita la comprensión y el análisis de los resultados
obtenidos.

Además del código y la documentación, el repositorio cuenta con una variedad de material auxiliar, como scripts
y funcionalidades adicionales. Estos recursos complementarios ofrecen soporte adicional para aquellos interesados
en ampliar o personalizar las funcionalidades de las bibliotecas.

\begin{verbatim}
    olivia_finder
    |-- docs
    |-- notebooks
    |   |-- common
    |   |-- olivia
    |   |-- olivia_finder
    |   |-- resources
    |   `-- results
    |       |-- csv_datasets
    |       |-- network_models
    |       |-- package_lists
    |       `-- package_managers
    |-- olivia
    |   |-- docs
    |   `-- olivia
    `-- olivia_finder
        |-- diagrams
        `-- olivia_finder
    \end{verbatim}
    

Sobre la raiz del repositorio, se encuentra el directorio \textit{docs} contiene la documentación de la biblioteca, mientras que el directorio \textit{notebooks}
contiene los notebooks que ofrecen demostraciones interactivas de las bibliotecas.

Dentro de \textit{notebooks}, el directorio \textit{common} contiene notebooks que ofrecen funcionalidades comunes
a ambas bibliotecas, mientras que los directorios \textit{olivia} y \textit{olivia\_finder} contienen notebooks
que ofrecen funcionalidades específicas de cada biblioteca. 
El directorio \textit{resources} contiene los algun scripts
y recursos adicionales de utilidad. Por último, el directorio \textit{results} contiene los datos generados por las
redes de dependencias, que se encuentran organizados en subdirectorios según su tipo.

En la raiz del repositorio, se encuentran los directorios \textit{olivia} y \textit{olivia\_finder}, que contienen
las bibliotecas OLIVIA y Olivia-Finder, respectivamente. Dentro de cada uno de estos directorios, se encuentra 
el directorio \textit{olivia} o \textit{olivia\_finder}, que contienen el código de las bibliotecas.


\section{Manual del programador}

\section{Compilación, instalación y ejecución del proyecto}

\section{Pruebas del sistema}

\apendice{Documentación técnica de programación}

\section{Introducción}
Se recoge en este apartado la información más relevante para la extensión
o adaptación del código de la biblioteca.

\section{Estructura de directorios}
El proyecto de software está disponible en el repositorio público\footnote{\url{https://github.com/dab0012/olivia-finder}}.
Este repositorio incluye los objetos de persistencia, los datos extraídos en los análisis y todo el
conjunto de scripts y demás material utilizado. Sin embargo, si lo que le interesa son los datos generados de las redes de dependencias,
es recomendable acceder a ellos desde los siguientes mirrors en \textit{Zenodo}.\footnote{\url{https://zenodo.org/record/8095863}}.

EL conjunto de datos completo usado en los experimentos esta disponble en \textit{Kaggle}.\footnote{\url{https://www.kaggle.com/datasets/danielalonsob/dependency-networks}}

\subsection{Estructura de directorios del repositorio}

El repositorio alberga una amplia gama de contenido, que incluye tanto la biblioteca Olivia-Finder que hemos
desarrollado como la biblioteca \textit{OLIVIA} realizada en el anterior Trabajo de Fin de Grado. Además, se encuentra
disponible documentación detallada sobre ambas bibliotecas, que proporciona información exhaustiva sobre su
funcionamiento y características.

Dentro del repositorio, también se pueden encontrar una serie de notebooks que ofrecen demostraciones interactivas
de la funcionalidad de las bibliotecas. Estos notebooks permiten a los usuarios explorar y experimentar con los
datos generados por las redes que hemos creado, lo que facilita la comprensión y el análisis de los resultados
obtenidos.

Además del código y la documentación, el repositorio cuenta con una variedad de material auxiliar, como scripts
y funcionalidades adicionales. Estos recursos complementarios ofrecen soporte adicional para aquellos interesados
en ampliar o personalizar las funcionalidades de las bibliotecas.

\begin{verbatim}
    olivia_finder
    |-- docs
    |-- notebooks
    |   |-- common
    |   |-- olivia
    |   |-- olivia_finder
    |   |-- resources
    |   `-- results
    |       |-- csv_datasets
    |       |-- network_models
    |       |-- package_lists
    |       `-- package_managers
    |-- olivia
    |   |-- docs
    |   `-- olivia
    `-- olivia_finder
        |-- diagrams
        `-- olivia_finder
    \end{verbatim}


Sobre la raiz del repositorio, se encuentra el directorio \textit{docs} contiene la documentación de la biblioteca, mientras que el directorio \textit{notebooks}
contiene los notebooks que ofrecen demostraciones interactivas de las bibliotecas.

Dentro de \textit{notebooks}, el directorio \textit{common} contiene notebooks que ofrecen funcionalidades comunes
a ambas bibliotecas, mientras que los directorios \textit{olivia} y \textit{olivia\_finder} contienen notebooks
que ofrecen funcionalidades específicas de cada biblioteca.
El directorio \textit{resources} contiene los algun scripts
y recursos adicionales de utilidad. Por último, el directorio \textit{results} contiene los datos generados por las
redes de dependencias, que se encuentran organizados en subdirectorios según su tipo.

En la raiz del repositorio, se encuentran los directorios \textit{olivia} y \textit{olivia\_finder}, que contienen
las bibliotecas \textit{OLIVIA} y Olivia-Finder, respectivamente. Dentro de cada uno de estos directorios, se encuentra
el directorio \textit{olivia} o \textit{olivia\_finder}, que contienen el código de las bibliotecas.


\section{Manual del programador}

\subsection{Entrorno de desarrollo}

El proyecto puede descargarse directamente o clonarse desde el repositorio \footnote{\url{https://github.com/dab0012/olivia-finder}}
con \texttt{git}. Ha sido desarrollado en \texttt{python3.11} buscando la compatibilidad con \texttt{python3.8} por
ser la máxima versión soportada por la librería \textit{OLIVIA}.
Las dependencias de la biblioteca se especifican en el archivo \texttt{requirements.txt}:


\textbf{Manejo de datos:}
\begin{itemize}
    \item \texttt{pandas}
    \item \texttt{networkx}
\end{itemize}

\textbf{Obtención de datos:}
\begin{itemize}
    \item \texttt{requests}
    \item \texttt{BeautifulSoup4}
    \item \texttt{selenium}
    \item \texttt{pybraries}
\end{itemize}

\textbf{Utilidad extra:}
\begin{itemize}
    \item \texttt{tqdm}
    \item \texttt{typing\_extensions}
\end{itemize}

En el caso de querer hacer un uso combinado de la biblioteca \textit{Olivia Finder} y \textit{OLIVIA}, se recomienda el uso de entornos
virtuales de Python para facilitar el despliegue. Se debe crear un entorno de la versión de \textit{Python3.8} e instalar los
requisitos de la biblioteca \textit{OLIVIA} en primer lugar debido a la necesidad de esta de versiones concretas de sus
dependencias, en concreto:

\begin{center}

    \texttt{intbitset==2.4.0}

    \texttt{numpy==1.18.5}

    \texttt{networkx==2.4}

\end{center}

La forma más sencilla de instalar los paquetes en el entorno es mediante el gestor de paquetes \texttt{pip} usando el
comando para cada biblioteca:

\begin{center}
    \texttt{pip install -r requirements.txt}
\end{center}

\subsection{Documentación para el programador}

Se recogen a continuación los principales recursos documentales necesarios para el desarrollo o extensión de la biblioteca \textit{Olivia Finder}:

\textbf{Documentación de la API de \textit{Olivia Finder}}:
\begin{itemize}
\item Disponible en la carpeta \textbackslash docs y accesible en línea a través de la página de Github: 

\url{https://dab0012.github.io/olivia-finder}
\end{itemize}

\textbf{Documentación de los paquetes}:
\begin{itemize}
\item Documentación de pandas:

\url{https://pandas.pydata.org/}
\item Documentación de tqdm:
 
\url{https://tqdm.github.io/}
\item Documentación de requests:
 
\url{https://docs.python-requests.org/}
\item Documentación de BeautifulSoup4:
 
\url{https://www.crummy.com/software/BeautifulSoup/bs4/doc/}
\item Documentación de selenium: 
 
\url{https://www.selenium.dev/documentation/en/}
\item Documentación de networkx: 
 
\url{https://networkx.org/documentation/stable/}
\item Documentación de matplotlib: 
 
\url{https://matplotlib.org/stable/contents.html}
\item Documentación de pybraries: 
 
\url{https://pybraries.readthedocs.io/}
\item Documentación de typing\_extensions: 
 
\url{https://typing-extensions.readthedocs.io/}
\end{itemize}

Como anexo, introducimos también la documentación de \textit{OLIVIA}:

\begin{itemize}
\item Documentación de la API: Disponible en la carpeta \textbackslash docs y accesible en línea a través de la página de Github: 

\url{https://dsr0018.github.io/olivia}
\item Documentación de NetworkX 2.4: 

\url{https://networkx.org/documentation/networkx-2.4.D.3}
\item Documentación de NumPy 1.18: 

\url{https://numpy.org/doc/1.18/}
\item Documentación de intbitset: 

\url{https://intbitset.readthedocs.io/en/latest/}
\end{itemize}

\section{Compilación, instalación y ejecución del proyecto}

El código de la biblioteca esta escrito en Python y debido a las caracteristicas de este lenguaje de programación
no requiere un paso de compilación explícita. Dadas por 
satisfechas las dependencias especificadas en \textit{requirements.txt}, los módulos pueden importarse 
en el proyecto actual como cualquier otro módulo Python.

No obstante, determinadas funcionalidades, como aquellas que requieren el uso de \textit{la API de libraries.io}, 
dependen de la \textit{API key} proporcionada por los proveedores del servicio. Esta \textit{API key} 
debe ser configurada en el archivo de configuración \textit{config.ini} en la raíz de la carpeta de 
código de la biblioteca. Desde este mismo archivo, también podemos configurar el sistema de logs de 
la herramienta.

\section{Pruebas del sistema}

Las pruebas del correcto funcionamiento del sistema se establecen mediante una serie de \textit{notebooks de demostración} 
que abarcan tanto la funcionalidad como el uso de la aplicación, los cuales comentamos en la seccion de documentacion de usuario. Estos notebooks permiten verificar exhaustivamente el 
comportamiento del sistema en diferentes escenarios y validar su conformidad con los requisitos establecidos.

Dado que no existe un método de validación preciso que nos permita determinar si el conjunto de datos utilizado es fiel a 
la realidad, se asume inicialmente que los datos son más exactos y actualizados en comparación con los proporcionados por 
\textit{libraries.io}. Esta percepción se basa en el \textit{aumento del número de paquetes} en el período temporal que 
separa ambos conjuntos de datos, así como en la aparente \textit{evolución de las relaciones de dependencia} de manera esperada.

\section{Documentación del código fuente}

La biblioteca Olivia Finder ha sido desarrollada con comentarios docstring en formato 
\textit{numpydoc}\footnote{\url{https://numpydoc.readthedocs.io/en/latest/}}, 
el cual sigue las convenciones de documentación de \textit{NumPy} y es compatible con \textit{Sphinx}\footnote{\url{https://www.sphinx-doc.org/}} y otros generadores 
automáticos de documentación. Estos comentarios docstring proporcionan descripciones detalladas de 
las clases, métodos y funciones, así como información sobre los parámetros, tipos de retorno y ejemplos 
de uso. Esto facilita la comprensión y el uso correcto de la biblioteca por parte de los desarrolladores.

Para convertir los docstrings en otros formatos, como reStructuredText o Markdown, se puede utilizar la 
herramienta Pyment\footnote{\url{https://github.com/dadadel/pyment/blob/master/doc/sphinx/source/pyment.rst}}. \textit{Pyment} es una utilidad de línea de comandos 
que permite extraer y convertir los comentarios docstring de un código fuente Python a diferentes formatos 
de documentación. Su flexibilidad y capacidad de personalización lo convierten en una herramienta útil 
para adaptar la documentación a las necesidades específicas del proyecto.

Por otro lado, la documentación publicada en el repositorio de la biblioteca Olivia Finder ha sido generada 
con \textit{pdoc}\footnote{\url{https://pdoc3.github.io/pdoc/}}. \textit{Pdoc} es una herramienta de generación de documentación 
para proyectos Python que se enfoca en la simplicidad y la facilidad de uso. Al ejecutar el comando:

\begin{center}
    \texttt{pdoc -html -o docs olivia}
\end{center}

se genera la documentación en formato HTML y se almacena en la 
carpeta \texttt{docs}. Esta documentación incluye descripciones de módulos, clases, métodos y funciones,
 así como ejemplos de uso y referencias cruzadas entre elementos de la biblioteca.

Es importante destacar que dentro de la carpeta \texttt{docs} se proporciona el script \texttt{build.sh}, 
el cual automatiza el proceso de generación de la documentación. Al ejecutar este script, se realiza la 
generación de la documentación de forma rápida y sencilla, asegurando la disponibilidad de una documentación 
actualizada y coherente para los usuarios de la biblioteca.
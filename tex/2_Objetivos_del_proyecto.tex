\capitulo{2}{Objetivos del proyecto}

\begin{itemize}
    \item Dar soporte a OLIVIA al proporcionarle los datos necesarios para enriquecer su modelo y seguir una estructura de datos similar que asegure la integración con OLIVIA.
    
    \item Proporcionar datos actualizados que mejoren su rendimiento y precisión, asegurando que OLIVIA esté equipado con los datos más recientes para su análisis y detección de defectos. Al proporcionar esta fuente constante de información actualizada, fortaleceremos la capacidad de OLIVIA para ofrecer resultados precisos y confiables, respaldando así su utilidad en la identificación y análisis de vulnerabilidades en bibliotecas de software.
    
    \item Realizar un análisis de los principales repositorios de software propuestos en OLIVIA, con el fin de buscar estrategias viables para extraer los datos de los paquetes que contienen. A través de este análisis, nos enfocaremos en identificar métodos eficientes y efectivos para obtener la información relevante de cada repositorio, considerando las restricciones y peculiaridades de cada plataforma.
    
    \item Enriquecer mediante la aportación de nuevos datos no disponibles anteriormente y usar fuentes de información adicionales.
    
    \item Publicar los datos obtenidos a través de nuestra investigación para que sirvan como referencia y recurso para investigaciones futuras. Al hacer públicos los datos recopilados en nuestro estudio sobre los repositorios de software, esperamos que otros investigadores puedan aprovecharlos y utilizarlos como base para sus propias investigaciones.
    
    \item Realizar una comparativa de la evolución de los paquetes y el estado de los repositorios tras el paso del tiempo.
    \end{itemize}
\capitulo{2}{Objetivos del proyecto}


\section{Objetivo general}

Este estudio analiza estrategias de extracción de datos de repositorios 
de paquetes, considerando restricciones y características específicas, y busca publicar 
los datos obtenidos como referencia para futuras investigaciones.

Podemos dividir los objetivos del proyecto en estos cuatro puntos:

\begin{itemize}
    \item Realizar un análisis de la viabilidad de las diferentes estrategias de extracción de datos de los repositorios 
    \textit{CRAN, Bioconductor, PyPI y npm} teniendo en cuenta las restricciones y particularidades de cada plataforma. 
    Consideramos aspectos como la disponibilidad de datos, las políticas de acceso y las limitaciones técnicas 
    para determinar la mejor manera de obtener y procesar la información requerida.
    \item Publicar los datos obtenidos con el propósito de que sirvan como referencia y recurso para futuras investigaciones en el campo, fomentando la colaboración en la comunidad científica, permitiendo a otros investigadores utilizarlos como base para nuevos análisis y descubrimientos.
    \item Realizar un análisis de las principales métricas de teoría de grafos utilizando el conjunto de datos disponible para comparar la evolución de los paquetes y evaluar el estado de los repositorios a lo largo del tiempo.
    \item Obtener las métricas propuestas por OLIVIA para el nuevo conjunto de datos y su comparación con los datos expuestos en el Trabajo de Fin de Grado anterior.
    \end{itemize}


\section{Objetivos técnicos}

Para la consecución de los objetivos generales, se han establecido los siguientes objetivos técnicos:

\begin{itemize}
  \item Aplicación de técnicas de \textit{web scraping} para la extracción de datos.
  
  \item Aplicación de técnicas de acceso a datos a través de una \textit{Web API}.
  
  \item Comprensión y aplicación de formatos de datos para almacenar persistentemente una red de dependencias.
  
  \item Medición y análisis de redes de dependencias mediante la aplicación de conceptos teóricos de la biblioteca \textit{networkx}.
  
  \item Acceso a las dependencias de un repositorio en GitHub utilizando la funcionalidad de \textit{grafo de dependencias} de GitHub.
\end{itemize}